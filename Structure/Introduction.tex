% !TEX root = ../BookTemplate.tex
%%%%%%%%%%%%%%%%%%%%%%%%%%%%%%%%%%%%%%%%%%%%%%%%%%%%%%%%%%%%%%%%%%%%%%%%%%%%%%%%%%
\chapter*{}\addcontentsline{toc}{chapter}{\IntroductionName}
\vspace{-1.5cm}
    \begin{minipage}[r]{.95\textwidth}\raggedleft
    \HUGE\bfseries\IntroductionName
    \end{minipage}
\vspace{2cm}

\noindent
Este livro nasce de uma inquietacao. Apos anos de pratica clinica em psiquiatria, acumulei mais perguntas do que respostas. E, curiosamente, quanto mais experiencia adquiri, mais as certezas que me foram ensinadas na formacao pareceram frageis.

\vspace{0.5cm}

A psiquiatria contemporanea vive uma crise silenciosa. Por tras da aparencia de solidez cientifica --- os manuais diagnosticos, os protocolos de tratamento, as diretrizes baseadas em evidencias --- esconde-se uma disciplina que ainda nao resolveu suas questoes fundamentais: O que e doenca mental? Onde termina a variacao humana normal e comeca a patologia? Por que medicamos tanto e conversamos tao pouco?

\vspace{0.5cm}

Nao escrevo este livro para demolir a psiquiatria. Escrevo porque a amo --- e porque acredito que ela pode ser melhor. Cada capitulo que se segue e um convite a reflexao critica sobre aspectos da pratica que tendemos a tomar como dados. Questiono nao para negar, mas para aprofundar. Critico nao para destruir, mas para reconstruir em bases mais solidas.

\vspace{0.5cm}

O leitor encontrara aqui uma psiquiatria que pergunta antes de responder, que duvida antes de afirmar, que escuta antes de prescrever. Uma psiquiatria que reconhece seus limites e, precisamente por isso, pode oferecer mais aos que a procuram.

\vspace{1cm}

\begin{flushright}
\textit{Gustavo Mendes e Silva, M.D.}
\end{flushright}
