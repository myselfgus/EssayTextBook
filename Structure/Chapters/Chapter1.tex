% !TEX root = ../../BookTemplate.tex
%%%%%%%%%%%%%%%%%%%%%%%%%%%%%%%%%%%%%%%%%%%%%%%%%%%%%%%%%%%%%%%%%%%%%%%%%%%%%%%%%%
\chapter{Para o que nao ha resposta, deve-se continuar perguntando}
\label{cap:pergunta}

\begin{center}
\textit{``A aceitacao de questoes abertas para o conhecimento e a consciencia dos proprios limites sao necessarias para estabelecer o circulo hermeneutico.''}\\[0.3cm]
--- Karl Jaspers, \textit{Psicopatologia Geral} (1913)
\end{center}

\vspace{1cm}

%%%%%%%%%%%%%%%%%%%%%%%%%%%%%%%%%%%%%%%%%%%%%%%%%%%%%%%%%%%%%%%%%%%%%%%%%%%%%%%%%%
\section{A pergunta como metodo}
\label{sec:pergunta-metodo}
\index{Epistemologia!pergunta como metodo}

Este livro comeca com uma confissao: nao tenho respostas. Ou melhor, as respostas que tenho sao provisorias, incompletas, sujeitas a revisao. E e precisamente isso que me parece o unico ponto de partida honesto para qualquer reflexao sobre psiquiatria.

O titulo deste capitulo --- e deste livro --- nao e uma evasiva. E uma \termotecnico{posicao epistemologica}. Para aquilo que resiste a resposta definitiva, a unica atitude intelectualmente integra e manter a pergunta aberta. Isso nao significa resignacao diante do desconhecido, nem paralisia frente a necessidade de agir clinicamente. Significa reconhecer que certas questoes --- o que e a mente? o que e o sofrimento? onde termina a variacao humana e comeca a patologia? --- nao admitem fechamento. Responde-las prematuramente e falsifica-las.

\begin{conceitochave}[title={Posicao Epistemologica Fundamental}]
A psiquiatria lida com questoes que resistem a respostas definitivas:
\begin{itemize}
    \item O que e a mente?
    \item O que e o sofrimento?
    \item Onde termina a variacao humana e comeca a patologia?
\end{itemize}
Responder prematuramente a essas questoes e falsifica-las.
\end{conceitochave}

A psiquiatria contemporanea, contudo, parece ter desenvolvido alergia a pergunta. Onde deveria haver investigacao, ha protocolo. Onde deveria haver escuta, ha checklist. Onde deveria haver incerteza assumida, ha certezas fabricadas. O problema nao e que a psiquiatria tenha deixado de fazer perguntas --- ela faz perguntas o tempo todo. O problema e que faz as perguntas erradas.

\begin{questao}
``Quantos criterios este paciente preenche?'' e uma pergunta. ``Qual antidepressivo prescrever primeiro?'' e uma pergunta. ``Este quadro e F32.1 ou F32.2?'' e uma pergunta. Mas nenhuma delas toca o essencial. A pergunta que importa --- ``o que significa este sofrimento para esta pessoa, neste momento de sua vida, neste contexto?'' --- essa raramente e formulada. E quando formulada, raramente e sustentada tempo suficiente para que uma resposta genuina emerja.

\textbf{Voce, na sua pratica clinica, sustenta as perguntas dificeis ou busca respostas rapidas?}
\end{questao}

%%%%%%%%%%%%%%%%%%%%%%%%%%%%%%%%%%%%%%%%%%%%%%%%%%%%%%%%%%%%%%%%%%%%%%%%%%%%%%%%%%
\section{Medicina e filosofia}
\label{sec:medicina-filosofia}
\index{Filosofia da medicina}
\index{Hipocrates}

Ha uma confusao comum que precisa ser desfeita: a ideia de que medicina e filosofia sao dominios separados, que a medicina ``nasceu da'' filosofia ou ``foi influenciada pela'' filosofia em seus primordios, mas depois se emancipou para tornar-se ciencia autonoma. Essa narrativa de separacao e historicamente imprecisa e conceitualmente empobrecedora.

\begin{observacao}
\textbf{Medicina e filosofia. Nao ``era'', nao ``foi'', nao ``deveria ser''. E.}
\end{observacao}

A tradicao hipocratica, frequentemente invocada como origem da medicina ocidental, nao se caracterizava pela aplicacao tecnica de conhecimentos fixos. O \textit{Corpus Hippocraticum} e descrito pelos historiadores como um conjunto de textos marcados por ``discussoes questionadoras, argumentativas e especulativas'' \cite{Potter1996}. A escola de Cos, de onde Hipocrates provinha, distinguia-se precisamente por sua enfase nao no diagnostico classificatorio --- essa era a especialidade da escola rival de Cnido --- mas no cuidado com o paciente, na observacao clinica prolongada, no prognostico como arte de compreender a trajetoria singular de cada adoecimento.

\begin{figure}[H]
\centering
\begin{tikzpicture}[
    node distance=2cm,
    box/.style={rectangle, draw=chaptercolor, fill=boxblue, text width=4cm, text centered, rounded corners, minimum height=1.5cm, font=\small},
    arrow/.style={-Stealth, thick, color=sectioncolor}
]
    \node[box] (phronesis) {\textbf{Phronesis}\\Sabedoria pratica};
    \node[box, below left=2cm and 0.5cm of phronesis] (compreensao) {\textbf{Compreensao}\\Insight clinico};
    \node[box, below right=2cm and 0.5cm of phronesis] (julgamento) {\textbf{Julgamento}\\Decisao pratica};
    \node[box, below=4cm of phronesis] (acao) {\textbf{Acao Clinica}\\Cuidado singular};

    \draw[arrow] (phronesis) -- (compreensao);
    \draw[arrow] (phronesis) -- (julgamento);
    \draw[arrow] (compreensao) -- (acao);
    \draw[arrow] (julgamento) -- (acao);
\end{tikzpicture}
\caption{Phronesis: a competencia medica segundo Aristoteles}
\label{fig:phronesis}
\end{figure}

O que caracterizava o bom medico hipocratico nao era o dominio de um catalogo de doencas, mas uma competencia que \autor{Aristoteles} chamaria de \termotecnico{phronesis}: sabedoria pratica, a capacidade de deliberar bem em situacoes singulares onde regras gerais sao insuficientes. E o que um estudo contemporaneo sobre epistemologia medica descreve como ``a combinacao de compreensao, insight e julgamento usada para acao pratica'' --- competencia que ``permaneceu inalterada atraves dos seculos, intocada pela evolucao do conhecimento factual, inovacao tecnologica e avancos terapeuticos'' \cite{Macnaughton2013}.

\begin{definicaobox}[title={Phronesis (Aristoteles)}]
\textbf{Phronesis} (\textit{phronesis}): Sabedoria pratica; a capacidade de deliberar bem em situacoes singulares onde regras gerais sao insuficientes. Diferente do conhecimento teorico (\textit{episteme}) ou tecnico (\textit{techne}), a phronesis envolve discernimento moral e pratico aplicado a casos particulares.
\end{definicaobox}

Medicos como Hipocrates, Maimonides, Avicena, Sydenham e Osler nao tinham nosso conhecimento biologico nem nossas ferramentas diagnosticas. Do nosso ponto de vista, eram ``ignorantes e impotentes''. Suas teorias de anatomia, fisiologia e causacao de doencas nos parecem bizarras. Seus recursos terapeuticos eram poucos e largamente ineficazes. E, no entanto, sao venerados como exemplares do que significa ser medico. Por que?

Porque sabiam algo que permanece verdadeiro: que medicina nao e aplicacao de conhecimento tecnico a casos particulares, mas \textbf{investigacao continua sobre corpo, doenca, saude, sofrimento}. E pratica interrogativa. Cada paciente e uma pergunta que exige nova formulacao. \autor{Edmund Pellegrino}, um dos fundadores da filosofia da medicina contemporanea, definiu filosofia \textit{na} medicina como ``reflexao critica, raciocinio dialetico, desvelamento de valor e proposito, ou fazer perguntas de primeira ordem'' --- e acrescentou que essa definicao descreve exatamente o que o bom clinico faz diante de cada paciente \cite{Pellegrino1998}.

%%%%%%%%%%%%%%%%%%%%%%%%%%%%%%%%%%%%%%%%%%%%%%%%%%%%%%%%%%%%%%%%%%%%%%%%%%%%%%%%%%
\section{As perguntas erradas}
\label{sec:perguntas-erradas}
\index{DSM}
\index{Diagnostico!criterios operacionais}

A psiquiatria nao parou de perguntar. Pergunta incessantemente. Mas suas perguntas refletem pressupostos que raramente sao examinados.

Quando o DSM se tornou a linguagem franca da psiquiatria, um deslocamento epistemologico ocorreu. O manual foi desenhado para resolver um problema real: a baixa confiabilidade diagnostica entre psiquiatras. Dois clinicos examinando o mesmo paciente frequentemente chegavam a diagnosticos diferentes. A solucao proposta foi operacionalizar os diagnosticos --- definir cada transtorno por criterios explicitos, observaveis, contaveis. O ganho em confiabilidade foi real. Mas o custo foi a transformacao da pergunta fundamental.

\begin{table}[H]
\centering
\caption{Transformacao das perguntas fundamentais na psiquiatria}
\label{tab:perguntas}
\begin{tabular}{p{6cm}p{6cm}}
\toprule
\textbf{Pergunta Tradicional} & \textbf{Pergunta Contemporanea} \\
\midrule
O que esta acontecendo com esta pessoa? & Quais criterios esta pessoa preenche? \\
\addlinespace
Como este sofrimento se conecta com a historia, o contexto, os significados desta vida? & Em qual categoria nosografica este conjunto de sintomas se encaixa? \\
\addlinespace
O que este sofrimento significa para este paciente? & Qual e o codigo CID/DSM apropriado? \\
\addlinespace
Como posso compreender esta experiencia? & Qual protocolo devo seguir? \\
\bottomrule
\end{tabular}
\end{table}

\autor{Karl Jaspers}, escrevendo em 1913 --- portanto muito antes do DSM-III inaugurar a era dos criterios operacionais --- ja alertava para esse risco. Jaspers criticava o que chamava de \termotecnico{Hirnmythologie} (``mitologia cerebral''): a assuncao de que todos os transtornos mentais sao causados por processos cerebrais e que a tarefa da psiquiatria e identificar esses processos. Criticava igualmente a ``psicomitologia'': a assuncao de que todos os transtornos podem ser explicados psicologicamente. Ambas as posicoes, para Jaspers, eram ``generalizacoes inadmissiveis'' que obstruiam a compreensao genuina \cite{Jaspers1913}.

\begin{figure}[H]
\centering
\begin{tikzpicture}[
    node distance=3cm,
    box/.style={rectangle, draw, text width=3.5cm, text centered, rounded corners, minimum height=2cm, font=\small},
    bluebox/.style={box, draw=chaptercolor, fill=boxblue},
    greenbox/.style={box, draw=OliveGreen, fill=boxgreen},
    redbox/.style={box, draw=BrickRed, fill=boxred}
]
    \node[redbox] (hirn) {\textbf{Hirnmythologie}\\Tudo e cerebral\\(Reducao biologica)};
    \node[greenbox, right=of hirn] (jaspers) {\textbf{Jaspers}\\Explicar + Compreender\\(Abordagem dual)};
    \node[redbox, right=of jaspers] (psico) {\textbf{Psicomitologia}\\Tudo e psicologico\\(Reducao psicologica)};

    \draw[-Stealth, thick, dashed, color=BrickRed] (hirn) -- node[above, font=\footnotesize] {Critica} (jaspers);
    \draw[-Stealth, thick, dashed, color=BrickRed] (psico) -- node[above, font=\footnotesize] {Critica} (jaspers);
\end{tikzpicture}
\caption{Jaspers e a critica aos reducionismos}
\label{fig:jaspers-critica}
\end{figure}

O que Jaspers propunha era uma psiquiatria que distinguisse entre \termotecnico{Erklaren} (\textit{explicar}) e \termotecnico{Verstehen} (\textit{compreender}). Explicar e identificar causas no sentido das ciencias naturais: conexoes causais entre eventos fisicos. Compreender e apreender conexoes de sentido entre vivencias: como um evento psiquico emerge de outro, como a raiva segue a ofensa, o ciume a traicao, o luto a perda. Os dois modos de abordagem sao legitimos, mas irredutveis um ao outro.

\begin{conceitochave}[title={Erklaren vs. Verstehen (Jaspers)}]
\begin{description}
    \item[\textit{Erklaren} (Explicar):] Identificar causas no sentido das ciencias naturais --- conexoes causais entre eventos fisicos. Dominio das ciencias naturais.

    \item[\textit{Verstehen} (Compreender):] Apreender conexoes de sentido entre vivencias --- como um evento psiquico emerge de outro. Dominio da fenomenologia.
\end{description}
\vspace{0.3cm}
\textbf{Os dois modos sao legitimos, mas irredutveis um ao outro.}
\end{conceitochave}

O metodo para compreender, segundo Jaspers, envolvia tres procedimentos: ``(1) imersao nos gestos, comportamentos, movimentos expressivos do paciente; (2) questionamento direto; (3) autodetextos escritas''. Nenhum desses procedimentos pode ser substituido por escalas padronizadas ou entrevistas estruturadas. Exigem tempo, atencao, presenca. Exigem, fundamentalmente, que o psiquiatra \textbf{converse} com o paciente.

%%%%%%%%%%%%%%%%%%%%%%%%%%%%%%%%%%%%%%%%%%%%%%%%%%%%%%%%%%%%%%%%%%%%%%%%%%%%%%%%%%
\section{O silencio psiquiatrico}
\label{sec:silencio}
\index{Comunicacao medico-paciente}
\index{Medication management}

``Meu psiquiatra nao me escuta.'' E uma queixa tao frequente que se tornou lugar-comum. Forums de pacientes transbordam de relatos semelhantes. Ha sites inteiros dedicados a orientar pacientes sobre ``o que fazer quando seu psiquiatra nao ouve voce''. A queixa e tao disseminada que arrisca parecer exagero de pacientes insatisfeitos. Nao e.

\begin{casoclinico}[title={O modelo de ``medication management''}]
Nos Estados Unidos, o modelo dominante de pratica psiquiatrica ambulatorial e o chamado \textit{medication management}:
\begin{itemize}
    \item Consultas de 15-20 minutos
    \item Foco exclusivo no manejo farmacologico
    \item Pacientes vistos a cada 1-3 meses
    \item Psicoterapia delegada a outros profissionais (\textit{split treatment})
    \item Carga media: 150-200 pacientes por psiquiatra
\end{itemize}

Nesse modelo, o psiquiatra se ocupa essencialmente de prescrever, ajustar doses e monitorar efeitos colaterais.
\end{casoclinico}

Os dados empiricos confirmam o que os pacientes relatam. Estudos de comunicacao medico-paciente usando analise de interacao verbal mostram padroes reveladores. Em consultas de manejo medicamentoso, 67\% da fala dos pacientes concentra-se em temas biomedicos --- sintomas de depressao, efeitos de medicacao --- enquanto questoes psicossociais mais amplas raramente emergem \cite{Slade2012}. Pacientes nao ``refletem um nivel similar de parceria'' na conversa, sugerindo que a estrutura da consulta nao favorece dialogo genuino. Pesquisas sobre miscomunicacao em psiquiatria identificam que pacientes frequentemente ``nao se sentem compreendidos'' e que isso se correlaciona com ma adesao ao tratamento e piores desfechos \cite{McCabe2018}.

\begin{table}[H]
\centering
\caption{Distribuicao tipica da fala em consultas de medication management}
\label{tab:fala}
\begin{tabular}{lc}
\toprule
\textbf{Tema da conversa} & \textbf{Porcentagem} \\
\midrule
Sintomas e queixas biomedicas & 45\% \\
Medicacao e efeitos colaterais & 22\% \\
Questoes administrativas & 18\% \\
Contexto psicossocial & 10\% \\
Historia de vida e significados & 5\% \\
\bottomrule
\end{tabular}
\end{table}

E tentador atribuir o problema a fatores externos: pressoes do sistema de saude, demandas de produtividade, escassez de profissionais. Esses fatores sao reais e nao devem ser minimizados. Mas ha algo mais profundo em jogo. A estrutura conceitual da psiquiatria contemporanea --- sua epistemologia implicita --- \textbf{nao requer conversa}. Se o diagnostico e uma questao de identificar criterios, e o tratamento e uma questao de selecionar o farmaco adequado ao diagnostico, entao a fala do paciente e instrumentalizada: serve para extrair informacoes que alimentam o algoritmo diagnostico-terapeutico. O paciente como sujeito, como pessoa com uma historia e um contexto, como alguem cujo sofrimento tem significado, torna-se dispensavel.

%%%%%%%%%%%%%%%%%%%%%%%%%%%%%%%%%%%%%%%%%%%%%%%%%%%%%%%%%%%%%%%%%%%%%%%%%%%%%%%%%%
\section{Do que e feita uma psiquiatria que nao conversa?}
\label{sec:psiquiatria-sem-conversa}
\index{Subjetividade}
\index{Dialogo clinico}

Cada especialidade medica tem um substrato. A cardiologia tem o coracao, a nefrologia os rins, a dermatologia a pele. Sao orgaos que podem ser visualizados, palpados, medidos, biopsiados. A psiquiatria nao tem orgao. Seu objeto nao e o cerebro --- isso seria neurologia. Seu objeto e algo que nao pode ser apontado, localizado, quantificado diretamente: a \termotecnico{subjetividade}, a experiencia, a vida mental.

\begin{figure}[H]
\centering
\begin{tikzpicture}[
    node distance=0.8cm,
    specialty/.style={rectangle, draw=sectioncolor, fill=boxblue, text width=3cm, text centered, minimum height=1cm, font=\small},
    organ/.style={rectangle, draw=Gray, fill=boxgray, text width=3cm, text centered, minimum height=1cm, font=\small},
    psych/.style={rectangle, draw=chaptercolor, fill=boxorange, text width=3cm, text centered, minimum height=1cm, font=\small\bfseries}
]
    % Column 1
    \node[specialty] (cardio) {Cardiologia};
    \node[organ, right=of cardio] (coracao) {Coracao};

    % Column 2
    \node[specialty, below=of cardio] (nefro) {Nefrologia};
    \node[organ, right=of nefro] (rins) {Rins};

    % Column 3
    \node[specialty, below=of nefro] (derma) {Dermatologia};
    \node[organ, right=of derma] (pele) {Pele};

    % Column 4
    \node[specialty, below=of derma] (neuro) {Neurologia};
    \node[organ, right=of neuro] (cerebro) {Cerebro};

    % Psychiatry
    \node[psych, below=of neuro] (psiq) {Psiquiatria};
    \node[psych, right=of psiq] (subj) {Subjetividade\\(?)};

    % Arrows
    \draw[-Stealth] (cardio) -- (coracao);
    \draw[-Stealth] (nefro) -- (rins);
    \draw[-Stealth] (derma) -- (pele);
    \draw[-Stealth] (neuro) -- (cerebro);
    \draw[-Stealth, dashed, thick] (psiq) -- (subj);
\end{tikzpicture}
\caption{O objeto singular da psiquiatria}
\label{fig:substrato}
\end{figure}

Como se acessa a subjetividade de outra pessoa? Nao ha ressonancia magnetica da angustia, nao ha exame de sangue para detectar desespero, nao ha biopsia do sentido perdido. A unica via de acesso e a \textbf{linguagem}. E a palavra. E o encontro intersubjetivo em que uma pessoa tenta comunicar a outra algo de sua experiencia, e a outra tenta receber, compreender, responder.

\begin{observacao}
O dialogo nao e um acessorio da pratica psiquiatrica, um complemento humanistico a ser adicionado quando sobra tempo. \textbf{E o substrato mesmo da especialidade.} Uma psiquiatria que nao conversa nao e uma psiquiatria empobrecida ou apressada. E uma contradicao em termos. E como falar de uma oftalmologia que nao olha, uma otorrinolaringologia que nao ouve.
\end{observacao}

A pergunta entao se impoe: se a psiquiatria contemporanea reduz progressivamente o espaco do dialogo, do que ela passa a ser feita? A resposta e desconfortavel: de pressupostos nao examinados. De categorias diagnosticas tratadas como entidades reais. De mecanismos neuroquimicos postulados mas nao demonstrados. De protocolos que prometem certeza onde so ha incerteza. De uma cientificidade performatica que mascara ignorancia fundamental.

%%%%%%%%%%%%%%%%%%%%%%%%%%%%%%%%%%%%%%%%%%%%%%%%%%%%%%%%%%%%%%%%%%%%%%%%%%%%%%%%%%
\section{O proposito deste livro}
\label{sec:proposito}

Este livro nao oferece um novo sistema diagnostico. Nao propoe uma teoria unificada da mente. Nao apresenta protocolos de tratamento baseados em evidencias de alto nivel. Se o leitor busca respostas definitivas, ficara frustrado.

O que este livro propoe e mais modesto e, talvez, mais dificil: \textbf{restaurar a pratica de perguntar}. Cada capitulo seguinte toma uma questao que a psiquiatria contemporanea tende a tratar como resolvida e a reabre:

\begin{table}[H]
\centering
\caption{Estrutura tematica do livro}
\label{tab:estrutura}
\begin{tabular}{cp{10cm}}
\toprule
\textbf{Capitulo} & \textbf{Questao Central} \\
\midrule
2 & O que e diagnostico e qual seu rigor possivel? \\
3 & O que e doenca mental e como distingui-la do sofrimento normal? \\
4 & O que causa o sofrimento psiquico? \\
5 & Onde esta a fronteira entre sanidade e loucura? \\
6 & O que acontece nos estados de ruptura cognitiva? \\
7 & Quando e como usar antipsicoticos? \\
8 & Qual o preco temporal e existencial do tratamento? \\
9 & Para quem a medicacao realmente trabalha? \\
\bottomrule
\end{tabular}
\end{table}

Essas perguntas nao tem respostas simples. Mas a recusa em enfrenta-las tem consequencias reais --- para os pacientes que recebem diagnosticos precipitados, tratamentos inadequados, ou simplesmente nao sao ouvidos.

Jaspers escreveu, ha mais de um seculo, que ``a aceitacao de questoes abertas para o conhecimento e a consciencia dos proprios limites sao necessarias para estabelecer o circulo hermeneutico''. Traduzindo: e preciso tolerar nao saber para poder, eventualmente, saber algo. E preciso sustentar a pergunta para que uma resposta genuina --- sempre parcial, sempre revisavel --- possa emergir.

%%%%%%%%%%%%%%%%%%%%%%%%%%%%%%%%%%%%%%%%%%%%%%%%%%%%%%%%%%%%%%%%%%%%%%%%%%%%%%%%%%
\begin{sintese}
\textbf{Ideias centrais deste capitulo:}
\begin{enumerate}
    \item A pergunta aberta e uma posicao epistemologica, nao uma evasiva.
    \item Medicina e filosofia --- pratica interrogativa diante de cada paciente.
    \item A psiquiatria contemporanea faz perguntas, mas frequentemente as erradas.
    \item O silencio psiquiatrico resulta de uma epistemologia que nao requer conversa.
    \item O dialogo e o substrato da psiquiatria, nao um complemento opcional.
    \item Este livro busca restaurar a pratica de perguntar.
\end{enumerate}

\vspace{0.5cm}
\textbf{Conceitos-chave:} Phronesis $\bullet$ Erklaren vs. Verstehen $\bullet$ Hirnmythologie $\bullet$ Subjetividade $\bullet$ Circulo hermeneutico
\end{sintese}

\vspace{1cm}

\begin{center}
\textit{A psiquiatria que este livro imagina nao e uma psiquiatria de certezas. E uma psiquiatria que faz perguntas dificeis e suporta a angustia de nao ter respostas prontas. E uma psiquiatria que conversa. E uma psiquiatria que, diante do que nao pode ser respondido definitivamente, continua perguntando.}
\end{center}
