% !TEX root = ../../BookTemplate.tex
%%%%%%%%%%%%%%%%%%%%%%%%%%%%%%%%%%%%%%%%%%%%%%%%%%%%%%%%%%%%%%%%%%%%%%%%%%%%%%%%%%
\chapter{Antipsicóticos na Prática Clínica — Ferramenta Terapêutica ou Armadilha Medicalizante?}
\label{cap:antipsicoticos}

\begin{center}
\textit{``Entre a conquista técnica e a medicalização indiscriminada.''}
\end{center}

\vspace{1cm}

%%%%%%%%%%%%%%%%%%%%%%%%%%%%%%%%%%%%%%%%%%%%%%%%%%%%%%%%%%%%%%%%%%%%%%%%%%%%%%%%%%
\section{A Dupla Face dos Antipsicóticos}
\label{sec:dupla-face}
\index{Antipsicóticos}
\index{Medicalização}

Os antipsicóticos ocupam posição singular no arsenal terapêutico da psiquiatria contemporânea. Como observa \autor{Rose} (2019) em sua análise do "futuro psiquiátrico", estes fármacos representam simultaneamente conquista técnica inegável e vetor potencial de medicalização indiscriminada. Desde a introdução da clorpromazina em 1952 — evento que \autor{Healy} (2002) denominou "revolução psicofarmacológica" —, a psiquiatria oscila entre a celebração de sua capacidade de aliviar sofrimento psicótico genuíno e a crítica de seu uso como instrumento de controle social e normalização comportamental.

Esta tensão reflete o que \autor{Foucault} (1961/2006) identificou como ambiguidade constitutiva da psiquiatria moderna: disciplina que, ao mesmo tempo em que promete libertação do sofrimento mental, opera como dispositivo de poder que delimita fronteiras entre o normal e o patológico. Os antipsicóticos, neste sentido, materializam farmacologicamente esta dupla função — terapêutica e normalizadora.

O campo contemporâneo enfrenta o que \autor{Hacking} (1999) denominou "nominalismo dinâmico" aplicado à psicofarmacologia: as categorias diagnósticas que justificam prescrições não apenas descrevem estados mentais preexistentes, mas ativamente moldam a experiência subjetiva que pretendem tratar. Como alertam \autor{Moncrieff e Cohen} (2006) em sua crítica ao modelo centrado na doença (\textit{disease-centered model}), a suposição de que antipsicóticos corrigem desequilíbrios neuroquímicos específicos permanece empiricamente contestável, devendo ser substituída por um modelo centrado no fármaco (\textit{drug-centered model}) que reconhece seus efeitos como estados alterados induzidos farmacologicamente.

\begin{conceitochave}[title={Modelo Centrado na Doença vs. Modelo Centrado no Fármaco}]
\autor{Moncrieff e Cohen} (2006) propõem distinguir:
\begin{description}
    \item[Modelo centrado na doença:] Antipsicóticos ``corrigem'' desequilíbrios neuroquímicos específicos.
    \item[Modelo centrado no fármaco:] Antipsicóticos produzem estados mentais alterados que, incidentalmente, atenuam manifestações psicóticas.
\end{description}
A distinção possui relevância clínica: se antipsicóticos funcionam primariamente através de efeitos sobre estados mentais gerais, torna-se imperativo avaliar se estes efeitos são desejáveis para o paciente específico, considerando impactos sobre qualidade de vida, funcionamento cognitivo e experiência subjetiva.
\end{conceitochave}

%%%%%%%%%%%%%%%%%%%%%%%%%%%%%%%%%%%%%%%%%%%%%%%%%%%%%%%%%%%%%%%%%%%%%%%%%%%%%%%%%%
\section{O Que Define Indicação Genuína de Antipsicóticos?}
\label{sec:indicacao}
\index{Psicose}
\index{Perturbação da ipseidade}

\subsection{A Fenomenologia do Sintoma Psicótico Verdadeiro}

A questão da indicação adequada de antipsicóticos remete ao problema mais fundamental da definição de psicose genuína. \autor{Jaspers} (1913/1997), em sua \textit{Psicopatologia Geral}, estabeleceu distinção seminal entre o que denominou "desenvolvimento da personalidade" — transformações compreensíveis da experiência em resposta a circunstâncias existenciais — e "processo" — alterações que introduzem descontinuidade radical na biografia do sujeito, incompreensíveis em termos de conexões de sentido.

Esta distinção fenomenológica possui implicações clínicas decisivas. Como observa \autor{Stanghellini} (2016), o sintoma psicótico genuíno caracteriza-se pela "perturbação da ipseidade" (\textit{self-disturbance}) — alteração na estrutura básica da experiência de si mesmo que precede e fundamenta manifestações sintomáticas específicas como delírios e alucinações.

Os antipsicóticos encontram sua indicação genuína precisamente nestas condições caracterizadas por:

\begin{table}[H]
\centering
\caption{Características da psicose genuína que justificam o uso de antipsicóticos}
\label{tab:psicose-genuina}
\begin{tabular}{p{4cm}p{8cm}}
\toprule
\textbf{Característica} & \textbf{Descrição} \\
\midrule
Alterações estruturais da experiência & Transformações nos "sentimentos existenciais" (\autor{Ratcliffe}, 2008) — a estrutura básica através da qual experimentamos o mundo como real, significativo e habitável. Envolve perturbações como a "hiperreflexividade" (\autor{Sass}, 1992). \\
\addlinespace
Impacto funcional pervasivo e persistente & Comprometimento da "integração narrativa" (\autor{Bolton e Hill}, 2004) — a habilidade de manter coerência interpretativa sobre a própria experiência através do tempo. Perturbação na "síntese temporal implícita da consciência" (\autor{Fuchs}, 2012). \\
\addlinespace
Base neurobiológica identificável & Alterações consistentes na conectividade frontotemporal (\autor{Meyer-Lindenberg}, 2010). Modelo de "saliência aberrante" — disfunção dopaminérgica que atribui significado anômalo a estímulos neutros (\autor{Howes e Kapur}, 2009). \\
\bottomrule
\end{tabular}
\end{table}

\subsection{A Esquizofrenia como Paradigma de Indicação}

A esquizofrenia representa o paradigma de condição para a qual antipsicóticos demonstram eficácia robusta. Meta-análises de rede como a de \autor{Leucht} et al. (2023), incluindo 45 estudos e 11.238 participantes, documentam que antipsicóticos reduzem significativamente sintomas positivos em tratamento de longo prazo. Estudos de prevenção de recaída, como a meta-análise de \autor{Leucht} et al. (2012), demonstram redução de aproximadamente 60\% no risco relativo de recaída em comparação com placebo.

\begin{observacao}
\textbf{Cautela epistemológica:} A demonstração de eficácia em ensaios clínicos randomizados não implica necessariamente que antipsicóticos "corrijam" um desequilíbrio neuroquímico subjacente. Antes, pode refletir o que \autor{Moncrieff} (2013) denomina "modelo centrado no fármaco" — os antipsicóticos produzem estados mentais alterados (sedação, embotamento afetivo, redução da saliência experiencial) que, incidentalmente, atenuam manifestações psicóticas.
\end{observacao}

\subsection{Quadros Transitórios e o Papel do Aripiprazol}

Nem todo sintoma psicótico justifica tratamento antipsicótico prolongado. Como observa \autor{van Os} (2009) em seu trabalho sobre o "continuum psicótico", experiências semelhantes a alucinações e ideação paranoide ocorrem em proporção significativa da população geral sem configurar patologia. O que \autor{Jaspers} (1913/1997) denominou "reações psicógenas" — estados psicóticos transitórios compreensíveis como respostas a circunstâncias existenciais extremas — podem remitir espontaneamente ou com suporte psicoterapêutico adequado.

Para quadros caracterizados por descontinuidades mais sutis no teste de realidade — o que poderíamos denominar "disrupturas cognitivas" com colorido psicótico —, o aripiprazol oferece perfil farmacológico diferenciado. Como agonista parcial dopaminérgico D2, demonstra efeito "estabilizador" (\autor{Stahl}, 2013) — modulando a neurotransmissão dopaminérgica sem o bloqueio massivo característico de antipsicóticos convencionais.

Clinicamente, isto se traduz em menor sedação, menor embotamento afetivo e melhor preservação do que \autor{Merleau-Ponty} (1945/2006) denominou "arco intencional" — a estrutura dinâmica que organiza nossa experiência perceptiva, cognitiva e afetiva do mundo.

%%%%%%%%%%%%%%%%%%%%%%%%%%%%%%%%%%%%%%%%%%%%%%%%%%%%%%%%%%%%%%%%%%%%%%%%%%%%%%%%%%
\section{O Problema do Uso Inadequado}
\label{sec:uso-inadequado}
\index{Uso off-label}
\index{Autismo}
\index{Deficiência intelectual}

\subsection{A Medicalização do Sofrimento Existencial}

A expansão contínua das indicações de antipsicóticos representa manifestação paradigmática do que \autor{Illich} (1976/2010) identificou como "expropriação da saúde" — processo pelo qual dimensões da experiência humana são progressivamente colonizadas pela racionalidade biomédica. \autor{Conrad} (2007) documenta como condições previamente consideradas variantes normais — timidez, agitação infantil, tristeza — foram progressivamente transformadas em categorias patológicas sujeitas à intervenção farmacológica.

No caso dos antipsicóticos, esta tendência medicalizante manifesta-se particularmente em duas populações vulneráveis: pessoas com Transtorno do Espectro Autista (TEA) e pessoas com Deficiência Intelectual (DI).

\begin{figure}[H]
\centering
\begin{tikzpicture}[
    node distance=1cm,
    stat/.style={rectangle, draw=BrickRed, fill=boxred, text width=8cm, text centered, rounded corners, minimum height=1cm, font=\small}
]
    \node[stat] (stomp) {\textbf{17,1\%} dos adultos com DI na Inglaterra recebem prescrições de antipsicóticos (NHS STOMP, 2019)};
    \node[stat, below=of stomp] (lar) {\textbf{20\%} dos residentes em lares de idosos nos EUA recebem antipsicóticos (Análise de 12.000 instituições, 2024)};
    \node[stat, below=of lar] (swiss) {\textbf{94,3\%} das prescrições de quetiapina em lares suíços são off-label (Müller et al., 2020)};
\end{tikzpicture}
\caption{Estatísticas preocupantes sobre uso de antipsicóticos em populações vulneráveis}
\label{fig:estatisticas}
\end{figure}

\subsection{Diagnóstico Errôneo de Esquizofrenia}

A atribuição precipitada de diagnóstico de esquizofrenia representa uma das formas mais consequentes de uso inadequado de antipsicóticos. Como observa \autor{Frances} (2013) em sua autocrítica como coordenador do DSM-IV, a psiquiatria contemporânea sofre de "inflação diagnóstica" — tendência a patologizar experiências que, em outras circunstâncias históricas ou culturais, seriam interpretadas como variantes da normalidade ou respostas adaptativas a contextos adversos.

\begin{table}[H]
\centering
\caption{Condições frequentemente confundidas com esquizofrenia}
\label{tab:confusao}
\begin{tabular}{p{4cm}p{8cm}}
\toprule
\textbf{Condição} & \textbf{Distinção Clínica} \\
\midrule
Estados dissociativos & "Desagregação psíquica" (\autor{Janet}, 1889/1973) pode produzir fenômenos experienciais semelhantes a sintomas psicóticos, mas com etiologia e tratamento distintos (\autor{van der Hart et al.}, 2006). \\
\addlinespace
Transtorno de personalidade borderline & Episódios micropsicóticos transitórios em contextos de estresse interpessoal intenso; tipicamente breves e responsivos a intervenções psicossociais (\autor{Zanarini et al.}, 1990). \\
\addlinespace
Intoxicação e abstinência de substâncias & Estados psicóticos induzidos por substâncias podem mimetizar esquizofrenia, mas frequentemente remitem com abstinência e suporte (\autor{Caton et al.}, 2005). \\
\addlinespace
Condições médicas gerais & Encefalopatias, endocrinopatias e epilepsias do lobo temporal podem produzir fenomenologia psicótica secundária que requer tratamento da condição de base (\autor{Freudenreich et al.}, 2009). \\
\bottomrule
\end{tabular}
\end{table}

\subsection{Autismo e a Confusão entre Neurodivergência e Psicose}

O uso de antipsicóticos em pessoas com TEA ilustra exemplarmente os riscos da medicalização inadequada. Pessoas autistas frequentemente apresentam estrutura de linguagem e comunicação que, como observa \autor{Happé e Frith} (2006) em sua teoria da "coerência central fraca", difere qualitativamente dos padrões neurotípicos. Esta diferença pode ser erroneamente interpretada como "desorganização do pensamento" ou mesmo "delírio".

\begin{questao}
A meta-análise de rede Cochrane (2024-2025) sobre antipsicóticos atípicos em TEA demonstrou eficácia limitada para "irritabilidade" a curto prazo, porém com perfil significativo de efeitos adversos — ganho de peso, sintomas extrapiramidais — e ausência de dados robustos de longo prazo. Criticamente, nenhum estudo demonstrou efeito sobre sintomas centrais do TEA, sugerindo que antipsicóticos funcionam primariamente através de sedação inespecífica (\autor{Williams et al.}, 2025).

\textbf{Pergunta:} Estamos tratando uma patologia ou medicalizando uma diferença?
\end{questao}

\subsection{Deficiência Intelectual e o Manejo Químico de "Comportamentos Desafiadores"}

O uso de antipsicóticos para "comportamentos desafiadores" em pessoas com DI representa talvez a aplicação mais eticamente problemática destas medicações. Como documenta \autor{de Kuijper et al.} (2014), médicos frequentemente resistem à descontinuação de antipsicóticos nesta população apesar de evidências limitadas de eficácia.

O ensaio clínico randomizado de \autor{Tyrer et al.} (2008), comparando risperidona, haloperidol e placebo em pacientes com DI e comportamento agressivo, não demonstrou superioridade dos antipsicóticos sobre placebo — achado que, segundo os autores, "deveria ter mudado a prática clínica de forma mais radical do que ocorreu".

Como observa \autor{Hastings} (2013) em sua análise comportamental, "comportamentos desafiadores" em pessoas com DI frequentemente representam comunicação funcional em contextos de recursos expressivos limitados. A análise funcional do comportamento oferece alternativa não-farmacológica com base evidencial mais robusta.

%%%%%%%%%%%%%%%%%%%%%%%%%%%%%%%%%%%%%%%%%%%%%%%%%%%%%%%%%%%%%%%%%%%%%%%%%%%%%%%%%%
\section{Antipsicóticos no Espectro Terapêutico}
\label{sec:espectro}
\index{Intervenções psicossociais}
\index{TCCp}
\index{Graduação interventiva}

\subsection{O Princípio da Graduação Interventiva}

A prática psiquiátrica eticamente orientada opera segundo o que podemos denominar "princípio da graduação interventiva" — a preferência por intervenções menos invasivas quando igualmente eficazes. Como observa \autor{Beauchamp e Childress} (2019) em sua formulação dos princípios bioéticos, o imperativo de não-maleficência demanda que intervenções com potencial iatrogênico significativo sejam reservadas para situações em que alternativas menos arriscadas demonstraram-se insuficientes.

\begin{definicaobox}[title={Princípio da Graduação Interventiva}]
Preferência por intervenções menos invasivas quando igualmente eficazes.

Fundamentação epistemológica no modelo biopsicossocial (\autor{Engel}, 1977) — a compreensão de que fenômenos mentais emergem da interação de fatores biológicos, psicológicos e sociais. Intervenções exclusivamente farmacológicas podem ser insuficientes ou contraproducentes se negligenciarem dimensões psicológicas e sociais do sofrimento.
\end{definicaobox}

\subsection{Intervenções Psicossociais: Evidências Contemporâneas}

A meta-análise de rede de \autor{Salahuddin et al.} (2024), publicada em \textit{Lancet Psychiatry}, avaliou intervenções psicológicas e psicossociais para esquizofrenia resistente ao tratamento. Os resultados demonstraram que Terapia Cognitivo-Comportamental para psicose (TCCp) foi significativamente superior ao tratamento padrão na redução de sintomas globais (SMD -0,22; IC95\% -0,35 a -0,09), com tamanho de efeito comparável ao de antipsicóticos em várias análises.

\begin{figure}[H]
\centering
\begin{tikzpicture}[
    node distance=1.5cm,
    intervention/.style={rectangle, draw=sectioncolor, fill=boxblue, text width=4cm, text centered, rounded corners, minimum height=1.2cm, font=\small},
    arrow/.style={-Stealth, thick}
]
    \node[intervention] (tccp) {\textbf{TCCp}\\Normalização, redução de ruminação, explicações alternativas};
    \node[intervention, right=of tccp] (fam) {\textbf{Intervenções Familiares}\\Redução de EE, psicoeducação};
    \node[intervention, below=1.5cm of $(tccp)!0.5!(fam)$] (int) {\textbf{Integração com Farmacoterapia}\\Efeito sinérgico na redução de recaídas};

    \draw[arrow] (tccp) -- (int);
    \draw[arrow] (fam) -- (int);
\end{tikzpicture}
\caption{Integração de modalidades terapêuticas no tratamento da esquizofrenia}
\label{fig:integracao}
\end{figure}

\subsection{Integração de Modalidades Terapêuticas}

A prática clínica baseada em evidências não opõe intervenções farmacológicas a psicossociais, mas busca sua integração ótima. Como demonstra a meta-análise de \autor{Bighelli et al.} (2021), intervenções familiares, psicoeducação e TCCp reduzem significativamente taxas de recaída em esquizofrenia quando combinadas a farmacoterapia — efeito sinérgico que nenhuma modalidade isolada alcança.

Esta integração requer o que \autor{Priebe et al.} (2013) denominam "psiquiatria social" — reorientação da disciplina para além do foco exclusivo em intervenções técnicas individuais, reconhecendo o papel de fatores relacionais, comunitários e estruturais na determinação de desfechos em saúde mental.

%%%%%%%%%%%%%%%%%%%%%%%%%%%%%%%%%%%%%%%%%%%%%%%%%%%%%%%%%%%%%%%%%%%%%%%%%%%%%%%%%%
\section{Reflexões Éticas e Práticas}
\label{sec:etica-antipsicoticos}
\index{Decisão compartilhada}
\index{Consentimento informado}
\index{Autonomia}

\subsection{O Imperativo da Decisão Compartilhada}

A prescrição de antipsicóticos envolve decisões que afetam profundamente a experiência subjetiva, o funcionamento cognitivo e a qualidade de vida do paciente. Como argumenta \autor{Deegan} (2007), pioneira do movimento de \textit{recovery} em saúde mental, pacientes não são recipientes passivos de intervenções técnicas, mas agentes ativos na construção de suas trajetórias de recuperação.

\begin{conceitochave}[title={Decisão Compartilhada}]
\autor{Charles et al.} (1997) propõem que médico e paciente sejam reconhecidos como experts em domínios complementares:
\begin{itemize}
    \item \textbf{Clínico:} Evidências científicas e opções terapêuticas
    \item \textbf{Paciente:} Valores, preferências e experiência vivida
\end{itemize}
Esta perspectiva fundamenta-se no princípio de autonomia (\autor{Beauchamp e Childress}, 2019). Estudos como o de \autor{Hamann et al.} (2005) demonstram que pacientes com esquizofrenia desejam e são capazes de participar de decisões sobre seu tratamento.
\end{conceitochave}

\subsection{Preservação da Subjetividade}

Como observa \autor{Moncrieff} (2013), antipsicóticos produzem estados mentais alterados que, embora possam atenuar sintomas psicóticos, também afetam dimensões da experiência valorizadas pelo paciente. Estudos qualitativos documentam consistentemente que efeitos adversos subjetivos — sedação, embotamento afetivo, anedonia — constituem os principais determinantes de não-adesão ao tratamento (\autor{Harris et al.}, 2017).

O objetivo do tratamento não deve ser a "normalização" do paciente segundo padrões externamente definidos (\autor{Szasz}, 1961/1974), mas a "restauração da capacidade de ação intencional" (\autor{Fulford}, 1989) — a possibilidade de o sujeito realizar seus projetos existenciais de acordo com seus próprios valores.

\subsection{Educação e Consentimento Informado}

O consentimento verdadeiramente informado para tratamento antipsicótico requer que o paciente compreenda não apenas benefícios potenciais, mas também:

\begin{itemize}
    \item \textbf{Efeitos adversos:} Riscos metabólicos (ganho de peso, diabetes, dislipidemia), neurológicos (sintomas extrapiramidais, discinesia tardia) e subjetivos (sedação, embotamento afetivo).
    \item \textbf{Alternativas:} Incluindo intervenções psicossociais com base evidencial comparável para determinados desfechos.
    \item \textbf{Incertezas:} Mecanismos de ação incompletamente compreendidos; benefícios de tratamento de longo prazo menos robustamente documentados que benefícios de curto prazo (\autor{Moncrieff}, 2020).
\end{itemize}

%%%%%%%%%%%%%%%%%%%%%%%%%%%%%%%%%%%%%%%%%%%%%%%%%%%%%%%%%%%%%%%%%%%%%%%%%%%%%%%%%%
\section{Propostas para Uso Racional}
\label{sec:uso-racional}
\index{Avaliação fenomenológica}
\index{Monitoramento}
\index{Descontinuação}

\begin{table}[H]
\centering
\caption{Propostas para uso racional de antipsicóticos}
\label{tab:propostas}
\begin{tabular}{p{4cm}p{8cm}}
\toprule
\textbf{Proposta} & \textbf{Implementação} \\
\midrule
Avaliação fenomenológica criteriosa & Realizar "diagnóstico por penetração" (\autor{Minkowski}, 1927/1970) — compreensão empática da estrutura alterada da experiência. Diferenciar sintomas psicóticos genuínos, disrupturas cognitivas transitórias, variantes neurodivergentes e respostas adaptativas. \\
\addlinespace
Monitoramento sistemático & Incluir parâmetros metabólicos (peso, glicemia, lipídicos), função cognitiva (avaliação neuropsicológica) e experiência subjetiva (escalas de qualidade de vida). Seguir protocolos como o consenso ADA/APA (2004). \\
\addlinespace
Revisão periódica e possibilidade de descontinuação & Reavaliar necessidade de continuação. Considerar descontinuação gradual com "redução hiperbólica" (\autor{Horowitz e Taylor}, 2024) para minimizar risco de abstinência. Baseado em evidências como o ensaio RADAR (\autor{Moncrieff et al.}, 2023). \\
\addlinespace
Formação clínica ampliada & Desenvolver competências em semiologia fenomenológica, hermenêutica clínica, pensamento crítico e comunicação para decisão compartilhada. Adotar "pluralismo metodológico" (\autor{Ghaemi}, 2010). \\
\bottomrule
\end{tabular}
\end{table}

%%%%%%%%%%%%%%%%%%%%%%%%%%%%%%%%%%%%%%%%%%%%%%%%%%%%%%%%%%%%%%%%%%%%%%%%%%%%%%%%%%
\section{Conclusão: Entre a Ferramenta e a Armadilha}
\label{sec:conclusao}

Os antipsicóticos representam, como observamos, ferramenta terapêutica indispensável para condições psicóticas genuínas. A esquizofrenia, transtornos delirantes e psicoses afetivas frequentemente requerem intervenção farmacológica para restabelecer o que \autor{Jaspers} (1913/1997) denominou "contato com a realidade" — a capacidade de habitar um mundo compartilhado de significados intersubjetivos.

Entretanto, o uso criterioso destas medicações requer diferenciação rigorosa entre indicações genuínas e aplicações que representam medicalização inadequada de sofrimentos existenciais, variantes neurodivergentes ou respostas adaptativas a circunstâncias adversas. A expansão indiscriminada de prescrições — particularmente para populações vulneráveis como pessoas com TEA e DI — constitui o que \autor{Rose} (2007) caracteriza como "etopolítica" farmacológica: a normalização técnica de estados subjetivos que, embora indesejáveis do ponto de vista social, não necessariamente constituem patologia médica.

A prática clínica eticamente orientada demanda: avaliação fenomenológica profunda antes de qualquer prescrição; integração de intervenções farmacológicas com abordagens psicossociais baseadas em evidências; decisão compartilhada que respeite autonomia e preferências do paciente; monitoramento sistemático de efeitos adversos metabólicos, cognitivos e subjetivos; e revisão periódica da necessidade de continuação do tratamento, incluindo possibilidade de descontinuação gradual quando indicada.

Como observa \autor{Kleinman} (1988) em sua análise sobre "narrativas de doença" (\textit{illness narratives}), o sofrimento mental não se reduz a disfunção neuroquímica corrigível farmacologicamente, mas envolve dimensões experienciais, relacionais e significativas que requerem abordagem compreensiva. Os antipsicóticos, quando prescritos criteriosamente, podem facilitar esta abordagem; quando prescritos indiscriminadamente, podem obstaculizá-la, substituindo compreensão por supressão e transformação por normalização.

A responsabilidade do psiquiatra contemporâneo inclui, portanto, não apenas competência técnica para prescrever quando indicado, mas sabedoria clínica para abster-se quando não indicado — reconhecendo, com \autor{Illich} (1976/2010), que "uma sociedade que permite transformar toda dificuldade humana em problema médico está condenada a uma dependência progressiva da medicina, que será sempre incapaz de responder às suas expectativas".

%%%%%%%%%%%%%%%%%%%%%%%%%%%%%%%%%%%%%%%%%%%%%%%%%%%%%%%%%%%%%%%%%%%%%%%%%%%%%%%%%%
\begin{sintese}
\textbf{Ideias centrais deste capítulo:}
\begin{enumerate}
    \item Antipsicóticos são ferramenta indispensável para psicoses genuínas, mas seu uso requer diferenciação rigorosa entre indicações genuínas e medicalização inadequada.
    \item O modelo centrado no fármaco (vs. centrado na doença) enfatiza que os efeitos terapêuticos derivam de estados mentais alterados, necessitando avaliação cuidadosa de riscos e benefícios subjetivos.
    \item Populações vulneráveis (TEA, DI, idosos) estão em risco especial de uso inapropriado para "comportamentos desafiadores" ou confusão entre neurodivergência e psicose.
    \item Intervenções psicossociais, particularmente TCCp, têm base evidencial robusta e devem ser integradas ao tratamento farmacológico.
    \item A decisão compartilhada e o consentimento informado real são imperativos éticos, exigindo transparência sobre efeitos adversos, alternativas e incertezas.
    \item A prescrição racional exige avaliação fenomenológica criteriosa, monitoramento sistemático e revisão periódica com possibilidade de descontinuação gradual.
\end{enumerate}

\vspace{0.5cm}
\textbf{Conceitos-chave:} Modelo centrado no fármaco $\bullet$ Perturbação da ipseidade $\bullet$ Graduação interventiva $\bullet$ TCCp $\bullet$ Decisão compartilhada $\bullet$ Descontinuação $\bullet$ Medicalização
\end{sintese}
