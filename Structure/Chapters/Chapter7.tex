% !TEX root = ../../BookTemplate.tex
%%%%%%%%%%%%%%%%%%%%%%%%%%%%%%%%%%%%%%%%%%%%%%%%%%%%%%%%%%%%%%%%%%%%%%%%%%%%%%%%%%
\chapter{Antipsicoticos na Pratica Clinica}
\label{cap:antipsicoticos}

\begin{center}
\textit{``Ferramenta Terapeutica ou Armadilha Medicalizante?''}
\end{center}

\vspace{1cm}

%%%%%%%%%%%%%%%%%%%%%%%%%%%%%%%%%%%%%%%%%%%%%%%%%%%%%%%%%%%%%%%%%%%%%%%%%%%%%%%%%%
\section{A Dupla Face dos Antipsicoticos}
\label{sec:dupla-face}
\index{Antipsicoticos}
\index{Medicalizacao}

Os antipsicoticos ocupam posicao singular no arsenal terapeutico da psiquiatria. Como observa \autor{Rose} (2019), estes farmacos representam simultaneamente conquista tecnica inegavel e vetor potencial de medicalizacao indiscriminada.

\begin{conceitochave}[title={Modelo Centrado na Doenca vs. Modelo Centrado no Farmaco}]
\autor{Moncrieff e Cohen} (2006) propoem distinguir:
\begin{description}
    \item[Modelo centrado na doenca:] Antipsicoticos ``corrigem'' desequilibrios neuroquimicos especificos.
    \item[Modelo centrado no farmaco:] Antipsicoticos produzem estados mentais alterados que, incidentalmente, atenuam manifestacoes psicoticas.
\end{description}
A distincao tem implicacoes clinicas: se funcionam atraves de efeitos sobre estados mentais gerais, torna-se imperativo avaliar se estes efeitos sao desejaveis para o paciente especifico.
\end{conceitochave}

%%%%%%%%%%%%%%%%%%%%%%%%%%%%%%%%%%%%%%%%%%%%%%%%%%%%%%%%%%%%%%%%%%%%%%%%%%%%%%%%%%
\section{O Que Define Indicacao Genuina?}
\label{sec:indicacao}
\index{Psicose}
\index{Perturbacao da ipseidade}

\subsection{A Fenomenologia do Sintoma Psicotico Verdadeiro}

Antipsicoticos encontram sua indicacao genuina em condicoes caracterizadas por:

\begin{table}[H]
\centering
\caption{Caracteristicas da psicose genuina}
\label{tab:psicose-genuina}
\begin{tabular}{p{4cm}p{7.5cm}}
\toprule
\textbf{Caracteristica} & \textbf{Descricao} \\
\midrule
Alteracoes estruturais & Transformacoes nos ``sentimentos existenciais'' (Ratcliffe) --- estrutura basica da experiencia do mundo \\
\addlinespace
Impacto funcional pervasivo & Comprometimento da ``integracao narrativa'' (Bolton \& Hill) \\
\addlinespace
Base neurobiologica & Alteracoes em conectividade frontotemporal; ``saliencia aberrante'' (Kapur) \\
\bottomrule
\end{tabular}
\end{table}

\subsection{A Esquizofrenia como Paradigma}

A meta-analise de \autor{Leucht} et al. (2023), incluindo 45 estudos, documenta que antipsicoticos reduzem significativamente sintomas positivos. A meta-analise de \autor{Leucht} et al. (2012) demonstra reducao de aproximadamente 60\% no risco relativo de recaida.

\begin{observacao}
\textbf{Cautela epistemologica:} Demonstracao de eficacia em ensaios clinicos nao implica que antipsicoticos ``corrijam'' desequilibrio neuroquimico. Pode refletir que produzem estados mentais alterados (sedacao, embotamento) que atenuam manifestacoes psicoticas.
\end{observacao}

%%%%%%%%%%%%%%%%%%%%%%%%%%%%%%%%%%%%%%%%%%%%%%%%%%%%%%%%%%%%%%%%%%%%%%%%%%%%%%%%%%
\section{O Problema do Uso Inadequado}
\label{sec:uso-inadequado}
\index{Uso off-label}
\index{Autismo}
\index{Deficiencia intelectual}

\subsection{A Medicalizacao do Sofrimento Existencial}

A expansao continua das indicacoes representa o que \autor{Illich} (1976) identificou como ``expropriacao da saude''.

\begin{figure}[H]
\centering
\begin{tikzpicture}[
    node distance=1cm,
    stat/.style={rectangle, draw=BrickRed, fill=boxred, text width=8cm, text centered, rounded corners, minimum height=1cm, font=\small}
]
    \node[stat] (stomp) {\textbf{17,1\%} dos adultos com DI na Inglaterra recebem antipsicoticos\\(NHS STOMP Program, 2019)};
    \node[stat, below=of stomp] (lar) {\textbf{20\%} dos residentes em lares de idosos nos EUA\\(Analise de 12.000 instituicoes, 2024)};
    \node[stat, below=of lar] (swiss) {\textbf{94,3\%} das prescricoes de quetiapina em lares suicos sao off-label\\(Muller et al., 2020)};
\end{tikzpicture}
\caption{Estatisticas preocupantes sobre uso de antipsicoticos}
\label{fig:estatisticas}
\end{figure}

\subsection{Diagnostico Erroneo de Esquizofrenia}

\begin{table}[H]
\centering
\caption{Condicoes frequentemente confundidas com esquizofrenia}
\label{tab:confusao}
\begin{tabular}{p{4cm}p{7.5cm}}
\toprule
\textbf{Condicao} & \textbf{Distincao Clinica} \\
\midrule
Estados dissociativos & ``Desagregacao psiquica'' (Janet) com etiologia e tratamento distintos \\
Transtorno borderline & Episodios micropsicoticos transitórios, relacionados a estresse interpessoal \\
Intoxicacao/abstinencia & Remitem com abstinencia e suporte \\
Condicoes medicas & Encefalopatias, endocrinopatias, epilepsias do lobo temporal \\
\bottomrule
\end{tabular}
\end{table}

\subsection{Autismo e a Confusao com Psicose}

\begin{questao}
Pessoas autistas frequentemente apresentam estrutura de linguagem e comunicacao que difere qualitativamente dos padroes neurotipicos (\autor{Happe e Frith}, 2006). Esta diferenca pode ser erroneamente interpretada como ``desorganizacao do pensamento''.

A meta-analise Cochrane (2024-2025) demonstrou: eficacia limitada para ``irritabilidade'', perfil significativo de efeitos adversos, ausencia de efeito sobre sintomas centrais do TEA.

\textbf{Pergunta:} Estamos tratando uma patologia ou medicalizando uma diferenca?
\end{questao}

%%%%%%%%%%%%%%%%%%%%%%%%%%%%%%%%%%%%%%%%%%%%%%%%%%%%%%%%%%%%%%%%%%%%%%%%%%%%%%%%%%
\section{Principio da Graduacao Interventiva}
\label{sec:graduacao}
\index{Intervencoes psicossociais}
\index{TCCp}

\begin{definicaobox}[title={Principio da Graduacao Interventiva}]
Preferencia por intervencoes menos invasivas quando igualmente eficazes.

O imperativo de nao-maleficencia (\autor{Beauchamp \& Childress}, 2019) demanda que intervencoes com potencial iatrogenico significativo sejam reservadas para situacoes em que alternativas menos arriscadas demonstraram-se insuficientes.
\end{definicaobox}

\subsection{Evidencias para Intervencoes Psicossociais}

A meta-analise de \autor{Salahuddin} et al. (2024) em \textit{Lancet Psychiatry} demonstrou que Terapia Cognitivo-Comportamental para psicose (TCCp) foi significativamente superior ao tratamento padrao (SMD -0,22; IC95\% -0,35 a -0,09).

\begin{figure}[H]
\centering
\begin{tikzpicture}[
    node distance=1.5cm,
    intervention/.style={rectangle, draw=sectioncolor, fill=boxblue, text width=4cm, text centered, rounded corners, minimum height=1.2cm, font=\small},
    arrow/.style={-Stealth, thick}
]
    \node[intervention] (tccp) {\textbf{TCCp}\\Normalizacao, reducao de ruminacao};
    \node[intervention, right=of tccp] (fam) {\textbf{Intervencoes Familiares}\\Reducao de EE};
    \node[intervention, below=1.5cm of $(tccp)!0.5!(fam)$] (int) {\textbf{Integracao}\\+ Farmacoterapia};

    \draw[arrow] (tccp) -- (int);
    \draw[arrow] (fam) -- (int);
\end{tikzpicture}
\caption{Integracao de modalidades terapeuticas}
\label{fig:integracao}
\end{figure}

%%%%%%%%%%%%%%%%%%%%%%%%%%%%%%%%%%%%%%%%%%%%%%%%%%%%%%%%%%%%%%%%%%%%%%%%%%%%%%%%%%
\section{Reflexoes Eticas e Praticas}
\label{sec:etica-antipsicoticos}
\index{Decisao compartilhada}
\index{Consentimento informado}

\subsection{O Imperativo da Decisao Compartilhada}

\begin{conceitochave}[title={Decisao Compartilhada}]
\autor{Charles} et al. (1997) propoem que medico e paciente sejam reconhecidos como experts em dominios complementares:
\begin{itemize}
    \item \textbf{Clinico:} Evidencias cientificas e opcoes terapeuticas
    \item \textbf{Paciente:} Valores, preferencias e experiencia vivida
\end{itemize}

Estudos (\autor{Hamann} et al., 2005) demonstram que pacientes com esquizofrenia desejam e sao capazes de participar de decisoes sobre seu tratamento.
\end{conceitochave}

\subsection{Consentimento Informado Real}

O consentimento deve incluir:
\begin{itemize}
    \item \textbf{Efeitos adversos:} Riscos metabolicos, neurologicos, subjetivos
    \item \textbf{Alternativas:} Intervencoes psicossociais com base evidencial
    \item \textbf{Incertezas:} Mecanismos incompletamente compreendidos; beneficios de longo prazo menos robustos
\end{itemize}

%%%%%%%%%%%%%%%%%%%%%%%%%%%%%%%%%%%%%%%%%%%%%%%%%%%%%%%%%%%%%%%%%%%%%%%%%%%%%%%%%%
\section{Propostas para Uso Racional}
\label{sec:uso-racional}

\begin{table}[H]
\centering
\caption{Propostas para uso racional de antipsicoticos}
\label{tab:propostas}
\begin{tabular}{p{4cm}p{7.5cm}}
\toprule
\textbf{Proposta} & \textbf{Implementacao} \\
\midrule
Avaliacao fenomenologica & Diferenciar sintomas psicoticos genuinos de disrupturas transitorias, variantes neurodivergentes, respostas adaptativas \\
\addlinespace
Monitoramento sistematico & Parametros metabolicos, funcao cognitiva, experiencia subjetiva \\
\addlinespace
Revisao periodica & Possibilidade de descontinuacao; reducao hiperbolica (\autor{Horowitz \& Taylor}, 2024) \\
\addlinespace
Formacao ampliada & Semiologia fenomenologica, hermeneutica clinica, pensamento critico \\
\bottomrule
\end{tabular}
\end{table}

%%%%%%%%%%%%%%%%%%%%%%%%%%%%%%%%%%%%%%%%%%%%%%%%%%%%%%%%%%%%%%%%%%%%%%%%%%%%%%%%%%
\begin{sintese}
\textbf{Ideias centrais deste capitulo:}
\begin{enumerate}
    \item Antipsicoticos sao ferramenta indispensavel para psicoses genuinas.
    \item Uso criterioso requer diferenciacao rigorosa entre indicacoes genuinas e medicalizacao inadequada.
    \item Populacoes vulneraveis (TEA, DI, idosos) estao em risco de uso inapropriado.
    \item Intervencoes psicossociais tem base evidencial comparavel para varios desfechos.
    \item Decisao compartilhada e consentimento informado real sao imperativos eticos.
    \item Prescricao racional exige avaliacao fenomenologica, monitoramento e revisao periodica.
\end{enumerate}

\vspace{0.5cm}
\textbf{Conceitos-chave:} Modelo centrado no farmaco $\bullet$ Perturbacao da ipseidade $\bullet$ Graduacao interventiva $\bullet$ TCCp $\bullet$ Decisao compartilhada $\bullet$ Descontinuacao
\end{sintese}
