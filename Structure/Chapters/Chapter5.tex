% !TEX root = ../../BookTemplate.tex
%%%%%%%%%%%%%%%%%%%%%%%%%%%%%%%%%%%%%%%%%%%%%%%%%%%%%%%%%%%%%%%%%%%%%%%%%%%%%%%%%%
\chapter{O Inferno Sao os Outros}
\label{cap:inferno}

\begin{center}
\textit{``L'enfer, c'est les autres'' --- O inferno sao os outros.}\\[0.3cm]
--- Jean-Paul Sartre, \textit{Huis Clos} (1944)
\end{center}

\vspace{1cm}

%%%%%%%%%%%%%%%%%%%%%%%%%%%%%%%%%%%%%%%%%%%%%%%%%%%%%%%%%%%%%%%%%%%%%%%%%%%%%%%%%%
\section{O Olhar que Constitui}
\label{sec:olhar}
\index{Sartre}
\index{Intersubjetividade}

A celebre frase de \autor{Sartre} e frequentemente mal compreendida como misantropia. Sartre esclareceu seu sentido: nao se trata de afirmar que as relacoes humanas sao infernais por natureza, mas de reconhecer que \textbf{dependemos constitutivamente do olhar do outro} para saber quem somos. O inferno nao e o outro em si; e a impossibilidade de escapar ao julgamento alheio.

\begin{conceitochave}[title={O Louco como Designacao Social}]
Se dependemos do olhar do outro para constituir nossa identidade, entao o \textit{louco} --- aquele designado como tal --- nao e louco \textit{em si mesmo}, como substancia, mas louco \textit{para os outros}, como posicao em campo relacional.

A loucura nao e apenas estado mental interno; e tambem, e talvez primariamente, \textbf{designacao social}.
\end{conceitochave}

\autor{Foucault} (1961/2006) demonstrou que esta separacao nao e gesto natural, mas construcao historica contingente. A Grande Internacao do seculo XVII nao ``descobriu'' loucos preexistentes; \textit{produziu} a loucura como categoria ao criar instituicoes especificas para segregar aqueles que perturbavam a ordem social emergente.

%%%%%%%%%%%%%%%%%%%%%%%%%%%%%%%%%%%%%%%%%%%%%%%%%%%%%%%%%%%%%%%%%%%%%%%%%%%%%%%%%%
\section{O Continuum da Experiencia}
\label{sec:continuum}
\index{Continuum psicotico}
\index{Modelo dimensional}

\subsection{Do Modelo Categorial ao Dimensional}

A psiquiatria classica opera com modelo \termotecnico{categorial}: voce \textit{tem} ou \textit{nao tem} esquizofrenia, como \textit{tem} ou \textit{nao tem} tuberculose. Entretanto, como observa \autor{Kendell} (1975), os transtornos mentais resistem a esta binariedade.

\begin{figure}[H]
\centering
\begin{tikzpicture}[scale=0.9]
    % Continuum bar
    \draw[thick, fill=boxgray] (0,0) rectangle (12,1);
    \shade[left color=boxgreen, right color=boxred] (0,0) rectangle (12,1);

    % Labels
    \node at (0,1.5) [anchor=west, font=\small] {Experiencias};
    \node at (0,1.2) [anchor=west, font=\small] {Normais};
    \node at (12,1.5) [anchor=east, font=\small] {Psicose};
    \node at (12,1.2) [anchor=east, font=\small] {Franca};

    % Percentage markers
    \draw[thick] (0.8,0) -- (0.8,1.3);
    \node at (0.8,-0.3) [font=\footnotesize] {7\% PLEs};

    % Arrow
    \draw[-Stealth, thick] (0,-0.8) -- (12,-0.8);
    \node at (6,-1.2) [font=\small\itshape] {Intensidade/Persistencia};
\end{tikzpicture}
\caption{O continuum psicotico na populacao geral}
\label{fig:continuum}
\end{figure}

\autor{Van Os e Reininghaus} (2016) documentaram que aproximadamente 7\% da populacao geral relata \termotecnico{experiencias semelhantes a psicose} (PLEs) --- alucinacoes, ideacao paranoide --- sem preencher criterios para transtorno psicotico.

\subsection{A Fenomenologia do Espectro}
\index{Perturbacao da ipseidade}

\autor{Sass e Parnas} (2003) propoem que a especificidade da experiencia esquizofrenica reside na \termotecnico{perturbacao da ipseidade} --- alteracao na estrutura basica da experiencia de si mesmo:

\begin{itemize}
    \item \textbf{Hiperreflexividade:} Tornar-se consciente de aspectos normalmente tacitos da experiencia
    \item \textbf{Diminuicao da autoafeccao:} Enfraquecimento do senso basico de ser sujeito da propria experiencia
\end{itemize}

\begin{observacao}
A pessoa que ouve vozes apos perda traumatica e a pessoa com esquizofrenia que ouve vozes podem ambas relatar ``alucinacoes auditivas'' --- mas a estrutura experiencial subjacente e radicalmente distinta.
\end{observacao}

%%%%%%%%%%%%%%%%%%%%%%%%%%%%%%%%%%%%%%%%%%%%%%%%%%%%%%%%%%%%%%%%%%%%%%%%%%%%%%%%%%
\section{A Historicidade da Loucura}
\label{sec:historicidade}
\index{Historia da psiquiatria}

\subsection{Delirios de Ontem, Verdades de Hoje}

Em 1890, um homem afirmava que seria possivel construir maquinas mais pesadas que o ar capazes de voar. Treze anos depois, os irmaos Wright realizaram o primeiro voo motorizado.

\begin{questao}
\textbf{A pergunta incomoda:}
\begin{itemize}
    \item Quantos ``delirantes'' de hoje serao reconhecidos como visionarios amanha?
    \item Quantos ``delirantes'' do passado foram simplesmente pessoas cujas ideias nao encontraram ressonancia em seu tempo?
\end{itemize}
\end{questao}

\subsection{A Relatividade Historica do Patologico}

\begin{table}[H]
\centering
\caption{Exemplos historicos de patologizacao}
\label{tab:historico}
\begin{tabular}{p{3cm}p{4cm}p{4.5cm}}
\toprule
\textbf{Condicao} & \textbf{Contexto} & \textbf{Compreensao Atual} \\
\midrule
Drapetomania (1851) & Escravos que desejavam fugir & O patologico era a escravidao \\
\addlinespace
Homossexualidade & DSM ate 1973 & Variacao normal da sexualidade \\
\addlinespace
Histeria feminina & Seculo XIX & Expressao de sofrimento em contexto patriarcal \\
\bottomrule
\end{tabular}
\end{table}

%%%%%%%%%%%%%%%%%%%%%%%%%%%%%%%%%%%%%%%%%%%%%%%%%%%%%%%%%%%%%%%%%%%%%%%%%%%%%%%%%%
\section{Semiologia da Narrativa}
\label{sec:semiologia-narrativa}
\index{Semiologia fenomenologica}

\subsection{Criterios Estruturais de Diferenciacao}

\begin{figure}[H]
\centering
\begin{tikzpicture}[
    node distance=1cm,
    criterion/.style={rectangle, draw=sectioncolor, fill=boxblue, text width=5cm, text centered, rounded corners, minimum height=1.2cm, font=\small}
]
    \node[criterion] (vivido) {\textbf{Relacao com o Vivido}\\Conexoes de sentido compreensiveis?};
    \node[criterion, below=of vivido] (tempo) {\textbf{Estrutura Temporal}\\Continuidade biografica preservada?};
    \node[criterion, below=of tempo] (inter) {\textbf{Relacao com Interlocutor}\\Capacidade dialogica mantida?};
    \node[criterion, below=of inter] (self) {\textbf{Relacao com o Self}\\Ipseidade preservada?};
\end{tikzpicture}
\caption{Criterios estruturais para diferenciacao semiologica}
\label{fig:criterios-semiologicos}
\end{figure}

\begin{definicaobox}[title={Narrativa Psicotica vs. Nao-Psicotica}]
\textbf{Narrativa Nao-Psicotica:}
\begin{itemize}
    \item Conexoes de sentido compreensiveis com a biografia
    \item Estrutura temporal preservada
    \item Capacidade dialogica mantida
    \item Pode considerar perspectivas alternativas
\end{itemize}

\textbf{Narrativa Psicotica:}
\begin{itemize}
    \item ``Incompreensibilidade'' jasperiana
    \item Fragmentacao temporal
    \item ``Solipsismo'' --- fechamento em mundo proprio
    \item Perda de referencia ao campo intersubjetivo
\end{itemize}
\end{definicaobox}

\subsection{A Propedeutica do Encontro Clinico}

A diferenciacao semiologica requer:

\begin{enumerate}
    \item \textbf{Escuta demorada:} Permitir que o fenomeno se mostre em seus proprios termos
    \item \textbf{Atencao ao como, nao apenas ao que:} Estrutura formal mais informativa que conteudo
    \item \textbf{Exploracao da historia da narrativa:} Quando surgiu? Como evoluiu?
    \item \textbf{Avaliacao da funcionalidade:} A narrativa permite funcionar no mundo?
    \item \textbf{Verificacao de alteracoes experienciais basicas:} Despersonalizacao, desrealizacao, estranheza
\end{enumerate}

%%%%%%%%%%%%%%%%%%%%%%%%%%%%%%%%%%%%%%%%%%%%%%%%%%%%%%%%%%%%%%%%%%%%%%%%%%%%%%%%%%
\section{A Psicotizacao Contextual}
\label{sec:psicotizacao}
\index{Estruturas liminares}
\index{Vulnerabilidade}

\subsection{O Fenomeno da Liminaridade}

Entre a narrativa claramente nao-psicotica e a inequivocamente psicotica existe vasto territorio intermediario --- narrativas que \textit{parecem} nao-psicoticas em certos contextos, mas que, sob determinadas condicoes, revelam caracteristicas estruturalmente psicoticas.

\subsection{Gatilhos Descompensadores}

\begin{table}[H]
\centering
\caption{Gatilhos de descompensacao em estruturas vulneraveis}
\label{tab:gatilhos}
\begin{tabular}{p{4cm}p{7.5cm}}
\toprule
\textbf{Gatilho} & \textbf{Mecanismo} \\
\midrule
Ambiguidade interpessoal & Incerteza relacional precipita espiral interpretativa \\
Sobrecarga de significado & Experiencias intensas excedem capacidade de processamento \\
Isolamento prolongado & Ausencia de correcao social permite desenvolvimento de ideias idiossincraticas \\
Substancias psicoativas & Cannabis, psicodelicos podem precipitar estados psicoticos \\
\bottomrule
\end{tabular}
\end{table}

\subsection{A Clinica da Liminaridade}

\begin{observacao}
Para o paciente liminar, a pergunta ``voce tem esquizofrenia?'' pode ser menos util que:
\begin{itemize}
    \item Quais situacoes aumentam sua vulnerabilidade?
    \item Quais recursos voce possui para manter estabilidade?
    \item Como podemos fortalecer sua capacidade de atravessar momentos dificeis?
\end{itemize}
\end{observacao}

%%%%%%%%%%%%%%%%%%%%%%%%%%%%%%%%%%%%%%%%%%%%%%%%%%%%%%%%%%%%%%%%%%%%%%%%%%%%%%%%%%
\section{Implicacoes Clinicas e Eticas}
\label{sec:implicacoes-eticas}
\index{Responsabilidade diagnostica}
\index{Humildade epistemica}

\begin{conceitochave}[title={A Responsabilidade do Diagnostico}]
Diagnosticar alguem como psicotico e, em certo sentido, \textit{constitui-lo} como psicotico --- atribuir-lhe identidade, inseri-lo em trajetoria institucional, autorizar intervencoes que modificarao sua experiencia e sua biografia.

Esta responsabilidade nao paralisa; \textbf{informa}.
\end{conceitochave}

O reconhecimento de que nossos criterios sao historica e culturalmente situados fundamenta atitude de \termotecnico{humildade epistemica} --- consciencia de que podemos estar errados, de que o que hoje classificamos como delirio pode, amanha, ser reconhecido como visao antecipada.

%%%%%%%%%%%%%%%%%%%%%%%%%%%%%%%%%%%%%%%%%%%%%%%%%%%%%%%%%%%%%%%%%%%%%%%%%%%%%%%%%%
\begin{sintese}
\textbf{Ideias centrais deste capitulo:}
\begin{enumerate}
    \item A loucura e designacao social, nao apenas estado mental interno.
    \item Experiencias psicoticas distribuem-se em continuum na populacao.
    \item Categorias diagnosticas sao historicamente contingentes.
    \item A semiologia fenomenologica permite diferenciacao estrutural.
    \item Estruturas liminares oscilam entre posicoes psicoticas e nao-psicoticas.
    \item O diagnostico carrega responsabilidade etica que exige humildade.
\end{enumerate}

\vspace{0.5cm}
\textbf{Conceitos-chave:} Continuum psicotico $\bullet$ Perturbacao da ipseidade $\bullet$ Incompreensibilidade jasperiana $\bullet$ Solipsismo $\bullet$ Liminaridade $\bullet$ Humildade epistemica
\end{sintese}

\vspace{1cm}

\begin{center}
\large\itshape
O ``inferno'' de que fala Sartre nao e inevitavel; e possibilidade que se atualiza quando o olhar do outro e julgador, redutor, fixador. O olhar clinico pode ser diferente: pode ser olhar que acolhe a complexidade, que resiste a categorizacao precipitada, que reconhece humanidade mesmo na experiencia mais estranha.
\end{center}
