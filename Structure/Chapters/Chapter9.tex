% !TEX root = ../../BookTemplate.tex
%%%%%%%%%%%%%%%%%%%%%%%%%%%%%%%%%%%%%%%%%%%%%%%%%%%%%%%%%%%%%%%%%%%%%%%%%%%%%%%%%%
\chapter{Para Quem a Medicação Trabalha?}
\label{cap:para-quem}

\begin{center}
\textit{``A pergunta que deveria preceder toda prescrição psiquiátrica, mas que raramente é formulada.''}
\end{center}

\vspace{1cm}

%%%%%%%%%%%%%%%%%%%%%%%%%%%%%%%%%%%%%%%%%%%%%%%%%%%%%%%%%%%%%%%%%%%%%%%%%%%%%%%%%%
\section{A Pergunta Não Formulada}
\label{sec:pergunta-nao-formulada}
\index{Contenção química}
\index{Medicalização institucional}

\begin{casoclinico}[title={O Idoso ``Agitado''}]
Um idoso com demência em instituição de longa permanência recebe um antipsicótico para ``agitação''. A agitação cessa.

\textbf{Objetivo terapêutico alcançado?}

Mas a quem beneficia que esse idoso não esteja mais agitado?
\begin{itemize}
    \item A ele --- que agora está sedado, com risco aumentado de queda, pneumonia, morte?
    \item Ou à instituição --- que agora não precisa lidar com comportamento desafiador, investigar a causa, alocar funcionários para cuidado individualizado?
\end{itemize}
\end{casoclinico}

Este capítulo examina duas dimensões interligadas:
\begin{enumerate}
    \item A \textbf{banalização} de certas medicações para usos além de suas indicações
    \item A \textbf{pressuposição} de que a medicação é etapa em processo de cuidado, quando frequentemente é substituto desse processo
\end{enumerate}

%%%%%%%%%%%%%%%%%%%%%%%%%%%%%%%%%%%%%%%%%%%%%%%%%%%%%%%%%%%%%%%%%%%%%%%%%%%%%%%%%%
\section{A Banalização do Antipsicótico}
\label{sec:banalizacao}
\index{Quetiapina}
\index{Uso off-label}

\subsection{Quetiapina: De Antipsicótico a ``Sonífero de Luxo''}

A quetiapina (Seroquel) foi aprovada para tratamento de esquizofrenia, transtorno bipolar e, como adjuvante, depressão maior. São condições graves, que afetam profundamente a vida de quem as tem. O antipsicótico, nesse contexto, é ferramenta potente para situações potentes — não panaceia de uso cotidiano.

E no entanto.

\begin{figure}[H]
\centering
\begin{tikzpicture}[
    node distance=0.8cm,
    stat/.style={rectangle, draw=BrickRed, fill=boxred, text width=10cm, text centered, rounded corners, minimum height=0.8cm, font=\small}
]
    \node[stat] (swiss) {\textbf{Suíça:} 94,3\% das prescrições de quetiapina em lares são off-label (Müller et al., 2020)};
    \node[stat, below=of swiss] (can) {\textbf{Canadá:} Prescrições por médicos de família aumentaram 300\% entre 2005-2012};
    \node[stat, below=of can] (eua) {\textbf{EUA:} 6,4 milhões (2013) para 8,2 milhões (2021) --- antipsicótico mais prescrito};
\end{tikzpicture}
\caption{A domesticação da quetiapina}
\label{fig:quetiapina}
\end{figure}

\begin{observacao}
Estamos diante de epidemia de psicose? De explosão de transtorno bipolar?

\textbf{Não.} Estamos diante de algo mais prosaico e mais perturbador: a domesticação de uma droga potente para usos de conveniência.
\end{observacao}

\subsection{A Contenção Química em Instituições}

Análise de mais de 12.000 lares de idosos nos EUA (2024):
\begin{itemize}
    \item Mais de 1 em cada 5 residentes recebendo antipsicótico
    \item Taxa de prescrição quase 10x maior que taxa de diagnósticos apropriados
\end{itemize}

\begin{conceitochave}[title={Black Box Warning da FDA}]
Antipsicóticos em idosos com demência: aumento de mortalidade de \textbf{1,6 a 1,7 vezes}.

A relatora especial da ONU para direitos de pessoas idosas, \autor{Claudia Mahler} (2024), recomendou a \textbf{proibição de contenções químicas} em instituições de cuidado.
\end{conceitochave}

\subsection{Para Quem É Conveniente Que o Idoso Durma?}

\begin{questao}
A pergunta que não se faz: \textbf{por que} esse idoso está agitado?
\begin{itemize}
    \item Está com dor não tratada?
    \item Está com infecção urinária --- causa comum de confusão em idosos?
    \item Está assustado em ambiente desconhecido?
    \item Está tentando comunicar necessidade que não consegue verbalizar?
\end{itemize}

Um ensaio clínico em 18 lares noruegueses demonstrou que abordagem para tratamento de \textbf{dor} pode reduzir sintomas comportamentais em demência.

Tratar dor exige avaliação, atenção, tempo. O antipsicótico é mais rápido.
\end{questao}

%%%%%%%%%%%%%%%%%%%%%%%%%%%%%%%%%%%%%%%%%%%%%%%%%%%%%%%%%%%%%%%%%%%%%%%%%%%%%%%%%%
\section{O Controle Disfarçado de Tratamento}
\label{sec:controle}
\index{Manejável}
\index{Sedar vs. tratar}

\subsection{A Lógica do ``Manejável''}

\begin{table}[H]
\centering
\caption{A inversão semântica do ``tratamento''}
\label{tab:manejavel}
\begin{tabular}{p{5.5cm}p{5.5cm}}
\toprule
\textbf{Linguagem} & \textbf{Realidade} \\
\midrule
``Paciente mais manejável'' & Manejável por quem? \\
``Comportamento controlado'' & Controlado para benefício de quem? \\
``Situação estabilizada'' & Estabilizada a que custo? \\
\bottomrule
\end{tabular}
\end{table}

\subsection{A Diferença Entre Sedar e Tratar}

\begin{definicaobox}[title={Sedar vs. Tratar}]
\textbf{Tratar:} Abordar a causa do sofrimento.

\textbf{Sedar:} Suprimir sua expressão.

São operações radicalmente distintas, com implicações éticas igualmente distintas.
\end{definicaobox}

Quando idoso com demência recebe antipsicótico para ``agitação'':
\begin{itemize}
    \item A demência não melhora (antipsicóticos podem acelerar declínio cognitivo)
    \item A causa da agitação não é investigada
    \item O que acontece é \textbf{sedacao}: redução global de atividade
\end{itemize}

\subsection{O Risco Que Não Entra na Conversa}

Antipsicóticos em idosos com demência aumentam risco de morte. Isso não é hipótese; é fato estabelecido por múltiplos estudos, reconhecido por agências regulatórias, motivo de \textit{black box warnings} nas bulas. A magnitude do risco — aumento de 60-70\% na mortalidade em estudos de 10-12 semanas — é substancial.

E no entanto, essa informação frequentemente não chega à família que consente (ou não consente, quando a prescrição é feita sem consulta) com o uso da medicação. A conversa, quando existe, tende a ser sobre ``ajudar com a agitação'', ``melhorar a qualidade de vida'', ``facilitar o cuidado''. Não sobre aumento de risco de pneumonia, de eventos cardiovasculares, de morte.

Novamente: para quem a medicação trabalha? Se a família soubesse que a droga que vai ``acalmar'' seu ente querido também aumenta significativamente a chance de ele morrer nos próximos meses, consentiria da mesma forma? Se o médico tivesse que pesar, explicitamente, ``agitação controlada por alguns meses'' contra ``vida possivelmente encurtada'', a prescrição seria tão automática?

%%%%%%%%%%%%%%%%%%%%%%%%%%%%%%%%%%%%%%%%%%%%%%%%%%%%%%%%%%%%%%%%%%%%%%%%%%%%%%%%%%
\section{A Janela Que Ninguém Atravessa}
\label{sec:janela}
\index{Split treatment}
\index{Psicoterapia}

\subsection{O Modelo Implícito}

Há um modelo implícito que sustenta grande parte da prática psicofarmacológica contemporânea. Funciona assim:

\begin{figure}[H]
\centering
\begin{tikzpicture}[
    node distance=0.8cm,
    step/.style={rectangle, draw=sectioncolor, fill=boxblue, text width=9cm, text centered, rounded corners, minimum height=0.8cm, font=\small}
]
    \node[step] (1) {1. Paciente em crise};
    \node[step, below=of 1] (2) {2. Medicação estabiliza (reduz sintomas agudos)};
    \node[step, below=of 2] (3) {3. Abre-se ``janela terapêutica''};
    \node[step, below=of 3] (4) {4. Psicoterapia promove transformação duradoura};
    \node[step, below=of 4] (5) {5. Paciente melhora de forma sustentável};

    \draw[-Stealth] (1) -- (2);
    \draw[-Stealth] (2) -- (3);
    \draw[-Stealth, dashed, BrickRed, thick] (3) -- node[right, font=\small\itshape, text=BrickRed] {Problema: quem atravessa?} (4);
    \draw[-Stealth] (4) -- (5);
\end{tikzpicture}
\caption{O modelo implícito e seu ponto frágil}
\label{fig:janela}
\end{figure}

\subsection{O Problema do Passo 3}

Quem fará o trabalho psicoterapêutico? Na teoria, o mesmo profissional que prescreveu, ou um colega em trabalho colaborativo. Na prática, há uma cisão crescente entre quem medica e quem faz psicoterapia.

\begin{table}[H]
\centering
\caption{Estatísticas sobre acesso a psicoterapia}
\label{tab:acesso}
\begin{tabular}{p{7cm}c}
\toprule
\textbf{Indicador} & \textbf{Dado} \\
\midrule
Pacientes recebendo psicoterapia de psiquiatras (EUA, 2021) & 34\% \\
Pacientes em tratamento ambulatorial recebendo \textit{apenas} medicação (EUA) & 62\% \\
População com transtorno mental que recebe algum tratamento & $\sim$50\% \\
\bottomrule
\end{tabular}
\end{table}

\begin{observacao}
A janela terapêutica se abre --- e permanece vazia.

A medicação se torna não facilitador de transformação, mas \textbf{substituto} dela --- paliativo permanente em vez de ponte para mudança.
\end{observacao}

\subsection{O Sistema Que a Medicação Serve}

A quem esse arranjo beneficia? Ao paciente, que recebe tratamento incompleto? Ou ao sistema, que pode processar mais pacientes em menos tempo, que pode contabilizar ``casos tratados'' sem investir no trabalho lento e caro da psicoterapia?

A pergunta não implica conspiração ou má-fé. Implica reconhecer que sistemas têm lógicas próprias, que frequentemente divergem das necessidades dos indivíduos que deveriam servir. Um sistema de saúde sob pressão de custos tende a preferir soluções mais baratas (medicação) a soluções mais caras (psicoterapia). Um sistema de formação médica sob pressão de tempo tende a ensinar o que pode ser transmitido rapidamente (prescrição) mais que o que exige anos de supervisão (psicoterapia). Um sistema de remuneração que paga por ato tende a multiplicar atos breves (consultas de 15 minutos) em vez de poucos atos longos (sessões de psicoterapia).

%%%%%%%%%%%%%%%%%%%%%%%%%%%%%%%%%%%%%%%%%%%%%%%%%%%%%%%%%%%%%%%%%%%%%%%%%%%%%%%%%%
\section{A Psiquiatria Que Não Conversa}
\label{sec:nao-conversa}

Retornamos ao capítulo primeiro deste livro:

\begin{itemize}
    \item A consulta de 15 minutos não permite conversa
    \item A formação que enfatiza psicofarmacologia em detrimento de psicoterapia não prepara para conversa
    \item O sistema de remuneração que paga por procedimentos não incentiva conversa
\end{itemize}

\begin{questao}
Quando a medicação ``funciona'' (reduz sintomas) mas não há ninguém para atravessar a janela, o que acontece?

O paciente está ``tratado'' segundo critérios do sistema. Mas a dimensão existencial do sofrimento --- o que \autor{Jaspers} chamaria de \textit{Verstehen} --- permanece intocada.
\end{questao}

%%%%%%%%%%%%%%%%%%%%%%%%%%%%%%%%%%%%%%%%%%%%%%%%%%%%%%%%%%%%%%%%%%%%%%%%%%%%%%%%%%
\section{Uma Farmacologia a Serviço do Sujeito}
\label{sec:farmacologia-servico}

\subsection{Critérios para Prescrição Centrada no Paciente}

Se a crítica desenvolvida neste capítulo é válida, que implicações práticas se seguem? Proponho critérios para o que chamarei de \textit{prescrição centrada no paciente} --- farmacologia que serve ao sujeito, não ao sistema.

\begin{table}[H]
\centering
\caption{Critérios para prescrição centrada no paciente}
\label{tab:criterios}
\begin{tabular}{p{3.5cm}p{8cm}}
\toprule
\textbf{Critério} & \textbf{Pergunta a Fazer} \\
\midrule
Clareza sobre indicação & Esta medicação trata condição do paciente ou maneja situação que incomoda outros? \\
\addlinespace
Investigação de causas & O que o comportamento comunica? Agitação pode ser dor. Agressividade pode ser medo. \\
\addlinespace
Continuidade de cuidado & Haverá alguém para atravessar a janela terapêutica? \\
\addlinespace
Consentimento real & O paciente foi informado de riscos, alternativas e limitações? \\
\addlinespace
Reavaliação contínua & Esta medicação ainda serve ao paciente ou continua por inércia? \\
\bottomrule
\end{tabular}
\end{table}

\subsection{A Pergunta Que Deveria Preceder Toda Receita}

O capítulo começou com uma pergunta: \textit{para quem esta medicação trabalha?}

A resposta honesta, em muitos casos, é desconfortável. A medicação frequentemente trabalha para o sistema --- para a instituição que precisa de corpos dóceis, para o psiquiatra que precisa de consultas rápidas, para o plano de saúde que prefere pílulas a sessões, para a família que quer ``resolver'' o problema sem enfrentar suas dimensões relacionais.

Isso não significa que medicação seja sempre inapropriada, ou que todo uso serve a interesses espúrios. Significa que a pergunta deve ser feita --- explicitamente, conscientemente, a cada prescrição. E que a resposta deve ser, genuinamente, que a medicação serve ao paciente. Não ao paciente em abstrato, como categoria diagnóstica a ser tratada, mas ao paciente concreto, como pessoa com história, valores, contexto, e direito de participar das decisões sobre sua própria vida.

%%%%%%%%%%%%%%%%%%%%%%%%%%%%%%%%%%%%%%%%%%%%%%%%%%%%%%%%%%%%%%%%%%%%%%%%%%%%%%%%%%
\section{Conclusão: A Medicação e Seus Senhores}
\label{sec:conclusao}

A farmacologia psiquiátrica é conquista notável. Drogas que há meio século não existiam permitem hoje que pessoas com condições graves vivam vidas que antes eram impossíveis. Reconhecer isso não é capitular ao farmacologismo; é reconhecer realidade.

Mas reconhecer conquistas não impede crítica. A crítica desenvolvida nestes dois últimos capítulos --- sobre tempo e preço, sobre a quem a medicação serve --- não é antifarmacológica. É, se quisermos usar termo, \textit{farmacologia crítica}: reflexão sobre os usos da farmacologia, sobre os valores que incorpora, sobre os interesses que serve.

A conclusão não é ``não prescreva''. É ``pergunte, antes de prescrever''. Pergunte pelo tempo do paciente, não apenas pelo tempo da molécula. Pergunte pelo preço que o paciente está disposto a pagar, não pelo preço que o prescritor considera aceitável. Pergunte para quem a medicação trabalha --- e só prescreva quando a resposta for, honestamente, ``para o paciente''.

Há, nessa exigência, algo de retorno ao fundamento hipocrático: \textit{primum non nocere}, primeiro não prejudicar. Mas também algo mais: reconhecimento de que o paciente é sujeito de sua vida, não objeto de intervenção. Que tem valores próprios, tempos próprios, prioridades próprias. E que a medicina --- incluindo a psiquiatria, incluindo a farmacologia --- existe para servir a esses valores, tempos e prioridades. Não o contrário.

%%%%%%%%%%%%%%%%%%%%%%%%%%%%%%%%%%%%%%%%%%%%%%%%%%%%%%%%%%%%%%%%%%%%%%%%%%%%%%%%%%
\begin{sintese}
\textbf{Ideias centrais deste capítulo:}
\begin{enumerate}
    \item A pergunta ``para quem a medicação trabalha?'' deve preceder toda prescrição.
    \item A banalização de antipsicóticos serve frequentemente ao sistema, não ao paciente.
    \item Contenção química em instituições aumenta mortalidade e viola direitos.
    \item Sedar é radicalmente distinto de tratar.
    \item A janela terapêutica frequentemente permanece vazia por falta de psicoterapia.
    \item Prescrição centrada no paciente exige clareza sobre indicação, investigação de causas, consentimento real e reavaliação contínua.
\end{enumerate}

\vspace{0.5cm}
\textbf{Conceitos-chave:} Contenção química $\bullet$ Sedar vs. tratar $\bullet$ Janela terapêutica $\bullet$ Split treatment $\bullet$ Prescrição centrada no paciente
\end{sintese}

\vspace{1cm}

\begin{center}
\large\itshape
A farmacologia psiquiátrica é conquista notável. Mas a medicação é ferramenta --- e, como toda ferramenta, pode ser bem ou mal usada. O critério de bom uso é simples de formular e difícil de implementar: a medicação deve servir ao sujeito.
\end{center}
