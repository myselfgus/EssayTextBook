% !TEX root = ../../BookTemplate.tex
%%%%%%%%%%%%%%%%%%%%%%%%%%%%%%%%%%%%%%%%%%%%%%%%%%%%%%%%%%%%%%%%%%%%%%%%%%%%%%%%%%
\chapter{Para Quem a Medicacao Trabalha?}
\label{cap:para-quem}

\begin{center}
\textit{``A pergunta que deveria preceder toda prescricao psiquiatrica, mas que raramente e formulada.''}
\end{center}

\vspace{1cm}

%%%%%%%%%%%%%%%%%%%%%%%%%%%%%%%%%%%%%%%%%%%%%%%%%%%%%%%%%%%%%%%%%%%%%%%%%%%%%%%%%%
\section{A Pergunta Nao Formulada}
\label{sec:pergunta-nao-formulada}
\index{Contencao quimica}
\index{Medicalizacao institucional}

\begin{casoclinico}[title={O Idoso ``Agitado''}]
Um idoso com demencia em instituicao de longa permanencia recebe um antipsicotico para ``agitacao''. A agitacao cessa.

\textbf{Objetivo terapeutico alcancado?}

Mas a quem beneficia que esse idoso nao esteja mais agitado?
\begin{itemize}
    \item A ele --- que agora esta sedado, com risco aumentado de queda, pneumonia, morte?
    \item Ou a instituicao --- que agora nao precisa lidar com comportamento desafiador, investigar a causa, alocar funcionarios para cuidado individualizado?
\end{itemize}
\end{casoclinico}

Este capitulo examina duas dimensoes interligadas:
\begin{enumerate}
    \item A \textbf{banalizacao} de certas medicacoes para usos alem de suas indicacoes
    \item A \textbf{pressuposicao} de que a medicacao e etapa em processo de cuidado, quando frequentemente e substituto desse processo
\end{enumerate}

%%%%%%%%%%%%%%%%%%%%%%%%%%%%%%%%%%%%%%%%%%%%%%%%%%%%%%%%%%%%%%%%%%%%%%%%%%%%%%%%%%
\section{A Banalizacao do Antipsicotico}
\label{sec:banalizacao}
\index{Quetiapina}
\index{Uso off-label}

\subsection{Quetiapina: De Antipsicotico a ``Sonifero de Luxo''}

\begin{figure}[H]
\centering
\begin{tikzpicture}[
    node distance=0.8cm,
    stat/.style={rectangle, draw=BrickRed, fill=boxred, text width=10cm, text centered, rounded corners, minimum height=0.8cm, font=\small}
]
    \node[stat] (swiss) {\textbf{Suica:} 94,3\% das prescricoes de quetiapina em lares sao off-label (Muller et al., 2020)};
    \node[stat, below=of swiss] (can) {\textbf{Canada:} Prescricoes por medicos de familia aumentaram 300\% entre 2005-2012};
    \node[stat, below=of can] (eua) {\textbf{EUA:} 6,4 milhoes (2013) para 8,2 milhoes (2021) --- antipsicotico mais prescrito};
\end{tikzpicture}
\caption{A domesticacao da quetiapina}
\label{fig:quetiapina}
\end{figure}

\begin{observacao}
Estamos diante de epidemia de psicose? De explosao de transtorno bipolar?

\textbf{Nao.} Estamos diante de algo mais prosáico e mais perturbador: a domesticacao de uma droga potente para usos de conveniencia.
\end{observacao}

\subsection{A Contencao Quimica em Instituicoes}

Analise de mais de 12.000 lares de idosos nos EUA (2024):
\begin{itemize}
    \item Mais de 1 em cada 5 residentes recebendo antipsicotico
    \item Taxa de prescricao quase 10x maior que taxa de diagnosticos apropriados
\end{itemize}

\begin{conceitochave}[title={Black Box Warning da FDA}]
Antipsicoticos em idosos com demencia: aumento de mortalidade de \textbf{1,6 a 1,7 vezes}.

A relatora especial da ONU para direitos de pessoas idosas, \autor{Claudia Mahler} (2024), recomendou a \textbf{proibicao de contencoes quimicas} em instituicoes de cuidado.
\end{conceitochave}

\subsection{Para Quem E Conveniente Que o Idoso Durma?}

\begin{questao}
A pergunta que nao se faz: \textbf{por que} esse idoso esta agitado?
\begin{itemize}
    \item Esta com dor nao tratada?
    \item Esta com infeccao urinaria --- causa comum de confusao em idosos?
    \item Esta assustado em ambiente desconhecido?
    \item Esta tentando comunicar necessidade que nao consegue verbalizar?
\end{itemize}

Um ensaio clinico em 18 lares noruegueses demonstrou que abordagem para tratamento de \textbf{dor} pode reduzir sintomas comportamentais em demencia.

Tratar dor exige avaliacao, atencao, tempo. O antipsicotico e mais rapido.
\end{questao}

%%%%%%%%%%%%%%%%%%%%%%%%%%%%%%%%%%%%%%%%%%%%%%%%%%%%%%%%%%%%%%%%%%%%%%%%%%%%%%%%%%
\section{O Controle Disfarcado de Tratamento}
\label{sec:controle}
\index{Manejavel}
\index{Sedar vs. tratar}

\subsection{A Logica do ``Manejavel''}

\begin{table}[H]
\centering
\caption{A inversao semantica do ``tratamento''}
\label{tab:manejavel}
\begin{tabular}{p{5.5cm}p{5.5cm}}
\toprule
\textbf{Linguagem} & \textbf{Realidade} \\
\midrule
``Paciente mais manejavel'' & Manejavel por quem? \\
``Comportamento controlado'' & Controlado para beneficio de quem? \\
``Situacao estabilizada'' & Estabilizada a que custo? \\
\bottomrule
\end{tabular}
\end{table}

\subsection{A Diferenca Entre Sedar e Tratar}

\begin{definicaobox}[title={Sedar vs. Tratar}]
\textbf{Tratar:} Abordar a causa do sofrimento.

\textbf{Sedar:} Suprimir sua expressao.

Sao operacoes radicalmente distintas, com implicacoes eticas igualmente distintas.
\end{definicaobox}

Quando idoso com demencia recebe antipsicotico para ``agitacao'':
\begin{itemize}
    \item A demencia nao melhora (antipsicoticos podem acelerar declinio cognitivo)
    \item A causa da agitacao nao e investigada
    \item O que acontece e \textbf{sedacao}: reducao global de atividade
\end{itemize}

%%%%%%%%%%%%%%%%%%%%%%%%%%%%%%%%%%%%%%%%%%%%%%%%%%%%%%%%%%%%%%%%%%%%%%%%%%%%%%%%%%
\section{A Janela Que Ninguem Atravessa}
\label{sec:janela}
\index{Split treatment}
\index{Psicoterapia}

\subsection{O Modelo Implicito}

\begin{figure}[H]
\centering
\begin{tikzpicture}[
    node distance=0.8cm,
    step/.style={rectangle, draw=sectioncolor, fill=boxblue, text width=9cm, text centered, rounded corners, minimum height=0.8cm, font=\small}
]
    \node[step] (1) {1. Paciente em crise};
    \node[step, below=of 1] (2) {2. Medicacao estabiliza (reduz sintomas agudos)};
    \node[step, below=of 2] (3) {3. Abre-se ``janela terapeutica''};
    \node[step, below=of 3] (4) {4. Psicoterapia promove transformacao duradoura};
    \node[step, below=of 4] (5) {5. Paciente melhora de forma sustentavel};

    \draw[-Stealth] (1) -- (2);
    \draw[-Stealth] (2) -- (3);
    \draw[-Stealth, dashed, BrickRed, thick] (3) -- node[right, font=\small\itshape, text=BrickRed] {Problema: quem atravessa?} (4);
    \draw[-Stealth] (4) -- (5);
\end{tikzpicture}
\caption{O modelo implicito e seu ponto fragil}
\label{fig:janela}
\end{figure}

\subsection{O Problema do Passo 3}

\begin{table}[H]
\centering
\caption{Estatisticas sobre acesso a psicoterapia}
\label{tab:acesso}
\begin{tabular}{p{7cm}c}
\toprule
\textbf{Indicador} & \textbf{Dado} \\
\midrule
Pacientes recebendo psicoterapia de psiquiatras (EUA, 2021) & 34\% \\
Pacientes em tratamento ambulatorial recebendo \textit{apenas} medicacao (EUA) & 62\% \\
Populacao com transtorno mental que recebe algum tratamento & $\sim$50\% \\
\bottomrule
\end{tabular}
\end{table}

\begin{observacao}
A janela terapeutica se abre --- e permanece vazia.

A medicacao se torna nao facilitador de transformacao, mas \textbf{substituto} dela --- paliativo permanente em vez de ponte para mudanca.
\end{observacao}

%%%%%%%%%%%%%%%%%%%%%%%%%%%%%%%%%%%%%%%%%%%%%%%%%%%%%%%%%%%%%%%%%%%%%%%%%%%%%%%%%%
\section{A Psiquiatria Que Nao Conversa}
\label{sec:nao-conversa}

Retornamos ao capitulo primeiro deste livro:

\begin{itemize}
    \item A consulta de 15 minutos nao permite conversa
    \item A formacao que enfatiza psicofarmacologia em detrimento de psicoterapia nao prepara para conversa
    \item O sistema de remuneracao que paga por procedimentos nao incentiva conversa
\end{itemize}

\begin{questao}
Quando a medicacao ``funciona'' (reduz sintomas) mas nao ha ninguem para atravessar a janela, o que acontece?

O paciente esta ``tratado'' segundo criterios do sistema. Mas a dimensao existencial do sofrimento --- o que \autor{Jaspers} chamaria de \textit{Verstehen} --- permanece intocada.
\end{questao}

%%%%%%%%%%%%%%%%%%%%%%%%%%%%%%%%%%%%%%%%%%%%%%%%%%%%%%%%%%%%%%%%%%%%%%%%%%%%%%%%%%
\section{Prescricao Centrada no Paciente}
\label{sec:centrada-paciente}

\begin{table}[H]
\centering
\caption{Criterios para prescricao centrada no paciente}
\label{tab:criterios}
\begin{tabular}{p{3.5cm}p{8cm}}
\toprule
\textbf{Criterio} & \textbf{Pergunta a Fazer} \\
\midrule
Clareza sobre indicacao & Esta medicacao trata condicao do paciente ou maneja situacao que incomoda outros? \\
\addlinespace
Investigacao de causas & O que o comportamento comunica? Agitacao pode ser dor. Agressividade pode ser medo. \\
\addlinespace
Continuidade de cuidado & Havera alguem para atravessar a janela terapeutica? \\
\addlinespace
Consentimento real & O paciente foi informado de riscos, alternativas e limitacoes? \\
\addlinespace
Reavaliacao continua & Esta medicacao ainda serve ao paciente ou continua por inercia? \\
\bottomrule
\end{tabular}
\end{table}

%%%%%%%%%%%%%%%%%%%%%%%%%%%%%%%%%%%%%%%%%%%%%%%%%%%%%%%%%%%%%%%%%%%%%%%%%%%%%%%%%%
\section{Retorno ao Fundamento}
\label{sec:fundamento}

Este livro comecou com uma provocacao: a psiquiatria contemporanea faz as perguntas erradas. Pergunta por criterios diagnosticos, nao pelo sentido do sofrimento. Pergunta por sintomas a serem eliminados, nao por pessoa a ser compreendida.

\begin{conceitochave}[title={O Criterio de Bom Uso}]
A medicacao deve servir ao \textbf{sujeito}:
\begin{itemize}
    \item Nao ao sistema
    \item Nao a conveniencia
    \item Nao a metrica de ``casos tratados''
\end{itemize}

Ao sujeito --- em sua singularidade, em sua dignidade, em seu direito de ser tratado como \textbf{fim, nunca apenas como meio}.
\end{conceitochave}

%%%%%%%%%%%%%%%%%%%%%%%%%%%%%%%%%%%%%%%%%%%%%%%%%%%%%%%%%%%%%%%%%%%%%%%%%%%%%%%%%%
\begin{sintese}
\textbf{Ideias centrais deste capitulo:}
\begin{enumerate}
    \item A pergunta ``para quem a medicacao trabalha?'' deve preceder toda prescricao.
    \item A banalizacao de antipsicoticos serve frequentemente ao sistema, nao ao paciente.
    \item Contencao quimica em instituicoes aumenta mortalidade e viola direitos.
    \item Sedar e radicalmente distinto de tratar.
    \item A janela terapeutica frequentemente permanece vazia por falta de psicoterapia.
    \item Prescricao centrada no paciente exige clareza sobre indicacao, investigacao de causas, consentimento real e reavaliacao continua.
\end{enumerate}

\vspace{0.5cm}
\textbf{Conceitos-chave:} Contencao quimica $\bullet$ Sedar vs. tratar $\bullet$ Janela terapeutica $\bullet$ Split treatment $\bullet$ Prescricao centrada no paciente
\end{sintese}

\vspace{1cm}

\begin{center}
\large\itshape
A farmacologia psiquiatrica e conquista notavel. Mas a medicacao e ferramenta --- e, como toda ferramenta, pode ser bem ou mal usada. O criterio de bom uso e simples de formular e dificil de implementar: a medicacao deve servir ao sujeito.
\end{center}
