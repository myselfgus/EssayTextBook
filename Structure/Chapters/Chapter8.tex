% !TEX root = ../../BookTemplate.tex
%%%%%%%%%%%%%%%%%%%%%%%%%%%%%%%%%%%%%%%%%%%%%%%%%%%%%%%%%%%%%%%%%%%%%%%%%%%%%%%%%%
\chapter{O Tempo e o Preço}
\label{cap:tempo-preco}

\begin{center}
\textit{``Este medicamento leva algumas semanas para fazer efeito. Tenha paciência.''}\\[0.3cm]
--- Frase rotineira que ninguém mais questiona
\end{center}

\vspace{1cm}

%%%%%%%%%%%%%%%%%%%%%%%%%%%%%%%%%%%%%%%%%%%%%%%%%%%%%%%%%%%%%%%%%%%%%%%%%%%%%%%%%%
\section{A Fila Invisível}
\label{sec:fila}
\index{Temporalidade}
\index{Latença farmacológica}

Uma pessoa que talvez tenha levado meses para superar o estigma, semanas para conseguir consulta, anos para admitir que precisava de ajuda --- finalmente senta-se diante do psiquiatra. E então recebe uma receita acompanhada de instrução para ter \textit{paciência}.

\begin{questao}
A pessoa que chegou em crise, que não consegue trabalhar, que não consegue dormir, que talvez não consiga ver sentido em continuar vivendo --- essa pessoa é instruída a aguardar.

A entrar em uma fila invisível onde seu sofrimento será processado no \textbf{tempo da molécula}, não no \textbf{tempo da urgência}.
\end{questao}

Este capítulo examina duas dimensões frequentemente negligenciadas:
\begin{itemize}
    \item A \textbf{temporalidade} do tratamento
    \item O \textbf{preço} que se paga por ele
\end{itemize}

%%%%%%%%%%%%%%%%%%%%%%%%%%%%%%%%%%%%%%%%%%%%%%%%%%%%%%%%%%%%%%%%%%%%%%%%%%%%%%%%%%
\section{O Tempo do Tratamento e o Tempo do Sofrimento}
\label{sec:tempo-sofrimento}

\subsection{A Latência Como Dogma}

A afirmação de que antidepressivos levam duas a quatro semanas para ``começar a fazer efeito'' adquiriu estatuto de verdade inquestionável.

\begin{observacao}
\autor{Machado-Vieira} et al. (2023): ``Antidepressivos convencionais que atuam em receptores monoaminérgicos requerem várias semanas para serem eficazes. Este atraso representa um \textbf{problema significativo} nos tratamentos atualmente disponíveis para depressão grave.''

\textbf{O que significa, concretamente?}
\begin{itemize}
    \item Pessoa com risco de suicídio recebe tratamento sem alívio imediato
    \item Capacidade funcional permanece comprometida por semanas
    \item Intervalo entre buscar ajuda e recebê-la pode estender-se por meses
\end{itemize}
\end{observacao}

\subsection{A Prova de Que Poderia Ser Diferente}

\begin{casoclinico}[title={A Revolução da Ketamina}]
Em 2000, \autor{Berman} e colaboradores publicaram estudo que deveria ter revolucionado a psiquiatria:

Uma dose subanestésica de ketamina produziu melhora significativa dos sintomas depressivos em \textbf{horas}, não semanas.

\textbf{Implicação:} Se resposta rápida é possível, por que tratamentos de ação lenta permanecem como primeira linha?
\end{casoclinico}

\begin{figure}[H]
\centering
\begin{tikzpicture}[scale=0.9]
    % Timeline
    \draw[thick, -Stealth] (0,0) -- (12,0) node[right] {Tempo};

    % Ketamine
    \draw[thick, OliveGreen] (0.5,0.2) -- (0.5,2);
    \node[above, font=\small, OliveGreen] at (0.5,2) {Ketamina};
    \node[below, font=\footnotesize] at (0.5,-0.3) {24h};

    % SSRIs
    \draw[thick, BrickRed] (4,0.2) -- (4,2);
    \draw[thick, BrickRed, dashed] (4,2) -- (8,2);
    \node[above, font=\small, BrickRed] at (6,2) {ISRSs};
    \node[below, font=\footnotesize] at (4,-0.3) {2-4 sem};
    \node[below, font=\footnotesize] at (8,-0.3) {6-8 sem};
\end{tikzpicture}
\caption{Comparação de latência: ketamina vs. antidepressivos convencionais}
\label{fig:latencia}
\end{figure}

\subsection{O Paciente Como Request em Fila}

\autor{Minkowski} (1933/1970), em seus estudos sobre tempo vivido, demonstrou que estados depressivos alteram profundamente a experiência temporal: o futuro se fecha, o presente se arrasta, cada minuto pesa como hora.

\begin{observacao}
Dizer a alguém nesse estado que ``espere algumas semanas'' é impor uma temporalidade objetiva --- o tempo do calendário, o tempo da farmacologia --- sobre uma temporalidade subjetiva radicalmente distinta.

É, em certo sentido, uma \textbf{violência epistemológica}: a afirmação de que o tempo do tratamento é mais real que o tempo do sofrimento.
\end{observacao}

%%%%%%%%%%%%%%%%%%%%%%%%%%%%%%%%%%%%%%%%%%%%%%%%%%%%%%%%%%%%%%%%%%%%%%%%%%%%%%%%%%
\section{O Preço Não Anunciado}
\label{sec:preco}
\index{Embotamento emocional}
\index{Disfunção sexual}

\subsection{Emotional Blunting: Efeito Colateral ou Mecanismo de Ação?}

Pacientes em uso de ISRSs frequentemente relatam uma espécie de anestesia afetiva: redução não apenas da tristeza patológica, mas da capacidade de experimentar todo o espectro emocional.

\begin{table}[H]
\centering
\caption{Emotional blunting: o que se ganha e o que se perde}
\label{tab:blunting}
\begin{tabular}{p{5cm}p{5cm}}
\toprule
\textbf{O Que Se Ganha} & \textbf{O Que Se Perde} \\
\midrule
Redução da tristeza patológica & Alegria \\
Redução da ansiedade & Excitação \\
Estabilidade emocional & Ternura \\
\  & Raiva apropriada \\
\  & Capacidade de aprendizado por reforço \\
\bottomrule
\end{tabular}
\end{table}

\begin{questao}
O \textit{emotional blunting} é \textbf{efeito colateral} dos ISRSs ou \textbf{mecanismo de ação}?

Se a redução da ansiedade vem acompanhada de redução da capacidade de sentir em geral, estamos \textit{tratando} a ansiedade ou \textit{suprimindo} a afetividade?
\end{questao}

\subsection{Disfunção Sexual Como Regra, Não Exceção}

\begin{conceitochave}[title={Estatísticas da Disfunção Sexual por ISRSs}]
Estudos estimam que \textbf{50\% a 70\%} dos pacientes em ISRSs experimentam alguma forma de comprometimento sexual:
\begin{itemize}
    \item Redução de libido
    \item Dificuldade de excitação
    \item Anorgasmia
    \item Ejaculação retardada
\end{itemize}

Não estamos falando de efeito raro. A disfunção sexual é, para ISRSs, \textbf{mais regra que exceção}.
\end{conceitochave}

\begin{observacao}
\textbf{Post-SSRI Sexual Dysfunction (PSSD):} Disfunção sexual que persiste após a descontinuação do medicamento, potencialmente de forma \textbf{permanente}.

A Agência Europeia de Medicamentos reconheceu o PSSD em 2019, exigindo atualização das bulas. A FDA americana, até dezembro de 2025, ainda não o fez.
\end{observacao}

\subsection{O Que Não É Dito}

\begin{table}[H]
\centering
\caption{O que é dito vs. o que não é dito ao prescrever ISRSs}
\label{tab:dito}
\begin{tabular}{p{5.5cm}p{5.5cm}}
\toprule
\textbf{O Que É Dito} & \textbf{O Que Não É Dito} \\
\midrule
Ajudará com ansiedade/depressão & Probabilidade de 50-70\% de disfunção sexual \\
Leva algumas semanas & Disfunção pode persistir após parar \\
Efeitos colaterais transitórios & Embotamento emocional, sentir-se ``anestesiado'' \\
Não causa dependência & Existem alternativas com perfil diferente \\
\  & (ex: bupropiona com menor incidência de disfunção sexual) \\
\bottomrule
\end{tabular}
\end{table}

%%%%%%%%%%%%%%%%%%%%%%%%%%%%%%%%%%%%%%%%%%%%%%%%%%%%%%%%%%%%%%%%%%%%%%%%%%%%%%%%%%
\section{Quem Decide O Que Vale O Quê?}
\label{sec:hierarquia-valores}
\index{Autonomia}
\index{Valores do paciente}

\subsection{A Hierarquia Implícita de Valores}

\begin{observacao}
Toda prescrição psiquiátrica implica uma hierarquia de valores:
\begin{itemize}
    \item Ao prescrever ISRS sabendo que provavelmente causará disfunção sexual: ``não ter ansiedade'' vale mais que ``ter vida sexual satisfatória''
    \item Ao prescrever antipsicótico sabendo que causará ganho de peso: ``não ter sintomas psicóticos'' vale mais que ``manter peso saudável''
\end{itemize}

\textbf{Essas hierarquias refletem valores de quem prescreve mais que de quem vai viver com as consequências.}
\end{observacao}

\subsection{O Paciente Como Objeto, Não Sujeito}

O paciente --- que veio ao consultório como sujeito de seu sofrimento, buscando tornar-se sujeito de sua cura --- é transformado em \textit{objeto} da prescrição.

\begin{itemize}
    \item ``Adesão ao tratamento'' --- o tratamento é dado, ao paciente cabe aderir
    \item ``Resposta à medicação'' --- como se fosse organismo que responde a estímulo
    \item ``Efeitos colaterais toleráveis'' --- sem perguntar a quem cabe definir tolerável
\end{itemize}

%%%%%%%%%%%%%%%%%%%%%%%%%%%%%%%%%%%%%%%%%%%%%%%%%%%%%%%%%%%%%%%%%%%%%%%%%%%%%%%%%%
\section{Uma Farmacologia Centrada no Sujeito}
\label{sec:farmacologia-sujeito}

\subsection{O Tempo do Paciente}

Uma farmacologia centrada no sujeito começaria por reconhecer que o tempo do tratamento deve dialogar com o tempo do sofrimento. Concretamente:
\begin{itemize}
    \item Avaliação explícita do risco durante o período de latência
    \item Estratégias de suporte intensivo nas primeiras semanas
    \item Consideração de tratamentos de ação rápida para casos de maior urgência
    \item Reconhecimento de que dizer a alguém em crise para ``ter paciência'' é resposta inadequada
\end{itemize}

\subsection{Os Valores do Paciente}

Antes de prescrever, perguntar quais valores o paciente prioriza. Não há resposta universal correta --- há respostas singulares que dependem de quem vai viver com as consequências.

\subsection{O Preço Explicitado}

Informar sobre disfunção sexual antes de prescrever ISRS. Discutir embotamento emocional como possibilidade real. Apresentar alternativas com perfis diferentes de benefícios e custos. Permitir que o paciente escolha qual preço está disposto a pagar.

\begin{table}[H]
\centering
\caption{Princípios de uma farmacologia centrada no sujeito}
\label{tab:principios}
\begin{tabular}{p{4cm}p{7.5cm}}
\toprule
\textbf{Princípio} & \textbf{Implementação} \\
\midrule
O tempo do paciente & Avaliação de risco durante latência; suporte intensivo inicial; considerar opções de ação rápida \\
\addlinespace
Os valores do paciente & Perguntar prioridades antes de prescrever; não há hierarquia universal de valores \\
\addlinespace
O preço explicitado & Informar sobre disfunção sexual, embotamento; apresentar alternativas \\
\addlinespace
Decisão genuinamente compartilhada & Médico e paciente deliberam juntos sobre meios e fins \\
\bottomrule
\end{tabular}
\end{table}

%%%%%%%%%%%%%%%%%%%%%%%%%%%%%%%%%%%%%%%%%%%%%%%%%%%%%%%%%%%%%%%%%%%%%%%%%%%%%%%%%%
\section{Conclusão: A Urgência e o Custo}
\label{sec:conclusao}

A psicofarmacologia contemporânea opera com um tempo e um sistema de valores que frequentemente não são os do paciente.

O tempo do tratamento é o tempo da molécula --- semanas para receptores se regularem. O tempo do sofrimento é outro --- urgente, insuportável, medido em minutos que não passam. Tratar o primeiro como dado natural e o segundo como variável a ser ``manejada'' é inverter prioridades.

O preço do tratamento é definido pelo prescritor --- efeitos colaterais são ``toleráveis'' quando quem prescreve não vai tolerá-los. Quem decide que não ter ansiedade vale mais que ter vida sexual? Quem decide que alívio sintomático vale mais que capacidade de sentir? Se o paciente não participa dessas decisões, não está sendo tratado como sujeito; está sendo tratado como objeto de intervenção farmacológica.

O próximo capítulo examinará outra dimensão dessa objetificação: a pergunta sobre \textit{para quem} a medicação realmente trabalha. Pois há casos --- perturbadoramente frequentes --- em que a medicação serve menos ao paciente que a toma do que ao sistema que o cerca.

%%%%%%%%%%%%%%%%%%%%%%%%%%%%%%%%%%%%%%%%%%%%%%%%%%%%%%%%%%%%%%%%%%%%%%%%%%%%%%%%%%
\begin{sintese}
\textbf{Ideias centrais deste capítulo:}
\begin{enumerate}
    \item A latência farmacológica é tratada como dogma, mas a ketamina prova que antidepressivos de ação rápida são possíveis.
    \item O tempo do tratamento (objetivo) frequentemente ignora o tempo do sofrimento (subjetivo).
    \item Emotional blunting e disfunção sexual são precos raramente anunciados.
    \item A disfunção sexual por ISRSs afeta 50-70\% dos pacientes --- é regra, não exceção.
    \item Toda prescrição implica uma hierarquia de valores que reflete mais o prescritor que o paciente.
    \item Uma farmacologia centrada no sujeito deve explicitarcustos, respeitar valores e reconhecer a urgência.
\end{enumerate}

\vspace{0.5cm}
\textbf{Conceitos-chave:} Latência farmacológica $\bullet$ Tempo vivido $\bullet$ Emotional blunting $\bullet$ PSSD $\bullet$ Hierarquia de valores $\bullet$ Decisão compartilhada
\end{sintese}
