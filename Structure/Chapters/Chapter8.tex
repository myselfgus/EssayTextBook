% !TEX root = ../../BookTemplate.tex
%%%%%%%%%%%%%%%%%%%%%%%%%%%%%%%%%%%%%%%%%%%%%%%%%%%%%%%%%%%%%%%%%%%%%%%%%%%%%%%%%%
\chapter{O Tempo e o Preco}
\label{cap:tempo-preco}

\begin{center}
\textit{``Este medicamento leva algumas semanas para fazer efeito. Tenha paciencia.''}\\[0.3cm]
--- Frase rotineira que ninguem mais questiona
\end{center}

\vspace{1cm}

%%%%%%%%%%%%%%%%%%%%%%%%%%%%%%%%%%%%%%%%%%%%%%%%%%%%%%%%%%%%%%%%%%%%%%%%%%%%%%%%%%
\section{A Fila Invisivel}
\label{sec:fila}
\index{Temporalidade}
\index{Latencia farmacologica}

Uma pessoa que talvez tenha levado meses para superar o estigma, semanas para conseguir consulta, anos para admitir que precisava de ajuda --- finalmente senta-se diante do psiquiatra. E entao recebe uma receita acompanhada de instrucao para ter \textit{paciencia}.

\begin{questao}
A pessoa que chegou em crise, que nao consegue trabalhar, que nao consegue dormir, que talvez nao consiga ver sentido em continuar vivendo --- essa pessoa e instruida a aguardar.

A entrar em uma fila invisivel onde seu sofrimento sera processado no \textbf{tempo da molecula}, nao no \textbf{tempo da urgencia}.
\end{questao}

Este capitulo examina duas dimensoes frequentemente negligenciadas:
\begin{itemize}
    \item A \textbf{temporalidade} do tratamento
    \item O \textbf{preco} que se paga por ele
\end{itemize}

%%%%%%%%%%%%%%%%%%%%%%%%%%%%%%%%%%%%%%%%%%%%%%%%%%%%%%%%%%%%%%%%%%%%%%%%%%%%%%%%%%
\section{O Tempo do Tratamento e o Tempo do Sofrimento}
\label{sec:tempo-sofrimento}

\subsection{A Latencia Como Dogma}

A afirmacao de que antidepressivos levam duas a quatro semanas para ``comecar a fazer efeito'' adquiriu estatuto de verdade inquestionavel.

\begin{observacao}
\autor{Machado-Vieira} et al. (2023): ``Antidepressivos convencionais que atuam em receptores monoaminergicos requerem varias semanas para serem eficazes. Este atraso representa um \textbf{problema significativo} nos tratamentos atualmente disponiveis para depressao grave.''

\textbf{O que significa, concretamente?}
\begin{itemize}
    \item Pessoa com risco de suicidio recebe tratamento sem alivio imediato
    \item Capacidade funcional permanece comprometida por semanas
    \item Intervalo entre buscar ajuda e recebe-la pode estender-se por meses
\end{itemize}
\end{observacao}

\subsection{A Prova de Que Poderia Ser Diferente}

\begin{casoclinico}[title={A Revolucao da Ketamina}]
Em 2000, \autor{Berman} e colaboradores publicaram estudo que deveria ter revolucionado a psiquiatria:

Uma dose subanestesica de ketamina produziu melhora significativa dos sintomas depressivos em \textbf{horas}, nao semanas.

\textbf{Implicacao:} Se resposta rapida e possivel, por que tratamentos de acao lenta permanecem como primeira linha?
\end{casoclinico}

\begin{figure}[H]
\centering
\begin{tikzpicture}[scale=0.9]
    % Timeline
    \draw[thick, -Stealth] (0,0) -- (12,0) node[right] {Tempo};

    % Ketamine
    \draw[thick, OliveGreen] (0.5,0.2) -- (0.5,2);
    \node[above, font=\small, OliveGreen] at (0.5,2) {Ketamina};
    \node[below, font=\footnotesize] at (0.5,-0.3) {24h};

    % SSRIs
    \draw[thick, BrickRed] (4,0.2) -- (4,2);
    \draw[thick, BrickRed, dashed] (4,2) -- (8,2);
    \node[above, font=\small, BrickRed] at (6,2) {ISRSs};
    \node[below, font=\footnotesize] at (4,-0.3) {2-4 sem};
    \node[below, font=\footnotesize] at (8,-0.3) {6-8 sem};
\end{tikzpicture}
\caption{Comparacao de latencia: ketamina vs. antidepressivos convencionais}
\label{fig:latencia}
\end{figure}

\subsection{O Paciente Como Request em Fila}

\autor{Minkowski} (1933), em seus estudos sobre tempo vivido, demonstrou que estados depressivos alteram profundamente a experiencia temporal: o futuro se fecha, o presente se arrasta, cada minuto pesa como hora.

\begin{observacao}
Dizer a alguem nesse estado que ``espere algumas semanas'' e impor uma temporalidade objetiva --- o tempo do calendario, o tempo da farmacologia --- sobre uma temporalidade subjetiva radicalmente distinta.

E, em certo sentido, uma \textbf{violencia epistemologica}: a afirmacao de que o tempo do tratamento e mais real que o tempo do sofrimento.
\end{observacao}

%%%%%%%%%%%%%%%%%%%%%%%%%%%%%%%%%%%%%%%%%%%%%%%%%%%%%%%%%%%%%%%%%%%%%%%%%%%%%%%%%%
\section{O Preco Nao Anunciado}
\label{sec:preco}
\index{Embotamento emocional}
\index{Disfuncao sexual}

\subsection{Emotional Blunting: Efeito Colateral ou Mecanismo de Acao?}

Pacientes em uso de ISRSs frequentemente relatam uma especie de anestesia afetiva: reducao nao apenas da tristeza patologica, mas da capacidade de experimentar todo o espectro emocional.

\begin{table}[H]
\centering
\caption{Emotional blunting: o que se perde}
\label{tab:blunting}
\begin{tabular}{cc}
\toprule
\textbf{O Que Se Ganha} & \textbf{O Que Se Perde} \\
\midrule
Reducao da tristeza patologica & Alegria \\
Reducao da ansiedade & Excitacao \\
Estabilidade emocional & Ternura \\
 & Raiva apropriada \\
\bottomrule
\end{tabular}
\end{table}

\begin{questao}
O \textit{emotional blunting} e \textbf{efeito colateral} dos ISRSs ou \textbf{mecanismo de acao}?

Se a reducao da ansiedade vem acompanhada de reducao da capacidade de sentir em geral, estamos \textit{tratando} a ansiedade ou \textit{suprimindo} a afetividade?
\end{questao}

\subsection{Disfuncao Sexual Como Regra, Nao Excecao}

\begin{conceitochave}[title={Estatisticas da Disfuncao Sexual por ISRSs}]
Estudos estimam que \textbf{50\% a 70\%} dos pacientes em ISRSs experimentam alguma forma de comprometimento sexual:
\begin{itemize}
    \item Reducao de libido
    \item Dificuldade de excitacao
    \item Anorgasmia
    \item Ejaculacao retardada
\end{itemize}

Nao estamos falando de efeito raro. A disfuncao sexual e, para ISRSs, \textbf{mais regra que excecao}.
\end{conceitochave}

\begin{observacao}
\textbf{Post-SSRI Sexual Dysfunction (PSSD):} Disfuncao sexual que persiste apos a descontinuacao do medicamento, potencialmente de forma \textbf{permanente}.

A Agencia Europeia de Medicamentos reconheceu o PSSD em 2019, exigindo atualizacao das bulas. A FDA americana, ate dezembro de 2025, ainda nao o fez.
\end{observacao}

\subsection{O Que Nao E Dito}

\begin{table}[H]
\centering
\caption{O que e dito vs. o que nao e dito ao prescrever ISRSs}
\label{tab:dito}
\begin{tabular}{p{5.5cm}p{5.5cm}}
\toprule
\textbf{O Que E Dito} & \textbf{O Que Nao E Dito} \\
\midrule
Ajudara com ansiedade/depressao & Probabilidade de 50-70\% de disfuncao sexual \\
Leva algumas semanas & Disfuncao pode persistir apos parar \\
Efeitos colaterais transitórios & Embotamento emocional, sentir-se ``anestesiado'' \\
Nao causa dependencia & Existem alternativas com perfil diferente \\
\bottomrule
\end{tabular}
\end{table}

%%%%%%%%%%%%%%%%%%%%%%%%%%%%%%%%%%%%%%%%%%%%%%%%%%%%%%%%%%%%%%%%%%%%%%%%%%%%%%%%%%
\section{Quem Decide O Que Vale O Que?}
\label{sec:hierarquia-valores}
\index{Autonomia}
\index{Valores do paciente}

\subsection{A Hierarquia Implicita de Valores}

\begin{observacao}
Toda prescricao psiquiatrica implica uma hierarquia de valores:
\begin{itemize}
    \item Ao prescrever ISRS sabendo que provavelmente causara disfuncao sexual: ``nao ter ansiedade'' vale mais que ``ter vida sexual satisfatoria''
    \item Ao prescrever antipsicotico sabendo que causara ganho de peso: ``nao ter sintomas psicoticos'' vale mais que ``manter peso saudavel''
\end{itemize}

\textbf{Essas hierarquias refletem valores de quem prescreve mais que de quem vai viver com as consequencias.}
\end{observacao}

\subsection{O Paciente Como Objeto, Nao Sujeito}

O paciente --- que veio ao consultorio como sujeito de seu sofrimento, buscando tornar-se sujeito de sua cura --- e transformado em \textit{objeto} da prescricao.

\begin{itemize}
    \item ``Adesao ao tratamento'' --- o tratamento e dado, ao paciente cabe aderir
    \item ``Resposta a medicacao'' --- como se fosse organismo que responde a estimulo
    \item ``Efeitos colaterais toleraveis'' --- sem perguntar a quem cabe definir toleravel
\end{itemize}

%%%%%%%%%%%%%%%%%%%%%%%%%%%%%%%%%%%%%%%%%%%%%%%%%%%%%%%%%%%%%%%%%%%%%%%%%%%%%%%%%%
\section{Uma Farmacologia Centrada no Sujeito}
\label{sec:farmacologia-sujeito}

\begin{table}[H]
\centering
\caption{Principios de uma farmacologia centrada no sujeito}
\label{tab:principios}
\begin{tabular}{p{4cm}p{7.5cm}}
\toprule
\textbf{Principio} & \textbf{Implementacao} \\
\midrule
O tempo do paciente & Reconhecer urgencia como urgencia; estrategias de suporte intensivo nas primeiras semanas \\
\addlinespace
Os valores do paciente & Perguntar prioridades antes de prescrever; nao ha resposta universal correta \\
\addlinespace
O preco explicitado & Informar sobre disfuncao sexual, embotamento; apresentar alternativas \\
\addlinespace
Decisao genuinamente compartilhada & Medico e paciente deliberam juntos sobre o que fazer \\
\bottomrule
\end{tabular}
\end{table}

%%%%%%%%%%%%%%%%%%%%%%%%%%%%%%%%%%%%%%%%%%%%%%%%%%%%%%%%%%%%%%%%%%%%%%%%%%%%%%%%%%
\begin{sintese}
\textbf{Ideias centrais deste capitulo:}
\begin{enumerate}
    \item O tempo do tratamento e o tempo da molecula; o tempo do sofrimento e outro.
    \item A existencia de antidepressivos de acao rapida (ketamina) prova que a latencia nao e inevitavel.
    \item Emotional blunting e disfuncao sexual sao precos raramente anunciados.
    \item A disfuncao sexual por ISRSs afeta 50-70\% dos pacientes --- regra, nao excecao.
    \item Hierarquias de valores implicitas na prescricao refletem perspectiva do prescritor.
    \item Uma farmacologia centrada no sujeito explicita precos e respeita valores do paciente.
\end{enumerate}

\vspace{0.5cm}
\textbf{Conceitos-chave:} Latencia farmacologica $\bullet$ Tempo vivido $\bullet$ Emotional blunting $\bullet$ PSSD $\bullet$ Hierarquia de valores $\bullet$ Decisao compartilhada
\end{sintese}
