% !TEX root = ../../BookTemplate.tex
%%%%%%%%%%%%%%%%%%%%%%%%%%%%%%%%%%%%%%%%%%%%%%%%%%%%%%%%%%%%%%%%%%%%%%%%%%%%%%%%%%
\chapter{Disruptura Cognitiva}
\label{cap:disruptura}

\begin{center}
\textit{``A disruptura nao e apenas desarranjo a ser corrigido; e, potencialmente, reorganizacao em curso.''}
\end{center}

\vspace{1cm}

%%%%%%%%%%%%%%%%%%%%%%%%%%%%%%%%%%%%%%%%%%%%%%%%%%%%%%%%%%%%%%%%%%%%%%%%%%%%%%%%%%
\section{Do Territorio Liminar ao Conceito}
\label{sec:territorio}
\index{Disruptura cognitiva}

O capitulo anterior mapeou o territorio entre sanidade e loucura. Exploramos o continuum dimensional, a historicidade das categorias, a semiologia fenomenologica, e o fenomeno da psicotizacao contextual. Terminamos observando que existem experiencias que nao sao claramente psicoticas nem claramente ``normais'' --- estados liminares que resistem a classificacao binaria.

Este capitulo propoe um conceito para nomear estes estados: \textbf{disruptura cognitiva}.

\begin{definicaobox}[title={Disruptura Cognitiva --- Etimologia}]
O termo ``disruptura'' --- do latim \textit{disrumpere}, romper, quebrar --- carrega intencionalmente dupla valencia:
\begin{itemize}
    \item \textbf{Quebra:} Descontinuidade, perturbacao da ordem previa
    \item \textbf{Erupcao:} Emergencia de algo novo (proximidade fonetica com ``erupcao'')
\end{itemize}
\end{definicaobox}

%%%%%%%%%%%%%%%%%%%%%%%%%%%%%%%%%%%%%%%%%%%%%%%%%%%%%%%%%%%%%%%%%%%%%%%%%%%%%%%%%%
\section{Genealogia do Conceito: Tradicoes Convergentes}
\label{sec:genealogia}

\begin{figure}[H]
\centering
\begin{tikzpicture}[
    node distance=1.2cm,
    tradition/.style={rectangle, draw=sectioncolor, fill=boxblue, text width=5cm, text centered, rounded corners, minimum height=1cm, font=\small},
    center/.style={ellipse, draw=chaptercolor, fill=boxorange, text width=3cm, text centered, minimum height=2cm, font=\small\bfseries}
]
    \node[center] (dc) {DISRUPTURA\\COGNITIVA};
    \node[tradition, above left=1.5cm and 0.5cm of dc] (cont) {Continuum Psicotico\\(van Os, Linscott)};
    \node[tradition, above right=1.5cm and 0.5cm of dc] (atpd) {Psicose Breve\\(ATPD, bouffee delirante)};
    \node[tradition, left=1.5cm of dc] (luto) {Alucinacoes do Luto\\(Rees, Penberthy)};
    \node[tradition, right=1.5cm of dc] (esp) {Emergencia Espiritual\\(Grof)};
    \node[tradition, below left=1.5cm and 0.5cm of dc] (ptg) {Crescimento Pos-Traumatico\\(Tedeschi, Calhoun)};
    \node[tradition, below right=1.5cm and 0.5cm of dc] (spat) {Psicopatologia Espaciotemporal\\(Northoff)};

    \draw[-Stealth] (cont) -- (dc);
    \draw[-Stealth] (atpd) -- (dc);
    \draw[-Stealth] (luto) -- (dc);
    \draw[-Stealth] (esp) -- (dc);
    \draw[-Stealth] (ptg) -- (dc);
    \draw[-Stealth] (spat) -- (dc);
\end{tikzpicture}
\caption{Tradicoes convergentes no conceito de disruptura cognitiva}
\label{fig:tradicoes}
\end{figure}

\subsection{Alucinacoes do Luto: O Patologico que Nao E}
\index{Alucinacoes do luto}

Desde o estudo seminal de \autor{Rees} (1971), a literatura documenta que 40-60\% dos enlutados relatam experiencias sensoriais do falecido. \autor{Penberthy} et al. (2023) documentou que 73,4\% relataram que estas experiencias trouxeram \textit{conforto e cura emocional}.

\begin{conceitochave}[title={O Paradoxo das Alucinacoes Adaptativas}]
Estas experiencias sao fenomenologicamente indistinguiveis de alucinacoes --- satisfazem todos os criterios formais de percepcao sem objeto. Entretanto:
\begin{itemize}
    \item Nao sao patologicas
    \item Nao indicam deterioracao
    \item Nao requerem tratamento
    \item Frequentemente sao \textbf{adaptativas}
\end{itemize}

\textbf{Pergunta:} Em que outros contextos experiencias formalmente ``psicoticas'' podem ser nao-patologicas ou beneficas?
\end{conceitochave}

\subsection{Emergencia Espiritual}
\index{Emergencia espiritual}

\autor{Stanislav e Christina Grof} (1989) propuseram que certas crises psicologicas intensas podem representar ``emergencias espirituais'': processos de transformacao psiquica que, se adequadamente apoiados, conduzem a maior integracao.

O trocadilho e intencional: \textit{emergency} (emergencia/crise) contem \textit{emergence} (emergencia/surgimento).

\subsection{Crescimento Pos-Traumatico}
\index{Crescimento pos-traumatico}

\autor{Tedeschi e Calhoun} (1995, 2004) descrevem o PTG como resultado de ``eventos sismicos'' que abalam estruturas de significado pre-existentes, forcando reorganizacao cognitiva e existencial:

\begin{itemize}
    \item Percepcao aumentada de forca pessoal
    \item Abertura a novas possibilidades
    \item Aprofundamento de relacoes
    \item Maior apreciacao da vida
    \item Desenvolvimento espiritual
\end{itemize}

%%%%%%%%%%%%%%%%%%%%%%%%%%%%%%%%%%%%%%%%%%%%%%%%%%%%%%%%%%%%%%%%%%%%%%%%%%%%%%%%%%
\section{Definicao Proposta}
\label{sec:definicao-disruptura}

\begin{conceitochave}[title={Definicao Operacional de Disruptura Cognitiva}]
\textbf{Ruptura transitoria na organizacao dos processos cognitivos, afetivos e perceptivos}, caracterizada por descontinuidade na experiencia de si e do mundo, que pode --- dependendo de fatores contextuais, relacionais e de suporte --- evoluir para:
\begin{enumerate}
    \item Reorganizacao adaptativa
    \item Cristalizacao em patologia
    \item Resolucao sem sequelas significativas
\end{enumerate}
\end{conceitochave}

\subsection{Elementos da Definicao}

\begin{table}[H]
\centering
\caption{Elementos definitorios da disruptura cognitiva}
\label{tab:elementos}
\begin{tabular}{p{4cm}p{7.5cm}}
\toprule
\textbf{Elemento} & \textbf{Significado} \\
\midrule
Ruptura transitoria & Estados delimitados no tempo (horas a meses, nao anos) \\
Organizacao dos processos & Afeta articulacao entre pensamento, emocao e percepcao \\
Descontinuidade & Experiencia de ``quebra'' --- algo mudou \\
Multiplos desfechos & A disruptura nao determina seu proprio desfecho \\
\bottomrule
\end{tabular}
\end{table}

\subsection{Caracteristicas Estruturais}

\begin{figure}[H]
\centering
\begin{tikzpicture}[
    node distance=0.8cm,
    char/.style={rectangle, draw=sectioncolor, fill=boxblue, text width=10cm, text centered, rounded corners, minimum height=0.8cm, font=\small}
]
    \node[char] (temp) {\textbf{Alteracao da temporalidade vivida:} O tempo acelera, desacelera, fragmenta-se};
    \node[char, below=of temp] (evid) {\textbf{Perturbacao das evidencias naturais:} Pressuposicoes tacitas tornam-se incertas};
    \node[char, below=of evid] (sal) {\textbf{Alteracao da saliencia:} Mundo hiper ou hipo-significativo};
    \node[char, below=of sal] (front) {\textbf{Porosidade das fronteiras:} Self/mundo, interno/externo menos definidos};
    \node[char, below=of front] (meta) {\textbf{Preservacao variavel da metacognicao:} Frequentemente reconhece alteracao};
\end{tikzpicture}
\caption{Caracteristicas estruturais das disrupturas cognitivas}
\label{fig:caracteristicas}
\end{figure}

%%%%%%%%%%%%%%%%%%%%%%%%%%%%%%%%%%%%%%%%%%%%%%%%%%%%%%%%%%%%%%%%%%%%%%%%%%%%%%%%%%
\section{Fenomenologia Clinica: Vinhetas Ilustrativas}
\label{sec:vinhetas}

\subsection{Marina: O Luto que Ve}

\begin{casoclinico}[title={Caso 1: Experiencias perceptivas no luto}]
Marina, 58 anos, viuva ha tres meses. Relata que tem ``visto'' e ``ouvido'' o marido falecido.

\textbf{Semiologia:}
\begin{itemize}
    \item Experiencias perceptivas do falecido (visuais, auditivas, tateis)
    \item Preservacao do teste de realidade (``sei que sao impossiveis'')
    \item Metacognicao intacta
    \item Valencia afetiva positiva (conforto, nao terror)
    \item Ausencia de outras perturbacoes
\end{itemize}

\textbf{Intervencao:} Normalizacao, exploracao do significado, acompanhamento do luto.

\textbf{Desfecho:} Integracao adaptativa da experiencia.
\end{casoclinico}

\subsection{Pedro: A Crise que Quase Virou Diagnostico}

\begin{casoclinico}[title={Caso 2: Estado expansivo apos imersao intelectual}]
Pedro, 22 anos, estudante de filosofia. Apos leituras intensas sobre fisica quantica e misticismo, teve ``insight'' de que ``tudo esta conectado''. Estado acelerado, reducao do sono, pressao para escrever.

\textbf{Diferencial:} Inicio de mania vs. emergencia espiritual vs. privacao de sono.

\textbf{Conduta:} Garantir sono, suspender estimulantes, afastar de material excitante, reavaliar em 48h.

\textbf{Desfecho:} Tres meses depois, sem medicacao de manutencao, Pedro integrou a experiencia como ``crise de crescimento''.
\end{casoclinico}

\subsection{Claudia: A Psicose que Era e Nao Era}

\begin{casoclinico}[title={Caso 3: Psicose breve reativa}]
Claudia, 34 anos, advogada. Internacao por crencas persecutorias, ideias de referencia.

\textbf{Contexto:} Preterida em promocao, descobriu infidelidade do marido, carga de trabalho insustentavel.

\textbf{Diagnostico:} F23 (Transtorno Psicotico Breve).

\textbf{Desfecho:} Cinco anos depois, sem outros episodios. Divorciou-se, mudou de emprego, usa o episodio como ``marco''.
\end{casoclinico}

\subsection{Rafael: A Disruptura que Cronificou}

\begin{casoclinico}[title={Caso 4: Evolucao para esquizofrenia}]
Rafael, 19 anos. Primeiro episodio aos 17: estranheza, pensamentos ``colocados'', vozes comentando acoes.

\textbf{Fatores de risco:} Historia familiar de esquizofrenia, inicio na adolescencia, perturbacao profunda da ipseidade.

\textbf{Desfecho:} Multiplas recaidas, sintomas residuais, funcionamento abaixo do esperado.

\textbf{Licao:} Disruptura cognitiva nao e diagnostico benigno por definicao. Algumas cronificam-se.
\end{casoclinico}

%%%%%%%%%%%%%%%%%%%%%%%%%%%%%%%%%%%%%%%%%%%%%%%%%%%%%%%%%%%%%%%%%%%%%%%%%%%%%%%%%%
\section{Implicacoes Clinicas: Avaliacao e Manejo}
\label{sec:manejo-disruptura}

\subsection{O Que Observar na Avaliacao}

\begin{enumerate}
    \item \textbf{Estrutura da experiencia:} Ha preservacao de metacognicao?
    \item \textbf{Contexto de emergencia:} Ha gatilho identificavel?
    \item \textbf{Historia longitudinal:} Funcionamento previo, episodios anteriores, historia familiar
    \item \textbf{Natureza das perturbacoes:} Predominantemente afetivas ou da ipseidade?
    \item \textbf{Resposta a intervencao inicial:} Melhora com medidas simples?
\end{enumerate}

\subsection{Principios de Manejo}

\begin{table}[H]
\centering
\caption{Principios de manejo de disrupturas cognitivas}
\label{tab:manejo}
\begin{tabular}{p{4cm}p{7.5cm}}
\toprule
\textbf{Principio} & \textbf{Aplicacao} \\
\midrule
Nao patologizar precipitadamente & Usar formulacoes provisorias antes de diagnosticos categoricos \\
Garantir necessidades basicas & Sono, alimentacao, seguranca \\
Criar continencia relacional & Escuta sem julgamento, tolerancia da perturbacao \\
Usar medicacao criteriosamente & Pode ser necessaria, mas pode impedir reorganizacao espontanea \\
Acompanhar longitudinalmente & Diagnostico diferencial frequentemente so e possivel retrospectivamente \\
\bottomrule
\end{tabular}
\end{table}

\subsection{Quando Preocupar-se Mais?}

\begin{observacao}
\textbf{Elementos que sugerem maior probabilidade de desfecho desfavoravel:}
\begin{itemize}
    \item Perturbacao profunda da ipseidade
    \item Ausencia de gatilho identificavel
    \item Historia familiar de esquizofrenia
    \item Inicio na adolescencia
    \item Funcionamento previo marginal
    \item Ausencia de resposta a medidas conservadoras
    \item Duracao prolongada sem melhora
    \item Deterioracao progressiva
\end{itemize}
\end{observacao}

%%%%%%%%%%%%%%%%%%%%%%%%%%%%%%%%%%%%%%%%%%%%%%%%%%%%%%%%%%%%%%%%%%%%%%%%%%%%%%%%%%
\section{Implicacoes Eticas}
\label{sec:etica-disruptura}
\index{Patologizacao}
\index{Romantizacao}

\subsection{Entre Patologizacao e Romantizacao}

\begin{figure}[H]
\centering
\begin{tikzpicture}[
    node distance=3cm,
    error/.style={rectangle, draw=BrickRed, fill=boxred, text width=4cm, text centered, rounded corners, minimum height=2cm, font=\small},
    balance/.style={rectangle, draw=OliveGreen, fill=boxgreen, text width=4cm, text centered, rounded corners, minimum height=2cm, font=\small}
]
    \node[error] (pat) {\textbf{Patologizacao}\\Toda perturbacao e doenca a ser suprimida};
    \node[balance] (bal) at (5.5,0) {\textbf{Disruptura Cognitiva}\\Ruptura com multiplos desfechos possiveis};
    \node[error] (rom) at (11,0) {\textbf{Romantizacao}\\Toda perturbacao e ``emergencia espiritual''};

    \draw[-Stealth, thick, dashed] (pat) -- (bal);
    \draw[-Stealth, thick, dashed] (rom) -- (bal);
\end{tikzpicture}
\caption{Posicionamento do conceito entre dois erros opostos}
\label{fig:erros}
\end{figure}

%%%%%%%%%%%%%%%%%%%%%%%%%%%%%%%%%%%%%%%%%%%%%%%%%%%%%%%%%%%%%%%%%%%%%%%%%%%%%%%%%%
\begin{sintese}
\textbf{Ideias centrais deste capitulo:}
\begin{enumerate}
    \item Disruptura cognitiva e ruptura transitoria na organizacao da experiencia com multiplos desfechos possiveis.
    \item O conceito sintetiza tradicoes convergentes: continuum psicotico, psicoses breves, alucinacoes do luto, emergencia espiritual, crescimento pos-traumatico.
    \item Caracteristicas estruturais incluem alteracao da temporalidade, perturbacao das evidencias naturais, alteracao da saliencia.
    \item Casos clinicos ilustram diversidade de desfechos: integracao adaptativa, resolucao espontanea, cronificacao.
    \item Manejo deve evitar tanto patologizacao precipitada quanto romantizacao ingenua.
    \item Preservacao de metacognicao e fator prognostico importante.
\end{enumerate}

\vspace{0.5cm}
\textbf{Conceitos-chave:} Disruptura cognitiva $\bullet$ Emergencia espiritual $\bullet$ Crescimento pos-traumatico $\bullet$ Saliencia aberrante $\bullet$ Metacognicao $\bullet$ Continencia relacional
\end{sintese}
