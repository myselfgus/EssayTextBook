% !TEX root = ../../BookTemplate.tex
%%%%%%%%%%%%%%%%%%%%%%%%%%%%%%%%%%%%%%%%%%%%%%%%%%%%%%%%%%%%%%%%%%%%%%%%%%%%%%%%%%
\chapter{O rigor do diagnostico clinico}
\label{cap:rigor}

\begin{center}
\textit{``A psicopatologia e, nao se deve esquecer, uma linguagem, um idioma, e uma ciencia bem feita necessita de um idioma bem construido, claro, compreensivel e honesto para a comunicacao dos fatos clinicos.''}\\[0.3cm]
--- Paulo Dalgalarrondo, \textit{Psicopatologia e Semiologia dos Transtornos Mentais} (2019)
\end{center}

\vspace{1cm}

%%%%%%%%%%%%%%%%%%%%%%%%%%%%%%%%%%%%%%%%%%%%%%%%%%%%%%%%%%%%%%%%%%%%%%%%%%%%%%%%%%
\section{O que voce sabe quando diz que sabe?}
\label{sec:saber}
\index{Diagnostico!fundamentos}

Comecemos com uma pergunta simples. Quando alguem afirma que um paciente ``tem depressao'', o que exatamente essa pessoa esta dizendo? Que observou um conjunto de sintomas? Que aplicou criterios de um manual? Que reconheceu um padrao clinico? Que compreendeu algo sobre o sofrimento daquela pessoa especifica?

Sao perguntas elementares. Deveriam ter respostas elementares. E no entanto, se as fizessemos a dez profissionais diferentes que rotineiramente emitem diagnosticos psiquiatricos, receberiamos dez respostas diferentes --- ou, mais provavelmente, receberiamos hesitacoes, circunloquios, e a sensacao desconfortavel de que a pergunta e, de algum modo, impertinente.

\begin{questao}
Mas por que seria impertinente perguntar a alguem que diagnostica o que significa diagnosticar?

\textbf{Antes de continuar a leitura, reflita: como voce definiria o ato de diagnosticar em psiquiatria?}
\end{questao}

%%%%%%%%%%%%%%%%%%%%%%%%%%%%%%%%%%%%%%%%%%%%%%%%%%%%%%%%%%%%%%%%%%%%%%%%%%%%%%%%%%
\section{A semiologia como disciplina}
\label{sec:semiologia}
\index{Semiologia}
\index{Propedeutica}

Existe uma disciplina medica dedicada precisamente a essa questao. Chama-se \termotecnico{semiologia} --- ou, em sua dimensao pratica, \termotecnico{propedeutica}. \autor{Celmo Celeno Porto}, no tratado que ha decadas forma geracoes de medicos brasileiros, define a semiologia como ``o estudo dos sinais e sintomas das doencas humanas'' e a propedeutica como ``a arte de examinar o paciente e avaliar os sinais e sintomas, capacitando o medico a elaborar o diagnostico e o prognostico de um paciente'' \cite{Porto2019}.

\begin{definicaobox}[title={Componentes da Semiologia Psiquiatrica}]
Segundo \autor{Paulo Dalgalarrondo}:
\begin{description}
    \item[Semiotecnica:] Tecnicas de coleta e observacao de sinais e sintomas.
    \item[Semiogenese:] Investigacao da origem e do significado clinico dos achados.
\end{description}
Nao se trata de opiniao, preferencia teorica ou escola de pensamento. Trata-se de tecnica --- tecnica que se aprende, se pratica, se aperfecoa, e que pode ser bem ou mal executada.
\end{definicaobox}

\begin{figure}[H]
\centering
\begin{tikzpicture}[
    node distance=1.5cm,
    box/.style={rectangle, draw=chaptercolor, fill=boxblue, text width=4cm, text centered, rounded corners, minimum height=1.2cm, font=\small},
    arrow/.style={-Stealth, thick, color=sectioncolor}
]
    \node[box] (semiologia) {\textbf{Semiologia}\\Estudo de sinais e sintomas};
    \node[box, below left=of semiologia] (semiotecnica) {\textbf{Semiotecnica}\\Tecnicas de coleta};
    \node[box, below right=of semiologia] (semiogenese) {\textbf{Semiogenese}\\Origem e significado};
    \node[box, below=3cm of semiologia] (diagnostico) {\textbf{Diagnostico}\\Sintese clinica};

    \draw[arrow] (semiologia) -- (semiotecnica);
    \draw[arrow] (semiologia) -- (semiogenese);
    \draw[arrow] (semiotecnica) -- (diagnostico);
    \draw[arrow] (semiogenese) -- (diagnostico);
\end{tikzpicture}
\caption{Estrutura da semiologia psiquiatrica}
\label{fig:semiologia}
\end{figure}

Linguagem. Idioma. Clareza. Honestidade. Comunicacao de fatos clinicos.

\textbf{Quantos dos que diagnosticam diariamente dominam esse idioma?}

%%%%%%%%%%%%%%%%%%%%%%%%%%%%%%%%%%%%%%%%%%%%%%%%%%%%%%%%%%%%%%%%%%%%%%%%%%%%%%%%%%
\section{A estranha excecao}
\label{sec:excecao}
\index{Diagnostico!democratizacao}

Ha algo peculiar na psiquiatria contemporanea que merece reflexao. Em nenhuma outra especialidade medica se observa tamanha democratizacao do ato diagnostico. Ninguem sem formacao cardiologica se aventura a diagnosticar uma arritmia complexa. Ninguem sem treinamento neurologico pretende identificar uma sindrome extrapiramidal. E no entanto, diagnosticos psiquiatricos sao emitidos rotineiramente por profissionais sem formacao especifica em psicopatologia, por nao-medicos que jamais estudaram semiologia, por coaches e influenciadores que confundem listas de sintomas com conhecimento clinico.

\begin{observacao}
\textbf{O sofisma da familiaridade}

``Todos sabemos o que e tristeza, ansiedade, desatencao.''

Mas todos sabemos o que e dor no peito, e nem por isso nos consideramos aptos a distinguir uma angina de uma pericardite. Todos sabemos o que e tontura, e nem por isso presumimos diferenciar uma vertigem periferica de uma central.

\textbf{O que nos faz pensar que reconhecer tristeza equivale a diagnosticar um episodio depressivo maior?}
\end{observacao}

A confusao, suspeito, esta na propria estrutura conceitual que orienta o diagnostico psiquiatrico contemporaneo.

%%%%%%%%%%%%%%%%%%%%%%%%%%%%%%%%%%%%%%%%%%%%%%%%%%%%%%%%%%%%%%%%%%%%%%%%%%%%%%%%%%
\section{Criterios operacionais e suas promessas}
\label{sec:criterios}
\index{DSM!criterios operacionais}
\index{Confiabilidade diagnostica}

A partir de 1980, com a publicacao do DSM-III, a psiquiatria adotou uma abordagem que prometia resolver seus problemas historicos de confiabilidade. Os \termotecnico{criterios operacionais} --- listas explicitas de sintomas necessarios e suficientes para cada diagnostico --- deveriam garantir que diferentes clinicos, diante do mesmo paciente, chegassem as mesmas conclusoes.

A promessa era elegante. A realidade provou-se mais complexa.

\begin{casoclinico}[title={Estudo UFRJ: Quando os criterios sao aplicados}]
\autor{Rocha Neto} e colaboradores (2024) pediram a residentes de psiquiatria que revisassem diagnosticos de pacientes que ja acompanhavam, desta vez aplicando formalmente os criterios da CID-10.

\textbf{Resultado:}
\begin{itemize}
    \item 30\% dos diagnosticos mudaram quando os criterios foram efetivamente aplicados
    \item 20\% dessas mudancas afetariam decisoes de tratamento
\end{itemize}

Note-se: nao eram estudantes iniciantes, mas residentes em treinamento avancado. E nao se tratava de pacientes novos, mas de pessoas que ja conheciam.
\end{casoclinico}

\begin{table}[H]
\centering
\caption{Abordagens diagnosticas em psiquiatria}
\label{tab:abordagens}
\begin{tabular}{p{5.5cm}p{5.5cm}}
\toprule
\textbf{Abordagem Prototipica} & \textbf{Abordagem Criteriologica} \\
\midrule
Reconhecimento de padroes & Aplicacao sistematica de criterios \\
Intuicao clinica & Checklist explicito \\
Impressao geral & Contagem de sintomas \\
Rapida, mas variavel & Lenta, mas padronizada \\
Dominante na pratica & Ideal teorico \\
\bottomrule
\end{tabular}
\end{table}

O que esse achado nos diz? Que os criterios operacionais, na pratica clinica cotidiana, frequentemente nao sao aplicados. Que diagnosticos sao feitos por reconhecimento de padroes, intuicao, impressao geral --- o que os pesquisadores chamam de abordagem ``prototipica'', em contraste com a abordagem criteriologica.

%%%%%%%%%%%%%%%%%%%%%%%%%%%%%%%%%%%%%%%%%%%%%%%%%%%%%%%%%%%%%%%%%%%%%%%%%%%%%%%%%%
\section{A pergunta que persiste}
\label{sec:pergunta-persiste}
\index{Validade diagnostica}

Mas entao, o que significa diagnosticar corretamente? Se a aplicacao formal dos criterios produz resultados diferentes do julgamento clinico habitual, qual dos dois esta certo?

A propria pergunta revela o problema. Estamos tratando o diagnostico como se fosse uma operacao binaria --- certo ou errado, presente ou ausente. E no entanto, a realidade clinica raramente se apresenta assim.

\begin{conceitochave}[title={Heterogeneidade de Medicao}]
Um artigo publicado em 2025 na revista \textit{History and Philosophy of the Life Sciences} argumenta que os proprios sintomas psiquiatricos --- nao apenas os diagnosticos --- sao construtos heterogeneos, medidos de formas diferentes por instrumentos diferentes, produzindo dados que nao necessariamente se referem aos mesmos fenomenos.

\textbf{Se nao sabemos ao certo o que estamos medindo, como podemos saber se medimos corretamente?}
\end{conceitochave}

%%%%%%%%%%%%%%%%%%%%%%%%%%%%%%%%%%%%%%%%%%%%%%%%%%%%%%%%%%%%%%%%%%%%%%%%%%%%%%%%%%
\section{O reconhecimento honesto da APA}
\label{sec:apa}
\index{APA}
\index{Confiabilidade inter-avaliadores}

E instrutivo notar que a propria American Psychiatric Association, responsavel pelo DSM, reconhece limitacoes que seus criticos frequentemente ignoram. A \termotecnico{confiabilidade inter-avaliadores} --- a concordancia entre diferentes clinicos avaliando o mesmo paciente --- e admitidamente baixa para diagnosticos tao comuns quanto transtorno depressivo maior e transtorno de ansiedade generalizada.

\begin{figure}[H]
\centering
\begin{tikzpicture}
    \begin{scope}[scale=0.8]
    % Barra de confiabilidade
    \draw[thick] (0,0) rectangle (12,1);
    \fill[boxred] (0,0) rectangle (4,1);
    \fill[boxorange] (4,0) rectangle (7,1);
    \fill[boxgreen] (7,0) rectangle (12,1);

    % Labels
    \node at (2,0.5) {\small Baixa};
    \node at (5.5,0.5) {\small Moderada};
    \node at (9.5,0.5) {\small Alta};

    % Marcadores
    \draw[thick, chaptercolor] (3.5,-0.3) -- (3.5,1.3);
    \node[below, font=\footnotesize] at (3.5,-0.5) {TDM};

    \draw[thick, chaptercolor] (4.2,-0.3) -- (4.2,1.3);
    \node[below, font=\footnotesize] at (4.2,-0.5) {TAG};

    \draw[thick, chaptercolor] (8,-0.3) -- (8,1.3);
    \node[below, font=\footnotesize] at (8,-0.5) {Esquizofrenia};

    % Titulo
    \node[above, font=\small\bfseries] at (6,1.3) {Confiabilidade Inter-avaliadores (Kappa)};
    \end{scope}
\end{tikzpicture}
\caption{Confiabilidade diagnostica para transtornos selecionados (adaptado de dados do DSM-5)}
\label{fig:confiabilidade}
\end{figure}

Nao se trata de critica externa; sao dados publicados pela propria organizacao que produz o manual. E no entanto, esses diagnosticos sao emitidos diariamente, milhoes de vezes, em todo o mundo, frequentemente em consultas de quinze minutos, por profissionais com graus variados de formacao em psicopatologia.

\begin{questao}
Como conciliar o reconhecimento institucional da fragilidade diagnostica com a pratica cotidiana que a ignora?
\end{questao}

%%%%%%%%%%%%%%%%%%%%%%%%%%%%%%%%%%%%%%%%%%%%%%%%%%%%%%%%%%%%%%%%%%%%%%%%%%%%%%%%%%
\section{O caso do TDAH}
\label{sec:tdah}
\index{TDAH}
\index{Sobrediagnostico}

Consideremos um exemplo concreto: o \termotecnico{transtorno de deficit de atencao e hiperatividade}. Uma revisao sistematica publicada no \textit{JAMA Network Open} \cite{Kazda2021} examinou 334 estudos sobre sobrediagnostico de TDAH em criancas e adolescentes.

\begin{observacao}
\textbf{Criterios de Sobrediagnostico Confirmados}

Todas as cinco condicoes que definem sobrediagnostico foram confirmadas:
\begin{enumerate}
    \item Expansao dos criterios diagnosticos
    \item Aumento das taxas de diagnostico
    \item Concentracao em casos leves
    \item Tratamento de casos que nao se beneficiam
    \item Evidencias de danos
\end{enumerate}
\end{observacao}

\begin{table}[H]
\centering
\caption{Evolucao do diagnostico de TDAH}
\label{tab:tdah}
\begin{tabular}{lcc}
\toprule
\textbf{Indicador} & \textbf{Antes} & \textbf{Atual} \\
\midrule
Prevalencia em criancas (EUA) & 8\% & 11\% \\
Idade de inicio (criterio DSM) & 7 anos & 12 anos \\
Sintomas minimos (adultos) & 6 & 5 \\
Atendimentos Brasil (2022) & --- & 230.000 \\
\bottomrule
\end{tabular}
\end{table}

\autor{Luis Rohde}, psiquiatra da UFRGS e uma das principais autoridades brasileiras no tema, reconhece a questao: ``ao ampliar os criterios, obviamente aumentamos ligeiramente a prevalencia''. A expansao do DSM-5 --- que reduziu o numero minimo de sintomas para adultos e aumentou a idade de inicio de 7 para 12 anos --- nao foi acidental. Foi uma decisao deliberada, com consequencias previsiveis.

\begin{questao}
\textbf{Pergunta socratica:}

Se uma condicao pode ser diagnosticada de forma tao variavel, dependendo de quem avalia, onde avalia, e com que criterios avalia --- o que exatamente esta sendo diagnosticado?
\end{questao}

%%%%%%%%%%%%%%%%%%%%%%%%%%%%%%%%%%%%%%%%%%%%%%%%%%%%%%%%%%%%%%%%%%%%%%%%%%%%%%%%%%
\section{A distincao essencial: Explicar vs. Compreender}
\label{sec:explicar-compreender}
\index{Jaspers, Karl}
\index{Erklaren}
\index{Verstehen}

\autor{Karl Jaspers}, ha mais de um seculo, propos uma distincao que permanece atual:

\begin{figure}[H]
\centering
\begin{tikzpicture}[
    node distance=2cm,
    box/.style={rectangle, draw, text width=5cm, text centered, rounded corners, minimum height=3cm, font=\small}
]
    \node[box, draw=BrickRed, fill=boxred] (explicar) {
        \textbf{\large Erklaren}\\[0.3cm]
        \textit{Explicar}\\[0.3cm]
        Conexoes causais externas\\
        Neurotransmissores, lesoes,\\fatores de risco\\[0.2cm]
        \textit{Registro das ciencias naturais}
    };

    \node[box, draw=OliveGreen, fill=boxgreen, right=2cm of explicar] (compreender) {
        \textbf{\large Verstehen}\\[0.3cm]
        \textit{Compreender}\\[0.3cm]
        Conexoes de sentido internas\\
        Por que esta pessoa, com esta\\historia, sofre assim\\[0.2cm]
        \textit{Registro da fenomenologia}
    };

    \node[below=1cm of $(explicar)!0.5!(compreender)$, text width=8cm, text centered, font=\small\itshape] {
        Os criterios operacionais do DSM e da CID operam no registro da explicacao.\\
        O que nao fazem --- e nao pretendem fazer --- e compreender.
    };
\end{tikzpicture}
\caption{A distincao de Jaspers entre explicar e compreender}
\label{fig:explicar-compreender}
\end{figure}

\autor{Dalgalarrondo} nota que a psicopatologia contemporanea busca integrar a rica tradicao descritiva com os avancos das neurociencias cognitivas. Mas integracao pressupoe dominio de ambos os polos. Pressupoe conhecer tanto os mecanismos neurobiologicos quanto a fenomenologia da experiencia vivida. Pressupoe, em suma, formacao --- formacao que nao se adquire lendo listas de criterios, assistindo videos de divulgacao, ou memorizando mnemonicos para provas.

%%%%%%%%%%%%%%%%%%%%%%%%%%%%%%%%%%%%%%%%%%%%%%%%%%%%%%%%%%%%%%%%%%%%%%%%%%%%%%%%%%
\section{O diagnostico como responsabilidade}
\label{sec:responsabilidade}
\index{Etica diagnostica}

Ha uma dimensao etica do diagnostico que frequentemente se perde nas discussoes tecnicas. Diagnosticar nao e apenas classificar --- e nomear, e enquadrar, e, de certo modo, \textbf{definir uma pessoa}. Um diagnostico psiquiatrico acompanha o individuo em prontuarios, relatorios, documentos previdenciarios, processos judiciais. Pode abrir portas (acesso a tratamentos, beneficios, compreensao) ou fecha-las (estigma, discriminacao, limitacoes profissionais).

\begin{table}[H]
\centering
\caption{Consequencias documentadas do sobrediagnostico}
\label{tab:consequencias}
\begin{tabular}{p{5cm}p{6cm}}
\toprule
\textbf{Dominio} & \textbf{Consequencias} \\
\midrule
Psicossocial & Estigmatizacao, isolamento, vergonha \\
Clinico & Efeitos adversos sem beneficio \\
Cardiovascular & Aumento significativo de risco (uso prolongado de estimulantes) \\
Identitario & Internalizacao do rotulo \\
\bottomrule
\end{tabular}
\end{table}

\autor{Zhang} e colaboradores (2024), publicando no \textit{JAMA Psychiatry}, demonstraram que o uso prolongado de medicacoes para TDAH aumenta significativamente o risco cardiovascular --- informacao raramente comunicada aos pacientes antes do inicio do tratamento \cite{Zhang2024}.

\begin{observacao}
\textbf{O dilema do diagnostico}

Se diagnosticamos de menos, deixamos pessoas sofrendo sem tratamento.

Se diagnosticamos de mais, impomos rotulos e intervencoes a quem deles nao precisa.

\textbf{Em ambos os casos, falhamos.}
\end{observacao}

%%%%%%%%%%%%%%%%%%%%%%%%%%%%%%%%%%%%%%%%%%%%%%%%%%%%%%%%%%%%%%%%%%%%%%%%%%%%%%%%%%
\section{A formacao como condicao}
\label{sec:formacao}
\index{Formacao psiquiatrica}
\index{Entrevista psiquiatrica}

Retorno a pergunta inicial: \textbf{o que voce sabe quando diz que sabe?}

A semiologia psiquiatrica --- como toda semiologia medica --- e disciplina que se aprende:

\begin{itemize}
    \item \textbf{Aprende-se a observar:} o modo como o paciente entra na sala, senta-se, movimenta-se, fala, cala.
    \item \textbf{Aprende-se a perguntar:} nao apenas o que perguntar, mas como, quando, em que tom.
    \item \textbf{Aprende-se a ouvir:} distinguindo o dito do nao-dito, o conteudo manifesto das lacunas significativas.
    \item \textbf{Aprende-se a integrar:} relacionando historia, exame, contexto, em uma sintese que e mais que soma de partes.
\end{itemize}

\begin{conceitochave}[title={Requisitos da Entrevista Psiquiatrica}]
Segundo o \textit{American Board of Psychiatry and Neurology}:
\begin{itemize}
    \item \textbf{Tempo:} Minimo de 30 minutos apenas para a entrevista
    \item \textbf{Supervisao:} Avaliacao por pares treinados
    \item \textbf{Repeticao:} Aprendizado consolidado atraves da pratica sistematica
    \item \textbf{Consciencia:} Reconhecimento de que ha algo a aprender
\end{itemize}
\end{conceitochave}

%%%%%%%%%%%%%%%%%%%%%%%%%%%%%%%%%%%%%%%%%%%%%%%%%%%%%%%%%%%%%%%%%%%%%%%%%%%%%%%%%%
\section{A modestia socratica}
\label{sec:socrates}
\index{Socrates}

\autor{Socrates} ficou conhecido por afirmar que so sabia que nada sabia. A frase, frequentemente banalizada, contem uma verdade metodologica profunda: o conhecimento comeca pelo reconhecimento da ignorancia. Quem pensa que ja sabe nao pergunta; quem nao pergunta nao aprende; quem nao aprende repete erros.

A psiquiatria contemporanea, paradoxalmente, sofre de excesso de confianca:

\begin{itemize}
    \item Confianca em criterios que nao garantem o que prometem
    \item Confianca em diagnosticos emitidos em tempo insuficiente
    \item Confianca na capacidade de reconhecer padroes sem o treinamento necessario para reconhece-los
\end{itemize}

\begin{center}
\large\itshape
O rigor do diagnostico clinico nao e perfeccionismo academico, obstaculo burocratico, ou preciosismo de especialista. E condicao minima para nao causar dano. E a diferenca entre pratica medica e opiniao leiga. E, em ultima instancia, respeito pela pessoa que busca ajuda e confia que quem a avalia sabe o que esta fazendo.
\end{center}

%%%%%%%%%%%%%%%%%%%%%%%%%%%%%%%%%%%%%%%%%%%%%%%%%%%%%%%%%%%%%%%%%%%%%%%%%%%%%%%%%%
\begin{sintese}
\textbf{Ideias centrais deste capitulo:}
\begin{enumerate}
    \item A semiologia psiquiatrica e uma disciplina tecnica que se aprende.
    \item Criterios operacionais prometiam confiabilidade, mas na pratica frequentemente nao sao aplicados.
    \item A abordagem prototipica domina a clinica, com resultados diferentes da abordagem criteriologica.
    \item O sobrediagnostico tem consequencias reais e documentadas.
    \item Explicar (Erklaren) e compreender (Verstehen) sao irredutveis um ao outro.
    \item Diagnosticar e um ato de responsabilidade etica, nao apenas tecnica.
    \item A formacao adequada e condicao minima para o diagnostico legitimo.
\end{enumerate}

\vspace{0.5cm}
\textbf{Conceitos-chave:} Semiologia $\bullet$ Criterios operacionais $\bullet$ Confiabilidade inter-avaliadores $\bullet$ Sobrediagnostico $\bullet$ Erklaren vs. Verstehen $\bullet$ Modestia socratica
\end{sintese}

\vspace{1cm}

\begin{center}
\textbf{\large Sabe?}
\end{center}
