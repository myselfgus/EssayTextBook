% !TEX root = ../../BookTemplate.tex
%%%%%%%%%%%%%%%%%%%%%%%%%%%%%%%%%%%%%%%%%%%%%%%%%%%%%%%%%%%%%%%%%%%%%%%%%%%%%%%%%%
\chapter{Etiologia da Doenca Mental}
\label{cap:etiologia}

\begin{center}
\textit{``A causalidade em psiquiatria e `matizada' (dappled): multiplos fatores interagem ao longo do tempo, nenhum suficiente isoladamente, nenhum necessario absolutamente.''}\\[0.3cm]
--- Kenneth Kendler (2012)
\end{center}

\vspace{1cm}

%%%%%%%%%%%%%%%%%%%%%%%%%%%%%%%%%%%%%%%%%%%%%%%%%%%%%%%%%%%%%%%%%%%%%%%%%%%%%%%%%%
\section{Introducao: Onde Mora a Doenca?}
\label{sec:onde-mora}
\index{Etiologia}

A pergunta pela etiologia --- a causa, a origem, o fundamento --- de uma doenca mental parece, a primeira vista, pertencer ao dominio das ciencias naturais. Identificar o agente causador, localizar a lesao, determinar o mecanismo fisiopatologico: este e o programa da medicina cientifica desde Virchow e sua patologia celular. Entretanto, como observa \autor{Jaspers} (1913/1997), a psiquiatria ocupa posicao peculiar entre as disciplinas medicas, pois seu objeto --- a experiencia subjetiva alterada --- resiste as categorias explicativas que funcionam tao bem para tumores, infeccoes e fraturas.

\begin{questao}
Se perguntamos ``onde esta a pneumonia?'', a resposta e relativamente simples: nos pulmoes, nas celulas alveolares inflamadas, no tecido infiltrado por neutrofilos.

Mas se perguntamos ``onde esta a depressao?'' --- onde, precisamente, ela \textit{mora} ---, a resposta ja nao e tao evidente:

\begin{itemize}
    \item No cerebro? Em quais circuitos? Em quais neurotransmissores?
    \item Na historia de perdas acumuladas?
    \item Na infancia marcada por negligencia?
    \item No casamento que sufoca? No trabalho que aliena?
    \item Na sociedade que exclui?
    \item Na linguagem que nomeia o sofrimento?
\end{itemize}
\end{questao}

\begin{conceitochave}[title={Tese Central}]
\textbf{A etiologia de uma doenca mental nao pode ser localizada exclusivamente no interior do individuo que a manifesta.}

Esta proposicao, uma vez articulada, desestabiliza o edificio conceitual sobre o qual repousa boa parte da pratica psiquiatrica contemporanea.
\end{conceitochave}

%%%%%%%%%%%%%%%%%%%%%%%%%%%%%%%%%%%%%%%%%%%%%%%%%%%%%%%%%%%%%%%%%%%%%%%%%%%%%%%%%%
\section{A Temporalidade do Adoecer Mental}
\label{sec:temporalidade}
\index{Temporalidade}

\subsection{A Doenca Como Evento ou Como Processo?}

A medicina somatica frequentemente opera com modelo de doenca como \textit{evento} --- algo que \textit{acontece} ao organismo em momento determinado. O infarto ocorre as 14h37 de uma terca-feira; a fratura acontece no instante do impacto; a infeccao inicia quando o patogeno penetra as barreiras do hospedeiro.

\begin{observacao}
\textbf{Mas quando comeca uma depressao? Quando comeca uma esquizofrenia?}

\autor{Fuchs} (2012) observa que os transtornos mentais caracterizam-se por estrutura temporal radicalmente distinta. Nao ha o \textit{momento} em que alguem ``contrai'' depressao como se contrai gripe.
\end{observacao}

O que observamos e processo gradual de transformacao --- o que \autor{Minkowski} (1927/1970) denominou alteracao no ``tempo vivido'' (\textit{temps vecu}) --- que frequentemente so se torna visivel retrospectivamente.

\begin{figure}[H]
\centering
\begin{tikzpicture}[
    node distance=0.5cm,
    event/.style={rectangle, draw=BrickRed, fill=boxred, text width=3cm, text centered, minimum height=1cm, font=\small},
    process/.style={rectangle, draw=OliveGreen, fill=boxgreen, text width=8cm, text centered, minimum height=1cm, font=\small}
]
    \node[event] (evento) {\textbf{Doenca como Evento}\\Pontual, datavel};
    \node[process, below=1cm of evento] (processo) {\textbf{Doenca Mental como Processo}\\Gradual, retrospectivo, emergente};

    \node[right=0.5cm of evento, font=\footnotesize] {Infarto, fratura, infeccao};
    \node[right=0.5cm of processo, font=\footnotesize] {Depressao, esquizofrenia};
\end{tikzpicture}
\caption{Modelos temporais de adoecimento}
\label{fig:temporalidade}
\end{figure}

Se a doenca nao \textit{comeca} em momento identificavel, mas \textit{emerge} de processo prolongado, entao a busca por \textit{a} causa --- singular, pontual, determinante --- revela-se inadequada.

\subsection{A Doenca Tem Historia}

\autor{Ricoeur} (1990/1992) propoe que o \textit{si mesmo} nao e substancia fixa, mas configuracao temporal --- somos as historias que contamos sobre nos mesmos. Se o sujeito e narrativa, entao sua doenca tambem e narrativa.

\begin{casoclinico}[title={A Depressao de Maria}]
Uma mulher de 45 anos apresenta episodio depressivo grave. O psiquiatra documenta: humor deprimido, anedonia, insonia, ideacao suicida. Pode localizar alteracoes em circuitos neurais, prescrever antidepressivo.

Mas \textit{esta} depressao, \textit{desta} mulher, e tambem:
\begin{itemize}
    \item O pai alcoolatra que a humilhava na infancia
    \item O primeiro casamento marcado por violencia
    \item A demissao apos vinte anos de dedicacao
    \item Os filhos que se afastaram
    \item O corpo que envelhece em sociedade que valoriza juventude
    \item A solidao em cidade que nao reconhece
\end{itemize}

\textbf{Onde esta a etiologia desta depressao?} No deficit de serotonina? Na historia de trauma? Na estrutura economica? Nos valores culturais?
\end{casoclinico}

\subsection{Temporalidade e Causalidade Circular}
\index{Causalidade circular}

\autor{Bateson} (1972) observa que em fenomenos complexos encontramos \termotecnico{causalidade circular} --- A causa B que causa C que retroage sobre A, em loops que se auto-mantem.

\begin{figure}[H]
\centering
\begin{tikzpicture}[
    node distance=2.5cm,
    state/.style={ellipse, draw=chaptercolor, fill=boxblue, text width=2.5cm, text centered, minimum height=1.5cm, font=\small}
]
    \node[state] (isolamento) {Isolamento\\Social};
    \node[state, right=of isolamento] (depressao) {Depressao};
    \node[state, below right=1.5cm and 1.25cm of isolamento] (mais) {Mais\\Isolamento};

    \draw[-Stealth, thick] (isolamento) -- node[above, font=\footnotesize] {causa} (depressao);
    \draw[-Stealth, thick] (depressao) -- node[right, font=\footnotesize] {causa} (mais);
    \draw[-Stealth, thick] (mais) -- node[below, font=\footnotesize] {aprofunda} (isolamento);
\end{tikzpicture}
\caption{Causalidade circular em depressao}
\label{fig:causalidade-circular}
\end{figure}

\begin{observacao}
Em sistemas circulares, a ``causa'' e o ``efeito'' sao distincoes arbitrarias que dependem de onde o observador escolhe \textit{pontuar} a sequencia (\autor{Watzlawick} et al., 1967).

O marido diz que bebe porque a esposa reclama; a esposa diz que reclama porque o marido bebe. Ambos estao ``certos''.
\end{observacao}

%%%%%%%%%%%%%%%%%%%%%%%%%%%%%%%%%%%%%%%%%%%%%%%%%%%%%%%%%%%%%%%%%%%%%%%%%%%%%%%%%%
\section{A Contextualidade do Adoecer Mental}
\label{sec:contextualidade}
\index{Contexto}

\subsection{O Contexto Como Fundo Necessario}

\autor{Merleau-Ponty} (1945/2006) demonstra que toda figura so e figura contra um fundo --- nao existe percepcao de objeto isolado, mas sempre de objeto \textit{em} campo perceptivo que o circunda e o constitui.

\begin{definicaobox}[title={Disease vs. Illness (Kleinman)}]
\autor{Kleinman} (1988) distingue:
\begin{description}
    \item[\textit{Disease}:] A disfuncao biologica objetivavel
    \item[\textit{Illness}:] A experiencia vivida do adoecimento, culturalmente mediada
\end{description}

O mesmo processo biologico pode constituir experiencias radicalmente diferentes dependendo do contexto cultural, social e biografico.
\end{definicaobox}

A ``mesma'' depressao e vivida diferentemente pelo executivo paulistano e pelo agricultor nordestino, pela adolescente e pela idosa, pelo heterossexual e pelo homossexual em sociedade homofobica.

\subsection{Contextos Patogenicos}
\index{Determinantes sociais}

Se o contexto constitui a doenca, entao certos contextos sao \termotecnico{patogenicos} --- nao no sentido de conterem ``germes'' psiquicos, mas no sentido de criarem condicoes nas quais o adoecimento torna-se mais provavel.

\begin{table}[H]
\centering
\caption{Contextos patogenicos documentados}
\label{tab:contextos}
\begin{tabular}{p{4cm}p{7cm}}
\toprule
\textbf{Contexto} & \textbf{Efeito Documentado} \\
\midrule
Pobreza & Aumento de prevalencia de transtornos mentais \\
Desigualdade & Afeta toda a populacao, nao apenas pobres \\
Discriminacao & Estresse cronico, trauma \\
Violencia & TEPT, depressao, ansiedade \\
Precarizacao do trabalho & Burnout, ansiedade, depressao \\
\bottomrule
\end{tabular}
\end{table}

\autor{Wilkinson e Pickett} (2009) demonstram que sociedades mais desiguais apresentam prevalencia maior de transtornos mentais --- a desigualdade e toxico social que afeta o tecido relacional como um todo.

\begin{questao}
\textbf{A Pergunta Incomoda}

Se o contexto e patogenico, por que tratamos apenas o individuo?

A resposta habitual --- que nao podemos mudar a sociedade, mas podemos tratar o paciente --- e pragmaticamente compreensivel, mas epistemologicamente insustentavel.

\textit{E como se, diante de epidemia de colera, a medicina se limitasse a reidratar os doentes sem questionar a qualidade da agua.}
\end{questao}

%%%%%%%%%%%%%%%%%%%%%%%%%%%%%%%%%%%%%%%%%%%%%%%%%%%%%%%%%%%%%%%%%%%%%%%%%%%%%%%%%%
\section{O Locus Etiologico: Onde Esta a Causa?}
\label{sec:locus}

\begin{figure}[H]
\centering
\begin{tikzpicture}[
    node distance=1.5cm,
    model/.style={rectangle, draw, text width=3.5cm, text centered, rounded corners, minimum height=2cm, font=\small}
]
    \node[model, draw=BrickRed, fill=boxred] (bio) at (0,0) {\textbf{Modelo Biomedico}\\Etiologia no EU\\(Cerebro)};
    \node[model, draw=OliveGreen, fill=boxgreen] (rel) at (4.5,0) {\textbf{Modelo Relacional}\\Etiologia no OUTRO\\(Relacoes)};
    \node[model, draw=BurntOrange, fill=boxorange] (soc) at (0,-3.5) {\textbf{Modelo Critico}\\Etiologia na SOCIEDADE\\(Estruturas)};
    \node[model, draw=chaptercolor, fill=boxblue] (disc) at (4.5,-3.5) {\textbf{Modelo Discursivo}\\Etiologia na LINGUAGEM\\(Discursos)};
\end{tikzpicture}
\caption{Quatro perspectivas sobre o locus etiologico}
\label{fig:locus}
\end{figure}

\subsection{A Etiologia no Eu: O Modelo Biomedico}
\index{Modelo biomedico}

O modelo biomedico localiza a etiologia no organismo individual --- transtornos mentais seriam ``doencas do cerebro'' causadas por disfuncoes em circuitos neurais especificos.

\begin{conceitochave}[title={Meritos e Limites do Modelo Biomedico}]
\textbf{Meritos:}
\begin{itemize}
    \item Legitimou o sofrimento mental como fenomeno ``real''
    \item Avancos psicofarmacologicos aliviaram milhoes
\end{itemize}

\textbf{Limites:}
\begin{itemize}
    \item Reducao epistemica: niveis superiores (experiencia, significado) ``explicados'' por inferiores (neuronios, moleculas)
    \item Correlacao nao e causacao
    \item A tristeza correlaciona-se com padroes cerebrais, mas pode ser \textit{causada} pela perda
\end{itemize}
\end{conceitochave}

\autor{Kirmayer e Gold} (2012) denominam essa tendencia de ``neurofagia'' --- que devora dimensoes do fenomeno irredutiveis a mecanismos neurais.

\subsection{A Etiologia no Outro: O Modelo Relacional}
\index{Teoria do apego}
\index{Winnicott}

\autor{Winnicott} (1960/1983), com sua celebre observacao de que ``nao existe bebe'' --- apenas diade mae-bebe ---, fundamenta esta perspectiva. O self \textit{emerge} das relacoes, e constituido pelo cuidado (ou negligencia) do outro.

\begin{figure}[H]
\centering
\begin{tikzpicture}[
    node distance=1.2cm,
    pattern/.style={rectangle, draw=sectioncolor, fill=boxblue, text width=3cm, text centered, minimum height=1cm, font=\small}
]
    \node[pattern] (seguro) {Apego Seguro};
    \node[pattern, right=of seguro] (ansioso) {Apego Ansioso};
    \node[pattern, right=of ansioso] (evitativo) {Apego Evitativo};
    \node[pattern, right=of evitativo] (desorg) {Desorganizado};

    \node[below=1cm of $(ansioso)!0.5!(evitativo)$, text width=10cm, text centered, font=\small\itshape] {Padroes formados na infancia predizem funcionamento ao longo da vida (Bowlby)};
\end{tikzpicture}
\caption{Padroes de apego e suas origens relacionais}
\label{fig:apego}
\end{figure}

\begin{observacao}
Se a etiologia esta nas relacoes, o tratamento deve ser relacional. A relacao terapeutica nao e mero veiculo para tecnicas; e, ela mesma, o agente de cura.
\end{observacao}

\subsection{A Etiologia na Sociedade: O Modelo Critico}
\index{Foucault}
\index{Antipsiquiatria}

\autor{Foucault} (1961/2006) demonstra que as categorias psiquiatricas nao sao descobertas neutras de entidades naturais, mas construcoes historicas inseparaveis de dispositivos de poder.

\begin{questao}
Considere-se a epidemia contemporanea de ``transtornos de ansiedade''.

Seria coincidencia que esta epidemia ocorra precisamente em sociedades caracterizadas por:
\begin{itemize}
    \item Precarizacao do trabalho
    \item Erosao de vinculos comunitarios
    \item Competicao generalizada
    \item Exposicao constante a noticias ameacadoras
    \item Demandas impossiveis de desempenho
\end{itemize}

Ou estamos medicalizando --- e despolitizando --- respostas razoaveis a condicoes objetivamente ansiogenicas?
\end{questao}

\subsection{A Etiologia na Linguagem: O Modelo Discursivo}
\index{Lacan}
\index{Performatividade}

\autor{Lacan} (1966/2006) propoe que ``o inconsciente e estruturado como linguagem''. Nao somos sujeitos que \textit{usam} linguagem; somos sujeitos \textit{constituidos} pela linguagem.

\autor{Butler} (1990/2003) demonstra que categorias como ``masculino'' e ``feminino'' sao performativamente constituidas. A menina que nao performa feminilidade ``adequadamente'' esta sendo constituida como ``desviante'' por discursos que definem o normal.

\begin{observacao}
O sofrimento que resulta desta constituicao discursiva --- disforia de genero, vergonha do corpo, sensacao de nao pertencer --- e real. Mas sua etiologia esta nos discursos normativos que definem certos corpos como patologicos.
\end{observacao}

%%%%%%%%%%%%%%%%%%%%%%%%%%%%%%%%%%%%%%%%%%%%%%%%%%%%%%%%%%%%%%%%%%%%%%%%%%%%%%%%%%
\section{Implicacoes da Deslocalizacao Etiologica}
\label{sec:implicacoes}

\begin{table}[H]
\centering
\caption{Implicacoes da deslocalizacao etiologica}
\label{tab:implicacoes}
\begin{tabular}{p{3cm}p{8.5cm}}
\toprule
\textbf{Dominio} & \textbf{Implicacao} \\
\midrule
Diagnostico & Deve incluir avaliacao do contexto: relacoes, condicoes sociais, discursos constituintes \\
\addlinespace
Tratamento & Deve ser distribuido: intervencao farmacologica + psicoterapia + acoes comunitarias + mudanca social \\
\addlinespace
Responsabilidade & Redistribuida: recusa da culpabilizacao do individuo por causas que estao fora dele \\
\addlinespace
Etica/Politica & Prevencao requer transformacao social, nao apenas melhores servicos \\
\bottomrule
\end{tabular}
\end{table}

%%%%%%%%%%%%%%%%%%%%%%%%%%%%%%%%%%%%%%%%%%%%%%%%%%%%%%%%%%%%%%%%%%%%%%%%%%%%%%%%%%
\section{Conclusao: A Etiologia Como Questao Aberta}
\label{sec:conclusao-etiologia}

Iniciamos este capitulo com pergunta aparentemente simples: onde esta a causa da doenca mental? Percorremos possiveis respostas --- no Eu, no Outro, na Sociedade, na Linguagem --- e encontramos, em cada caso, verdade parcial e limitacao.

\begin{sintese}
\textbf{Ideias centrais deste capitulo:}
\begin{enumerate}
    \item A doenca mental e processo temporal, nao evento pontual.
    \item O contexto nao e cenario; o contexto \textit{constitui} a doenca.
    \item A etiologia e \textbf{distribuida}: cerebro, relacoes, sociedade, linguagem.
    \item Tratar apenas o individuo e epistemologicamente insustentavel.
    \item A causalidade circular torna a busca por ``a causa'' frequentemente infrutifera.
    \item Reconhecer causas externas redistribui responsabilidade e despolitiza o sofrimento.
\end{enumerate}

\vspace{0.5cm}
\textbf{Conceitos-chave:} Temporalidade $\bullet$ Causalidade circular $\bullet$ Contextos patogenicos $\bullet$ Disease vs. Illness $\bullet$ Deslocalizacao etiologica
\end{sintese}

\vspace{1cm}

\begin{center}
\large\itshape
Se a doenca mental tem sua origem, ao menos parcialmente, fora do individuo que a manifesta, entao a tarefa do clinico nao e apenas tratar o paciente, mas tambem testemunhar a injustica que o adoece. E testemunhar, como nos ensina a tradicao das comissoes de verdade, e ja uma forma de cura.
\end{center}
