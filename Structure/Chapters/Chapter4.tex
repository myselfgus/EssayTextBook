% !TEX root = ../../BookTemplate.tex
%%%%%%%%%%%%%%%%%%%%%%%%%%%%%%%%%%%%%%%%%%%%%%%%%%%%%%%%%%%%%%%%%%%%%%%%%%%%%%%%%%
\chapter{Etiologia da Doença Mental}
\label{cap:etiologia}

\begin{center}
\textit{``A causalidade em psiquiatria é `matizada' (dappled): múltiplos fatores interagem ao longo do tempo, nenhum suficiente isoladamente, nenhum necessário absolutamente.''}\\[0.3cm]
--- Kenneth Kendler (2012)
\end{center}

\vspace{1cm}

%%%%%%%%%%%%%%%%%%%%%%%%%%%%%%%%%%%%%%%%%%%%%%%%%%%%%%%%%%%%%%%%%%%%%%%%%%%%%%%%%%
\section{Introdução: Onde Mora a Doença?}
\label{sec:onde-mora}
\index{Etiologia}

A pergunta pela etiologia --- a causa, a origem, o fundamento --- de uma doença mental parece, à primeira vista, pertencer ao domínio das ciências naturais. Identificar o agente causador, localizar a lesão, determinar o mecanismo fisiopatológico: este é o programa da medicina científica desde Virchow e sua patologia celular. Entretanto, como observa \autor{Jaspers} (1913/1997), a psiquiatria ocupa posição peculiar entre as disciplinas médicas, pois seu objeto --- a experiência subjetiva alterada --- resiste às categorias explicativas que funcionam tão bem para tumores, infecções e fraturas.

\begin{questao}
Se perguntamos ``onde está a pneumonia?'', a resposta é relativamente simples: nos pulmões, nas células alveolares inflamadas, no tecido infiltrado por neutrófilos.

Mas se perguntamos ``onde está a depressão?'' --- onde, precisamente, ela \textit{mora} ---, a resposta já não é tão evidente:

\begin{itemize}
    \item No cérebro? Em quais circuitos? Em quais neurotransmissores?
    \item Na história de perdas acumuladas?
    \item Na infância marcada por negligência?
    \item No casamento que sufoca? No trabalho que aliena?
    \item Na sociedade que exclui?
    \item Na linguagem que nomeia o sofrimento de maneiras que o amplificam ou atenuam?
\end{itemize}
\end{questao}

Este capítulo propõe uma investigação sobre o \textit{locus} etiológico das condições que chamamos ``doenças mentais''. Argumenta-se que a resposta a esta pergunta --- aparentemente técnica, aparentemente neutra --- carrega implicações profundas sobre como concebemos o sofrimento psíquico, como o tratamos, e como distribuímos responsabilidades entre o indivíduo que sofre e o mundo que o constitui.

\begin{conceitochave}[title={Tese Central}]
\textbf{A etiologia de uma doença mental não pode ser localizada exclusivamente no interior do indivíduo que a manifesta.}

Esta proposição, uma vez articulada, desestabiliza o edifício conceitual sobre o qual repousa boa parte da prática psiquiátrica contemporânea.
\end{conceitochave}

%%%%%%%%%%%%%%%%%%%%%%%%%%%%%%%%%%%%%%%%%%%%%%%%%%%%%%%%%%%%%%%%%%%%%%%%%%%%%%%%%%
\section{A Temporalidade do Adoecer Mental}
\label{sec:temporalidade}
\index{Temporalidade}

\subsection{A Doença Como Evento ou Como Processo?}

A medicina somática frequentemente opera com modelo de doença como \textit{evento} --- algo que \textit{acontece} ao organismo em momento determinado. O infarto ocorre às 14h37 de uma terça-feira; a fratura acontece no instante do impacto; a infecção inicia quando o patógeno penetra as barreiras do hospedeiro.

\begin{observacao}
\textbf{Mas quando começa uma depressão? Quando começa uma esquizofrenia?}

\autor{Fuchs} (2012) observa que os transtornos mentais caracterizam-se por estrutura temporal radicalmente distinta. Não há o \textit{momento} em que alguém ``contrai'' depressão como se contrai gripe.
\end{observacao}

O que observamos é processo gradual de transformação --- o que \autor{Minkowski} (1927/1970) denominou alteração no ``tempo vivido'' (\textit{temps vécu}) --- que frequentemente só se torna visível retrospectivamente.

\begin{figure}[H]
\centering
\begin{tikzpicture}[
    node distance=0.5cm,
    event/.style={rectangle, draw=BrickRed, fill=boxred, text width=3cm, text centered, minimum height=1cm, font=\small},
    process/.style={rectangle, draw=OliveGreen, fill=boxgreen, text width=8cm, text centered, minimum height=1cm, font=\small}
]
    \node[event] (evento) {\textbf{Doença como Evento}\\Pontual, datável};
    \node[process, below=1cm of evento] (processo) {\textbf{Doença Mental como Processo}\\Gradual, retrospectivo, emergente};

    \node[right=0.5cm of evento, font=\footnotesize] {Infarto, fratura, infecção};
    \node[right=0.5cm of processo, font=\footnotesize] {Depressão, esquizofrenia};
\end{tikzpicture}
\caption{Modelos temporais de adoecimento}
\label{fig:temporalidade}
\end{figure}

Se a doença não \textit{começa} em momento identificável, mas \textit{emerge} de processo prolongado, então a busca por \textit{a} causa --- singular, pontual, determinante --- revela-se inadequada.

\subsection{A Doença Tem História}

\autor{Ricoeur} (1990/1992) propõe que o \textit{si mesmo} não é substância fixa, mas configuração temporal --- somos as histórias que contamos sobre nós mesmos. Se o sujeito é narrativa, então sua doença também é narrativa.

\begin{casoclinico}[title={A Depressão de Maria}]
Uma mulher de 45 anos apresenta episódio depressivo grave. O psiquiatra documenta: humor deprimido, anedonia, insônia, ideação suicida. Pode localizar alterações em circuitos neurais, prescrever antidepressivo.

Mas \textit{esta} depressão, \textit{desta} mulher, é também:
\begin{itemize}
    \item O pai alcoólatra que a humilhava na infância
    \item O primeiro casamento marcado por violência
    \item A demissão após vinte anos de dedicação
    \item Os filhos que se afastaram
    \item O corpo que envelhece em sociedade que valoriza juventude
    \item A solidão em cidade que não reconhece
\end{itemize}

\textbf{Onde está a etiologia desta depressão?} No déficit de serotonina? Na história de trauma? Na estrutura econômica? Nos valores culturais?
\end{casoclinico}

\subsection{Temporalidade e Causalidade Circular}
\index{Causalidade circular}

\autor{Bateson} (1972) observa que em fenômenos complexos encontramos \termotecnico{causalidade circular} --- A causa B que causa C que retroage sobre A, em loops que se auto-mantêm.

\begin{figure}[H]
\centering
\begin{tikzpicture}[
    node distance=2.5cm,
    state/.style={ellipse, draw=chaptercolor, fill=boxblue, text width=2.5cm, text centered, minimum height=1.5cm, font=\small}
]
    \node[state] (isolamento) {Isolamento\\Social};
    \node[state, right=of isolamento] (depressao) {Depressão};
    \node[state, below right=1.5cm and 1.25cm of isolamento] (mais) {Mais\\Isolamento};

    \draw[-Stealth, thick] (isolamento) -- node[above, font=\footnotesize] {causa} (depressao);
    \draw[-Stealth, thick] (depressao) -- node[right, font=\footnotesize] {causa} (mais);
    \draw[-Stealth, thick] (mais) -- node[below, font=\footnotesize] {aprofunda} (isolamento);
\end{tikzpicture}
\caption{Causalidade circular em depressão}
\label{fig:causalidade-circular}
\end{figure}

\begin{observacao}
Em sistemas circulares, a ``causa'' e o ``efeito'' são distinções arbitrárias que dependem de onde o observador escolhe \textit{pontuar} a sequência (\autor{Watzlawick} et al., 1967).

O marido diz que bebe porque a esposa reclama; a esposa diz que reclama porque o marido bebe. Ambos estão ``certos''.
\end{observacao}

%%%%%%%%%%%%%%%%%%%%%%%%%%%%%%%%%%%%%%%%%%%%%%%%%%%%%%%%%%%%%%%%%%%%%%%%%%%%%%%%%%
\section{A Contextualidade do Adoecer Mental}
\label{sec:contextualidade}
\index{Contexto}

\subsection{O Contexto Como Fundo Necessário}

\autor{Merleau-Ponty} (1945/2006) demonstra que toda figura só é figura contra um fundo --- não existe percepção de objeto isolado, mas sempre de objeto \textit{em} campo perceptivo que o circunda e o constitui.

\begin{definicaobox}[title={Disease vs. Illness (Kleinman)}]
\autor{Kleinman} (1988) distingue:
\begin{description}
    \item[\textit{Disease}:] A disfunção biológica objetivável
    \item[\textit{Illness}:] A experiência vivida do adoecimento, culturalmente mediada
\end{description}

O mesmo processo biológico pode constituir experiências radicalmente diferentes dependendo do contexto cultural, social e biográfico.
\end{definicaobox}

A ``mesma'' depressão é vivida diferentemente pelo executivo paulistano e pelo agricultor nordestino, pela adolescente e pela idosa, pelo heterossexual e pelo homossexual em sociedade homofóbica.

\subsection{Contextos Patogênicos}
\index{Determinantes sociais}

Se o contexto constitui a doença, então certos contextos são \termotecnico{patogênicos} --- não no sentido de conterem ``germes'' psíquicos, mas no sentido de criarem condições nas quais o adoecimento torna-se mais provável.

\begin{table}[H]
\centering
\caption{Contextos patogênicos documentados}
\label{tab:contextos}
\begin{tabular}{p{4cm}p{7cm}}
\toprule
\textbf{Contexto} & \textbf{Efeito Documentado} \\
\midrule
Pobreza & Aumento de prevalência de transtornos mentais \\
Desigualdade & Afeta toda a população, não apenas pobres \\
Discriminação & Estresse crônico, trauma \\
Violência & TEPT, depressão, ansiedade \\
Precarização do trabalho & Burnout, ansiedade, depressão \\
\bottomrule
\end{tabular}
\end{table}

\autor{Wilkinson e Pickett} (2009) demonstram que sociedades mais desiguais apresentam prevalência maior de transtornos mentais --- a desigualdade é tóxico social que afeta o tecido relacional como um todo.

\begin{questao}
\textbf{A Pergunta Incômoda}

Se o contexto é patogênico, por que tratamos apenas o indivíduo?

A resposta habitual --- que não podemos mudar a sociedade, mas podemos tratar o paciente --- é pragmaticamente compreensível, mas epistemologicamente insustentável.

\textit{É como se, diante de epidemia de cólera, a medicina se limitasse a reidratar os doentes sem questionar a qualidade da água.}
\end{questao}

%%%%%%%%%%%%%%%%%%%%%%%%%%%%%%%%%%%%%%%%%%%%%%%%%%%%%%%%%%%%%%%%%%%%%%%%%%%%%%%%%%
\section{O Locus Etiológico: Onde Está a Causa?}
\label{sec:locus}

\begin{figure}[H]
\centering
\begin{tikzpicture}[
    node distance=1.5cm,
    model/.style={rectangle, draw, text width=3.5cm, text centered, rounded corners, minimum height=2cm, font=\small}
]
    \node[model, draw=BrickRed, fill=boxred] (bio) at (0,0) {\textbf{Modelo Biomédico}\\Etiologia no EU\\(Cérebro)};
    \node[model, draw=OliveGreen, fill=boxgreen] (rel) at (4.5,0) {\textbf{Modelo Relacional}\\Etiologia no OUTRO\\(Relações)};
    \node[model, draw=BurntOrange, fill=boxorange] (soc) at (0,-3.5) {\textbf{Modelo Crítico}\\Etiologia na SOCIEDADE\\(Estruturas)};
    \node[model, draw=chaptercolor, fill=boxblue] (disc) at (4.5,-3.5) {\textbf{Modelo Discursivo}\\Etiologia na LINGUAGEM\\(Discursos)};
\end{tikzpicture}
\caption{Quatro perspectivas sobre o locus etiológico}
\label{fig:locus}
\end{figure}

\subsection{A Etiologia no Eu: O Modelo Biomédico}
\index{Modelo biomédico}

O modelo biomédico localiza a etiologia no organismo individual --- transtornos mentais seriam ``doenças do cérebro'' causadas por disfunções em circuitos neurais específicos.

\begin{conceitochave}[title={Méritos e Limites do Modelo Biomédico}]
\textbf{Méritos:}
\begin{itemize}
    \item Legitimou o sofrimento mental como fenômeno ``real''
    \item Avanços psicofarmacológicos aliviaram milhões
\end{itemize}

\textbf{Limites:}
\begin{itemize}
    \item Redução epistêmica: níveis superiores (experiência, significado) ``explicados'' por inferiores (neurônios, moléculas)
    \item Correlação não é causação
    \item A tristeza correlaciona-se com padrões cerebrais, mas pode ser \textit{causada} pela perda
\end{itemize}
\end{conceitochave}

\autor{Kirmayer e Gold} (2012) denominam essa tendência de ``neurofagia'' --- que devora dimensões do fenômeno irredutíveis a mecanismos neurais.

\subsection{A Etiologia no Outro: O Modelo Relacional}
\index{Teoria do apego}
\index{Winnicott}

\autor{Winnicott} (1960/1983), com sua célebre observação de que ``não existe bebê'' --- apenas díade mãe-bebê ---, fundamenta esta perspectiva. O self \textit{emerge} das relações, é constituído pelo cuidado (ou negligência) do outro.

\begin{figure}[H]
\centering
\begin{tikzpicture}[
    node distance=1.2cm,
    pattern/.style={rectangle, draw=sectioncolor, fill=boxblue, text width=3cm, text centered, minimum height=1cm, font=\small}
]
    \node[pattern] (seguro) {Apego Seguro};
    \node[pattern, right=of seguro] (ansioso) {Apego Ansioso};
    \node[pattern, right=of ansioso] (evitativo) {Apego Evitativo};
    \node[pattern, right=of evitativo] (desorg) {Desorganizado};

    \node[below=1cm of $(ansioso)!0.5!(evitativo)$, text width=10cm, text centered, font=\small\itshape] {Padrões formados na infância predizem funcionamento ao longo da vida (Bowlby)};
\end{tikzpicture}
\caption{Padrões de apego e suas origens relacionais}
\label{fig:apego}
\end{figure}

\begin{observacao}
Se a etiologia está nas relações, o tratamento deve ser relacional. A relação terapêutica não é mero veículo para técnicas; é, ela mesma, o agente de cura.
\end{observacao}

\subsection{A Etiologia na Sociedade: O Modelo Crítico}
\index{Foucault}
\index{Antipsiquiatria}

\autor{Foucault} (1961/2006) demonstra que as categorias psiquiátricas não são descobertas neutras de entidades naturais, mas construções históricas inseparáveis de dispositivos de poder.

\begin{questao}
Considere-se a epidemia contemporânea de ``transtornos de ansiedade''.

Seria coincidência que esta epidemia ocorra precisamente em sociedades caracterizadas por:
\begin{itemize}
    \item Precarização do trabalho
    \item Erosão de vínculos comunitários
    \item Competição generalizada
    \item Exposição constante a notícias ameaçadoras
    \item Demandas impossíveis de desempenho
\end{itemize}

Ou estamos medicalizando --- e despolitizando --- respostas razoáveis a condições objetivamente ansiogênicas?
\end{questao}

\subsection{A Etiologia na Linguagem: O Modelo Discursivo}
\index{Lacan}
\index{Performatividade}

\autor{Lacan} (1966/2006) propõe que ``o inconsciente é estruturado como linguagem''. Não somos sujeitos que \textit{usam} linguagem; somos sujeitos \textit{constituídos} pela linguagem.

\autor{Butler} (1990/2003) demonstra que categorias como ``masculino'' e ``feminino'' são performativamente constituídas. A menina que não performa feminilidade ``adequadamente'' está sendo constituída como ``desviante'' por discursos que definem o normal.

\begin{observacao}
O sofrimento que resulta desta constituição discursiva --- disforia de gênero, vergonha do corpo, sensação de não pertencer --- é real. Mas sua etiologia está nos discursos normativos que definem certos corpos como patológicos.
\end{observacao}

%%%%%%%%%%%%%%%%%%%%%%%%%%%%%%%%%%%%%%%%%%%%%%%%%%%%%%%%%%%%%%%%%%%%%%%%%%%%%%%%%%
\section{As Implicações da Deslocalização Etiológica}
\label{sec:implicacoes}

\subsection{Implicações para o Diagnóstico}

Se a etiologia não está (apenas) no indivíduo, então o diagnóstico não pode (apenas) catalogar sintomas individuais. A avaliação clínica deve incluir o que \autor{Kleinman} (1988) denomina ``modelo explicativo'' do paciente --- como este compreende seu próprio sofrimento, que causas atribui, que significados confere.

Mas deve incluir também avaliação do contexto: Quais relações sustentam ou adoecem este paciente? Quais condições sociais contribuem para seu sofrimento? Quais discursos o constituíram como ``doente'' ou ``desviante''? Sem estas informações, o diagnóstico é, na melhor das hipóteses, incompleto; na pior, mistificação que atribui ao indivíduo causas que estão fora dele.

\subsection{Implicações para o Tratamento}

Se a etiologia está distribuída entre indivíduo, relações, sociedade e discursos, então o tratamento também deve ser distribuído. A intervenção farmacológica pode ser necessária, mas não é suficiente. A psicoterapia individual pode ser útil, mas não basta se as relações do paciente permanecem tóxicas. A mudança individual pode ser desejável, mas não é justa se a sociedade que adoece permanece intocada.

Esta perspectiva fundamenta o que tem sido chamado de ``psiquiatria social'' ou ``saúde mental coletiva'' --- abordagens que reconhecem a insuficiência de intervenções exclusivamente individuais e propõem ações em múltiplos níveis: clínico, comunitário, institucional, político.

\subsection{Implicações para a Responsabilidade}

A pergunta pela etiologia é também, implicitamente, pergunta pela \textit{responsabilidade}. Se a doença está no indivíduo, então é ele quem deve mudar, adaptar-se, curar-se. Se a doença está nas relações, na sociedade, nos discursos, então a responsabilidade é compartilhada --- e exigir que apenas o indivíduo mude é, além de ineficaz, injusto.

Esta redistribuição de responsabilidade não exime o indivíduo de participar de seu processo de cura --- a agência pessoal permanece relevante. Mas recusa a \textit{culpabilização} que frequentemente acompanha o adoecimento mental: a sugestão, implícita ou explícita, de que o paciente é responsável por sua condição, de que poderia ``superar'' se quisesse, de que seu sofrimento reflete fraqueza moral ou escolha equivocada.

\subsection{Implicações Éticas e Políticas}

A deslocalização etiológica tem, finalmente, implicações éticas e políticas que transcendem a clínica. Se reconhecemos que condições sociais produzem sofrimento mental, então a prevenção deste sofrimento requer transformação social --- não apenas melhores serviços de saúde mental, mas também redução da desigualdade, proteção contra violência, garantia de direitos, questionamento de discursos normativos opressivos.

\begin{table}[H]
\centering
\caption{Implicações da deslocalização etiológica}
\label{tab:implicacoes}
\begin{tabular}{p{3cm}p{8.5cm}}
\toprule
\textbf{Domínio} & \textbf{Implicação} \\
\midrule
Diagnóstico & Deve incluir avaliação do contexto: relações, condições sociais, discursos constituintes \\
Tratamento & Deve ser distribuído: intervenção farmacológica + psicoterapia + ações comunitárias + mudança social \\
Responsabilidade & Redistribuída: recusa da culpabilização do indivíduo por causas que estão fora dele \\
Ética/Política & Prevenção requer transformação social, não apenas melhores serviços \\
\bottomrule
\end{tabular}
\end{table}

%%%%%%%%%%%%%%%%%%%%%%%%%%%%%%%%%%%%%%%%%%%%%%%%%%%%%%%%%%%%%%%%%%%%%%%%%%%%%%%%%%
\section{Conclusão: A Etiologia Como Questão Aberta}
\label{sec:conclusao-etiologia}

Iniciamos este capítulo com pergunta aparentemente simples: onde está a causa da doença mental? Percorremos possíveis respostas --- no Eu, no Outro, na Sociedade, na Linguagem --- e encontramos, em cada caso, verdade parcial e limitação.

A resposta honesta, talvez, seja que a etiologia da doença mental é \textbf{distribuída} --- não localizável em ponto único, mas dispersa através de múltiplos níveis de organização que interagem de maneiras complexas. O cérebro importa, certamente --- mas o cérebro existe em corpo, o corpo existe em relações, as relações existem em sociedade, a sociedade existe em história, a história existe em linguagem. Cortar arbitrariamente esta cadeia --- declarar que a ``verdadeira'' causa está em um nível específico --- é decisão metodológica, não descoberta empírica.

\begin{sintese}
\textbf{Ideias centrais deste capítulo:}
\begin{enumerate}
    \item A doença mental é processo temporal, não evento pontual.
    \item O contexto não é cenário; o contexto \textit{constitui} a doença.
    \item A etiologia é \textbf{distribuída}: cérebro, relações, sociedade, linguagem.
    \item Tratar apenas o indivíduo é epistemologicamente insustentável.
    \item A causalidade circular torna a busca por ``a causa'' frequentemente infrutífera.
    \item Reconhecer causas externas redistribui responsabilidade e despolitiza o sofrimento.
\end{enumerate}

\vspace{0.5cm}
\textbf{Conceitos-chave:} Temporalidade $\bullet$ Causalidade circular $\bullet$ Contextos patogênicos $\bullet$ Disease vs. Illness $\bullet$ Deslocalização etiológica
\end{sintese}

\vspace{1cm}

\begin{center}
\large\itshape
Se a doença mental tem sua origem, ao menos parcialmente, fora do indivíduo que a manifesta, então a tarefa do clínico não é apenas tratar o paciente, mas também testemunhar a injustiça que o adoece. E testemunhar, como nos ensina a tradição das comissões de verdade, é já uma forma de cura.
\end{center}
